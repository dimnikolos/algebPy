\chapter{Εξισώσεις και προβλήματα}

Σε αυτό το κεφάλαιο θα χρησιμοποιήσουμε τη βιβλιοθήκη sympy.
Υπάρχει ένα περιβάλλον στο οποίο μπορούμε να πληκτρολογούμε εντολές της βιβλιοθήκης ώστε να βλέπουμε τα αποτελέσματα με φιλικό τρόπο στον φυλλομετρήτή μας, συνήθως Chrome, Firefox ή Microsoft Edge. Το περιβάλλον αυτό βρίσκεται στη διεύθυνση https://live.sympy.org/. Μπορούμε να κάνουμε τα ίδια παραδείγματα στον Mu Editor όπως έχουμε συνηθίσει χρησιμοποιώντας την εντολή:
\begin{lstlisting}
from sympy import *
\end{lstlisting}
όμως τα αποτελέσματα δεν θα εμφανίζονται με φιλικό τρόπο αλλά με τον συμβολισμό της Python.
\section{Η έννοια της εξίσωσης}
\begin{exercise}
Γράψε συντομότερα τις εκφράσεις:

(α) $x + x + x + x$, 

(β) $\alpha + \alpha + \alpha + \beta + \beta$, 

(γ) $3\cdot \alpha + 5 \cdot \alpha$, 

(δ) $18 \cdot x + 7 \cdot x + 4 \cdot x$, 

(ε) $15 \cdot \beta – 9 \cdot \beta$.
\end{exercise}

Επειδή τα σύμβολα είναι τα $x, a, b$ θα πρέπει να τα δηλώσουμε στο sympy. Αυτό γίνεται ως εξής:
\begin{lstlisting}
from sympy import *
x,a,b = symbols("x a b")
\end{lstlisting}

Στη συνέχεια όποτε αναφέρουμε τα $x, a, b$ η Python θα καταλαβαίνει ότι πρόκειται για σύμβολα και θα δρα ανάλογα.
Έτσι αν δώσουμε στην Python
\begin{lstlisting}
>>> x + x + x + x
\end{lstlisting}
Θα μας δώσει ως απάντηση
$$4x$$
στο live.sympy.org
και
\begin{lstlisting}
4*x
\end{lstlisting}
στην απλή Python ή στο Mu Editor.
Άρα το 
\begin{lstlisting}
>>> a + a + a + b + b
\end{lstlisting}
Θα μας δώσει σαν απάντηση:
$$3a+2b$$
και τα 
\begin{lstlisting}
>>> 3*a + 5*a 
>>> 18*x + 7*x + 4*x
>>> 15*b - 9b
\end{lstlisting}
$$8a$$
$$29x$$
$$6b$$
αντίστοιχα.
\begin{exercise}
Να αντικαταστήσεις το x, με τους αριθμούς 1, 3, 4, 5, 6 και 11, σε κάθε ισότητα της πρώτης στήλης, του παρακάτω πίνακα. Βρες ποιος από αυτούς την επαληθεύει και ποιος όχι.

\begin{tabular}{|c|c|c|}
Εξίσωση            &Αριθμοί που την επαληθεύουν     &Αριθμοί που δεν την επαληθεύουν\\
$x – 4 = 1$        &                             &                                \\
$5 – x = 4$        &                             &                                \\
$2x = 8$           &                             &                                \\
$\frac{6}{x} = 2$  &                             &                                \\
$\frac{x}{2} = 3$  &                             &                                \\
$x + 7 = 30$       &                             &                                \\
\end{tabular}
\end{exercise}

\begin{lstlisting}
>>> e = x - 4
>>> e.subs(x,1)
\end{lstlisting}
$$−3$$
\begin{lstlisting}
>>> e.subs(x,3)
\end{lstlisting}
$$−1$$
\begin{lstlisting}
>>> e.subs(x,4)
\end{lstlisting}
$$0$$
\begin{lstlisting}
>>> e.subs(x,5)
\end{lstlisting}
$$1$$
\begin{lstlisting}
>>> e.subs(x,6)
\end{lstlisting}
$$2$$
\begin{lstlisting}
>>> e.subs(x,11)
\end{lstlisting}
$$7$$

Οπότε ο αριθμός που την επαληθεύει είναι ο 5 και όλοι οι υπόλοιποι δεν την επαληθεύουν.

\begin{lstlisting}
>>> e = 5 - x
>>> e.subs(x,1)
\end{lstlisting}
$$4$$
\begin{lstlisting}
>>> e.subs(x,3)
\end{lstlisting}
$$2$$
\begin{lstlisting}
>>> e.subs(x,4)
\end{lstlisting}
$$1$$
\begin{lstlisting}
>>> e.subs(x,5)
\end{lstlisting}
$$0$$
\begin{lstlisting}
>>> e.subs(x,6)
\end{lstlisting}
$$-1$$
\begin{lstlisting}
>>> e.subs(x,11)
\end{lstlisting}
$$-6$$

Οπότε ο αριθμός που την επαληθεύει είναι ο 1 και όλοι οι υπόλοιποι δεν την επαληθεύουν.

\begin{lstlisting}
>>> e = 2*x
>>> e.subs(x,1)
\end{lstlisting}
$$2$$
\begin{lstlisting}
>>> e.subs(x,3)
\end{lstlisting}
$$6$$
\begin{lstlisting}
>>> e.subs(x,4)
\end{lstlisting}
$$8$$
\begin{lstlisting}
>>> e.subs(x,5)
\end{lstlisting}
$$10$$
\begin{lstlisting}
>>> e.subs(x,6)
\end{lstlisting}
$$12$$
\begin{lstlisting}
>>> e.subs(x,11)
\end{lstlisting}
$$22$$

Οπότε ο αριθμός που την επαληθεύει είναι ο 4 και όλοι οι υπόλοιποι δεν την επαληθεύουν.

\begin{lstlisting}
>>> e = 6/x
>>> e.subs(x,1)
\end{lstlisting}
$$6$$
\begin{lstlisting}
>>> e.subs(x,3)
\end{lstlisting}
$$2$$
\begin{lstlisting}
>>> e.subs(x,4)
\end{lstlisting}
$$\frac{3}{2}$$
\begin{lstlisting}
>>> e.subs(x,5)
\end{lstlisting}
$$\frac{6}{5}$$
\begin{lstlisting}
>>> e.subs(x,6)
\end{lstlisting}
$$1$$
\begin{lstlisting}
>>> e.subs(x,11)
\end{lstlisting}
$$\frac{6}{11}$$

Οπότε ο αριθμός 3 επαληθεύει την εξίσωση και όλοι οι υπόλοιποι δεν την επαληθεύουν:


\begin{lstlisting}
>>> e = x/2
>>> e.subs(x,1)
\end{lstlisting}
$$6$$
\begin{lstlisting}
>>> e.subs(x,3)
\end{lstlisting}
$$2$$
\begin{lstlisting}
>>> e.subs(x,4)
\end{lstlisting}
$$\frac{3}{2}$$
\begin{lstlisting}
>>> e.subs(x,5)
\end{lstlisting}
$$\frac{6}{5}$$
\begin{lstlisting}
>>> e.subs(x,6)
\end{lstlisting}
$$1$$
\begin{lstlisting}
>>> e.subs(x,11)
\end{lstlisting}
$$\frac{6}{11}$$

\begin{lstlisting}
>>> e = x/2
>>> e.subs(x,1)
\end{lstlisting}
$$\frac{1}{2}$$

\begin{lstlisting}
>>> e.subs(x,3)
\end{lstlisting}
$$\frac{3}{2}$$

\begin{lstlisting}
>>> e.subs(x,4)
\end{lstlisting}
$$2$$

\begin{lstlisting}
>>> e.subs(x,5)
\end{lstlisting}
$$\frac{5}{2}$$

\begin{lstlisting}
>>> e.subs(x,6)
\end{lstlisting}
$$3$$
\begin{lstlisting}
>>> e.subs(x,11)
\end{lstlisting}
$$\frac{11}{2}$$

Ο αριθμός που επαληθεύει την εξίσωση είναι ο 6, οι υπόλοιποι αριθμοί δεν την επαληθεύουν.

\begin{lstlisting}
>>> e = x + 7
>>> e.subs(x,1)
\end{lstlisting}
$$8$$

\begin{lstlisting}
>>> e.subs(x,3)
\end{lstlisting}
$$10$$

\begin{lstlisting}
>>> e.subs(x,4)
\end{lstlisting}
$$11$$

\begin{lstlisting}
>>> e.subs(x,5)
\end{lstlisting}
$$12$$

\begin{lstlisting}
>>> e.subs(x,6)
\end{lstlisting}
$$13$$
\begin{lstlisting}
>>> e.subs(x,11)
\end{lstlisting}
$$18$$

Κανένας από αυτούς τους αριθμούς δεν επαληθεύει την εξίσωση, οπότε:

\begin{tabular}{|c|c|c|}
Εξίσωση            &Αριθμοί που την επαληθεύουν  &Αριθμοί που δεν την επαληθεύουν\\
$x - 4 = 1$        &            5                &          1, 3, 4, 6 και 11    \\
$5 - x = 4$        &            1                &          3, 4, 5, 6 και 11    \\
$2x = 8$           &            4                &          1, 3, 5, 6 και 11    \\
$\frac{6}{x} = 2$  &            3                &          1, 4, 5, 6 και 11    \\
$\frac{x}{2} = 3$  &            6                &          1, 3, 4, 5 και 11    \\
$x + 7 = 30$       &                             &          1, 3, 4, 5, 6 και 11 \\
\end{tabular}

Ένας καλύτερος τρόπος για να έχουμε το ίδιο αποτέλεσμα είναι να γραφτεί ένα πρόγραμμα που να υπολογίζει τα αποτελέσματα για όλους τους αριθμούς και να συγκρίνει το αποτέλεσμα με το αναμενόμενο. Η enumerate μετράει τη λίστα και δημιουγεί έναν μετρητή με όνομα i που μπορούμε να τον χρησιμοποιήσουμε για να μετρήσουμε τα αναμενόμενα αποτελέσματα:
\begin{lstlisting}
for e in exprs: 
    for (i,xi) in enumerate([1,3,4,5,6,11]):
        print(e,xi,e.subs(x,xi),res[i])
        print(e.subs(x,xi)==res[i])
\end{lstlisting}
\begin{exercise}
\sel{73}
Να λυθούν οι εξισώσεις:
$$x+5=12$$
$$y-2=3$$
$$10-z =1$$
$$7\cdot phi = 14$$
$$w:5 = 4$$
$$24:\psi = 6$$
\end{exercise}
Η βιβλιοθήκη sympy έχει συνάρτηση solve για να λύνει εξισώσεις όταν το δεξί μέρος της εξίσωσης είναι 0 οπότε οι εξισώσεις πρέπει να μετατραπούν με το χέρι σε:
$$x+5-12 = 0$$
$$y-2-3 = 0$$
$$10-z -1 = 0$$
$$7\cdot phi - 14 = 0$$
$$w:5 - 4 = 0$$
$$24:\psi - 6 = 0$$

\begin{lstlisting}
>>> from sympy import *
>>> x,y,z,f,w,psi = symbols('x y z f w psi')
>>> solve(x+5-12)
[7]
>>> solve(y-2-3)
[5]
>>> solve(10-z -1)
[9]
>>> solve(7* f - 14)
[2]
>>> solve(w/5 - 4)
[20]
>>> solve(24/psi - 6)
[4]
\end{lstlisting}
Η συνάρτηση solve επιστρέφει μια λίστα με τις τιμές που επαληθεύουν την εξίσωση. Επειδή υπάρχει μόνο μία τιμή που επαληθεύει την εξίσωση για αυτό το λόγο υπάρχει μόνο μία τιμή στην κάθε λίστα.

\begin{exercise}
\sel{63}
Μια δεξαμενή χωρητικότητας 6m$^3$ που έχει μήκος 1,5m και πλάτος 2m, έχει
ύψος (α) 1,5m ή (β) 3m ή (γ) 2m;
\end{exercise}

\begin{lstlisting}
>>> solve(2*1.5*x - 6)
[2.0]
\end{lstlisting}
\begin{exercise}
\sel[4]{74}Γράψε με απλούστερο τρόπο τις μαθηματικές εκφράσεις: 

(α) $x+x$,

(β) $\alpha+\alpha+\alpha$,

(γ) $3\cdot \alpha+52\cdot \alpha$, 

(δ) $2\cdot \beta+\beta+3\cdot \alpha+2\cdot \alpha$, 

(ε) $4\cdot x+8\cdot x–3\cdot x$, 

(στ) $7\cdot \omega+4\cdot \omega–10\cdot \omega$

\end{exercise}

\begin{lstlisting}
>>> x+x
\end{lstlisting}

$$2x$$

\begin{lstlisting}
>>> a = symbols('a')
>>> a+a+a
\end{lstlisting}

$$3a$$

\begin{lstlisting}
>>>  3*a  + 52 * a
\end{lstlisting}

$$55a$$

\begin{lstlisting}
>>> a,b = symbols('a b')
>>> 2*b+b+3*a+2*a 
\end{lstlisting}

$$5a+3b$$

\begin{lstlisting}
>>> 4*x+8*x–3*x 
\end{lstlisting}

$$9x$$

\begin{lstlisting}
>>> w = symbols('w')
>>> 7*w+4*w–10*w
\end{lstlisting}

$$w$$

\begin{exercise}
\sel[6]{74}
Στην εξίσωση 2 + α = x, το α και το x είναι φυσικοί αριθμοί. Ποια από τις τιμές
0, 3, 1 μπορεί να πάρει το x ;
\end{exercise}
Θα λύσουμε την $$2+a-x=0$$ για αυτές τις τιμές:
\begin{lstlisting}
>>> solve(2+a-0)
[-2]
>>> solve(2+a-3)
[1]
>>> solve(2+a-1)
[-1]
\end{lstlisting}
Από αυτές τις λύσεις συμπεραίνουμε ότι μόνο η $2+a-3$ μπορεί να ισχύει για φυσικό αριθμό και άρα μόνο την τιμή $3$ μπορεί να πάρει το $x$.
\begin{exercise}
\sel[7]{74}
Να εξετάσεις, αν ο αριθμός 12 είναι η λύση της εξίσωσης: x + 13 = 25
\end{exercise}
\begin{lstlisting}
>>> e = x + 13
>>> e.subs(e,x,12)
\end{lstlisting}
$$25$$
