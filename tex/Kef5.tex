\chapter{Εξισώσεις και προβλήματα}

Σε αυτό το κεφάλαιο θα χρησιμοποιήσουμε τη βιβλιοθήκη sympy.
Υπάρχει ένα περιβάλλον στο οποίο μπορούμε να πληκτρολογούμε εντολές της βιβλιοθήκης ώστε να βλέπουμε τα αποτελέσματα με φιλικό τρόπο στον φυλλομετρήτή μας, συνήθως Chrome, Firefox ή Microsoft Edge. Το περιβάλλον αυτό βρίσκεται στη διεύθυνση https://live.sympy.org/. Μπορούμε να κάνουμε τα ίδια παραδείγματα στον Mu Editor όπως έχουμε συνηθίσει χρησιμοποιώντας την εντολή:
\begin{lstlisting}
from sympy import *
\end{lstlisting}
όμως τα αποτελέσματα δεν θα εμφανίζονται με φιλικό τρόπο αλλά με τον συμβολισμό της Python.
\section{Η έννοια της εξίσωσης}
\begin{exercise}
Γράψε συντομότερα τις εκφράσεις:

(α) $x + x + x + x$, 

(β) $\alpha + \alpha + \alpha + \beta + \beta$, 

(γ) $3\cdot \alpha + 5 \cdot \alpha$, 

(δ) $18 \cdot x + 7 \cdot x + 4 \cdot x$, 

(ε) $15 \cdot \beta – 9 \cdot \beta$.
\end{exercise}

Επειδή τα σύμβολα είναι τα $x, a, b$ θα πρέπει να τα δηλώσουμε στο sympy. Αυτό γίνεται ως εξής:
\begin{lstlisting}
from sympy import *
x,a,b = symbols("x a b")
\end{lstlisting}

Στη συνέχεια όποτε αναφέρουμε τα $x, a, b$ η Python θα καταλαβαίνει ότι πρόκειται για σύμβολα και θα δρα ανάλογα.
Έτσι αν δώσουμε στην Python
\begin{lstlisting}
>>> x + x + x + x
\end{lstlisting}
Θα μας δώσει ως απάντηση
$$4x$$
στο live.sympy.org
και
\begin{lstlisting}
4*x
\end{lstlisting}
στην απλή Python ή στο Mu Editor.
Άρα το 
\begin{lstlisting}
>>> a + a + a + b + b
\end{lstlisting}
Θα μας δώσει σαν απάντηση:
$$3a+2b$$
και τα 
\begin{lstlisting}
>>> 3*a + 5*a 
>>> 18*x + 7*x + 4*x
>>> 15*b - 9b
\end{lstlisting}
$$8a$$
$$29x$$
$$6b$$
αντίστοιχα.
\begin{exercise}
Να αντικαταστήσεις το x, με τους αριθμούς 1, 3, 4, 5, 6 και 11, σε κάθε ισότητα της πρώτης στήλης, του παρακάτω πίνακα. Βρες ποιος από αυτούς την επαληθεύει και ποιος όχι.

\begin{tabular}{|c|c|c|}
Εξίσωση            &Αριθμοί που την επαληθεύουν     &Αριθμοί που δεν την επαληθεύουν\\
$x – 4 = 1$        &                             &                                \\
$5 – x = 4$        &                             &                                \\
$2x = 8$           &                             &                                \\
$\frac{6}{x} = 2$  &                             &                                \\
$\frac{x}{2} = 3$  &                             &                                \\
$x + 7 = 30$       &                             &                                \\
\end{tabular}
\end{exercise}

\begin{lstlisting}
>>> e = x - 4
>>> e.subs(x,1)
\end{lstlisting}
$$−3$$
\begin{lstlisting}
>>> e.subs(x,3)
\end{lstlisting}
$$−1$$
\begin{lstlisting}
>>> e.subs(x,4)
\end{lstlisting}
$$0$$
\begin{lstlisting}
>>> e.subs(x,5)
\end{lstlisting}
$$1$$
\begin{lstlisting}
>>> e.subs(x,6)
\end{lstlisting}
$$2$$
\begin{lstlisting}
>>> e.subs(x,11)
\end{lstlisting}
$$7$$

Οπότε ο αριθμός που την επαληθεύει είναι ο 5 και όλοι οι υπόλοιποι δεν την επαληθεύουν.

\begin{lstlisting}
>>> e = 5 - x
>>> e.subs(x,1)
\end{lstlisting}
$$4$$
\begin{lstlisting}
>>> e.subs(x,3)
\end{lstlisting}
$$2$$
\begin{lstlisting}
>>> e.subs(x,4)
\end{lstlisting}
$$1$$
\begin{lstlisting}
>>> e.subs(x,5)
\end{lstlisting}
$$0$$
\begin{lstlisting}
>>> e.subs(x,6)
\end{lstlisting}
$$-1$$
\begin{lstlisting}
>>> e.subs(x,11)
\end{lstlisting}
$$-6$$

Οπότε ο αριθμός που την επαληθεύει είναι ο 1 και όλοι οι υπόλοιποι δεν την επαληθεύουν.

\begin{lstlisting}
>>> e = 2*x
>>> e.subs(x,1)
\end{lstlisting}
$$2$$
\begin{lstlisting}
>>> e.subs(x,3)
\end{lstlisting}
$$6$$
\begin{lstlisting}
>>> e.subs(x,4)
\end{lstlisting}
$$8$$
\begin{lstlisting}
>>> e.subs(x,5)
\end{lstlisting}
$$10$$
\begin{lstlisting}
>>> e.subs(x,6)
\end{lstlisting}
$$12$$
\begin{lstlisting}
>>> e.subs(x,11)
\end{lstlisting}
$$22$$

Οπότε ο αριθμός που την επαληθεύει είναι ο 4 και όλοι οι υπόλοιποι δεν την επαληθεύουν.

\begin{lstlisting}
>>> e = 6/x
>>> e.subs(x,1)
\end{lstlisting}
$$6$$
\begin{lstlisting}
>>> e.subs(x,3)
\end{lstlisting}
$$2$$
\begin{lstlisting}
>>> e.subs(x,4)
\end{lstlisting}
$$\frac{3}{2}$$
\begin{lstlisting}
>>> e.subs(x,5)
\end{lstlisting}
$$\frac{6}{5}$$
\begin{lstlisting}
>>> e.subs(x,6)
\end{lstlisting}
$$1$$
\begin{lstlisting}
>>> e.subs(x,11)
\end{lstlisting}
$$\frac{6}{11}$$

Οπότε ο αριθμός 3 επαληθεύει την εξίσωση και όλοι οι υπόλοιποι δεν την επαληθεύουν:


\begin{lstlisting}
>>> e = x/2
>>> e.subs(x,1)
\end{lstlisting}
$$6$$
\begin{lstlisting}
>>> e.subs(x,3)
\end{lstlisting}
$$2$$
\begin{lstlisting}
>>> e.subs(x,4)
\end{lstlisting}
$$\frac{3}{2}$$
\begin{lstlisting}
>>> e.subs(x,5)
\end{lstlisting}
$$\frac{6}{5}$$
\begin{lstlisting}
>>> e.subs(x,6)
\end{lstlisting}
$$1$$
\begin{lstlisting}
>>> e.subs(x,11)
\end{lstlisting}
$$\frac{6}{11}$$

\begin{lstlisting}
>>> e = x/2
>>> e.subs(x,1)
\end{lstlisting}
$$\frac{1}{2}$$

\begin{lstlisting}
>>> e.subs(x,3)
\end{lstlisting}
$$\frac{3}{2}$$

\begin{lstlisting}
>>> e.subs(x,4)
\end{lstlisting}
$$2$$

\begin{lstlisting}
>>> e.subs(x,5)
\end{lstlisting}
$$\frac{5}{2}$$

\begin{lstlisting}
>>> e.subs(x,6)
\end{lstlisting}
$$3$$
\begin{lstlisting}
>>> e.subs(x,11)
\end{lstlisting}
$$\frac{11}{2}$$

Ο αριθμός που επαληθεύει την εξίσωση είναι ο 6, οι υπόλοιποι αριθμοί δεν την επαληθεύουν.

\begin{lstlisting}
>>> e = x + 7
>>> e.subs(x,1)
\end{lstlisting}
$$8$$

\begin{lstlisting}
>>> e.subs(x,3)
\end{lstlisting}
$$10$$

\begin{lstlisting}
>>> e.subs(x,4)
\end{lstlisting}
$$11$$

\begin{lstlisting}
>>> e.subs(x,5)
\end{lstlisting}
$$12$$

\begin{lstlisting}
>>> e.subs(x,6)
\end{lstlisting}
$$13$$
\begin{lstlisting}
>>> e.subs(x,11)
\end{lstlisting}
$$18$$

Κανένας από αυτούς τους αριθμούς δεν επαληθεύει την εξίσωση, οπότε:

\begin{tabular}{|c|c|c|}
Εξίσωση            &Αριθμοί που την επαληθεύουν  &Αριθμοί που δεν την επαληθεύουν\\
$x - 4 = 1$        &            5                &          1, 3, 4, 6 και 11    \\
$5 - x = 4$        &            1                &          3, 4, 5, 6 και 11    \\
$2x = 8$           &            4                &          1, 3, 5, 6 και 11    \\
$\frac{6}{x} = 2$  &            3                &          1, 4, 5, 6 και 11    \\
$\frac{x}{2} = 3$  &            6                &          1, 3, 4, 5 και 11    \\
$x + 7 = 30$       &                             &          1, 3, 4, 5, 6 και 11 \\
\end{tabular}

Ένας καλύτερος τρόπος για να έχουμε το ίδιο αποτέλεσμα είναι να γραφτεί ένα πρόγραμμα που να υπολογίζει τα αποτελέσματα για όλους τους αριθμούς και να συγκρίνει το αποτέλεσμα με το αναμενόμενο. Η enumerate μετράει τη λίστα και δημιουγεί έναν μετρητή με όνομα i που μπορούμε να τον χρησιμοποιήσουμε για να μετρήσουμε τα αναμενόμενα αποτελέσματα:
\begin{lstlisting}
for e in exprs: 
    for (i,xi) in enumerate([1,3,4,5,6,11]):
        print(e,xi,e.subs(x,xi),res[i])
        print(e.subs(x,xi)==res[i])
\end{lstlisting}
\begin{exercise}
\sel{73}
Να λυθούν οι εξισώσεις:
$$x+5=12$$
$$y-2=3$$
$$10-z =1$$
$$7\cdot phi = 14$$
$$w:5 = 4$$
$$24:\psi = 6$$
\end{exercise}
Η βιβλιοθήκη sympy έχει συνάρτηση solve για να λύνει εξισώσεις όταν το δεξί μέρος της εξίσωσης είναι 0 οπότε οι εξισώσεις πρέπει να μετατραπούν με το χέρι σε:
$$x+5-12 = 0$$
$$y-2-3 = 0$$
$$10-z -1 = 0$$
$$7\cdot phi - 14 = 0$$
$$w:5 - 4 = 0$$
$$24:\psi - 6 = 0$$

\begin{lstlisting}
>>> from sympy import *
>>> x,y,z,f,w,psi = symbols('x y z f w psi')
>>> solve(x+5-12)
[7]
>>> solve(y-2-3)
[5]
>>> solve(10-z -1)
[9]
>>> solve(7* f - 14)
[2]
>>> solve(w/5 - 4)
[20]
>>> solve(24/psi - 6)
[4]
\end{lstlisting}
Η συνάρτηση solve επιστρέφει μια λίστα με τις τιμές που επαληθεύουν την εξίσωση. Επειδή υπάρχει μόνο μία τιμή που επαληθεύει την εξίσωση για αυτό το λόγο υπάρχει μόνο μία τιμή στην κάθε λίστα.

\begin{exercise}
\sel{63}
Μια δεξαμενή χωρητικότητας 6m$^3$ που έχει μήκος 1,5m και πλάτος 2m, έχει
ύψος (α) 1,5m ή (β) 3m ή (γ) 2m;
\end{exercise}

\begin{lstlisting}
>>> solve(2*1.5*x - 6)
[2.0]
\end{lstlisting}
\begin{exercise}
\sel[4]{74}Γράψε με απλούστερο τρόπο τις μαθηματικές εκφράσεις: 

(α) $x+x$,

(β) $\alpha+\alpha+\alpha$,

(γ) $3\cdot \alpha+52\cdot \alpha$, 

(δ) $2\cdot \beta+\beta+3\cdot \alpha+2\cdot \alpha$, 

(ε) $4\cdot x+8\cdot x–3\cdot x$, 

(στ) $7\cdot \omega+4\cdot \omega–10\cdot \omega$

\end{exercise}

\begin{lstlisting}
>>> x+x
\end{lstlisting}

$$2x$$

\begin{lstlisting}
>>> a = symbols('a')
>>> a+a+a
\end{lstlisting}

$$3a$$

\begin{lstlisting}
>>>  3*a  + 52 * a
\end{lstlisting}

$$55a$$

\begin{lstlisting}
>>> a,b = symbols('a b')
>>> 2*b+b+3*a+2*a 
\end{lstlisting}

$$5a+3b$$

\begin{lstlisting}
>>> 4*x+8*x–3*x 
\end{lstlisting}

$$9x$$

\begin{lstlisting}
>>> w = symbols('w')
>>> 7*w+4*w–10*w
\end{lstlisting}

$$w$$

\begin{exercise}
\sel[6]{74}
Στην εξίσωση 2 + α = x, το α και το x είναι φυσικοί αριθμοί. Ποια από τις τιμές
0, 3, 1 μπορεί να πάρει το x ;
\end{exercise}
Θα λύσουμε την $$2+a-x=0$$ για αυτές τις τιμές:
\begin{lstlisting}
>>> solve(2+a-0)
[-2]
>>> solve(2+a-3)
[1]
>>> solve(2+a-1)
[-1]
\end{lstlisting}
Από αυτές τις λύσεις συμπεραίνουμε ότι μόνο η $2+a-3$ μπορεί να ισχύει για φυσικό αριθμό και άρα μόνο την τιμή $3$ μπορεί να πάρει το $x$.
\begin{exercise}
\sel[7]{74}
Να εξετάσεις, αν ο αριθμός 12 είναι η λύση της εξίσωσης: x + 13 = 25
\end{exercise}
\begin{lstlisting}
>>> e = x + 13
>>> e.subs(e,x,12)
\end{lstlisting}
$$25$$
\begin{exercise}
\sel[8]{74}
Τοποθέτησε ένα “Χ” στη θέση εκείνη που ο αριθμός επαληθεύει την αντίστοιχη εξίσωση:

\begin{tabular}{|c|c|c|c|c|c|c|c|c|}
&1&2&3&4&5&6&7&8\\
$x – 2 = 4$&&&&&&&&\\
$1 + y = 4$&&&&&&&&\\
$18 – \omega = 10$&&&&&&&&\\
$2 – \alpha = 1$&&&&&&&&\\
$93 – \beta = 86$&&&&&&&&\\
\end{tabular}
\end{exercise}
Χρησιμοποιώντας έναν πίνακα για τα αποτελέσματα res γράφουμε:
\begin{lstlisting}
from sympy import *
x,y,w,a,b  = symbols('x y w a b')
s = [x,y,w,a,b]
e = [x-2,1+y,18-w,2-a,93-b]
res = [4,4,10,1,86]
for (en,expr) in enumerate(e):
    for i in range(9):
        if expr.subs(s[en],i) == res[en]:
            print('X',end='')
        else:
            print('O',end='')
    print()
\end{lstlisting}
που δίνει το αποτέλεσμα
\begin{lstlisting}
OOOOOOXOO
OOOXOOOOO
OOOOOOOOX
OXOOOOOOO
OOOOOOOXO
\end{lstlisting}
οπότε ο πίνακας διαμορφώνεται ως εξής:
\begin{tabular}{|c|c|c|c|c|c|c|c|c|}
&1&2&3&4&5&6&7&8\\
$x – 2 = 4$&&&&&&Χ&&\\
$1 + y = 4$&&&&Χ&&&&\\
$18 – \omega = 10$&&&&&&&&Χ\\
$2 – \alpha = 1$&&Χ&&&&&&\\
$93 – \beta = 86$&&&&&&&Χ&\\
\end{tabular}
\begin{exercise}
\sel[9]{74}
Ποιος αριθμός επαληθεύει κάθε μία από τις παρακάτω εξισώσεις;

(α) $x + 4,9 = 15,83$ 

(β) $40,4 + x = 93,19$ 

(γ) $53,404 – x = 4,19$ 

(δ) $38 – x = 7,1$.
\end{exercise}
Όπως και προηγουμένως η sympy μπορεί να λύσει την εξίσωση αρκεί το αριστερό μέλος να είναι 0. Οπότε:
\begin{lstlisting}
>>> solve(x+4.9-15.83)
[10.9300000000000]
>>> solve(40.4+x-93.19)
[52.7900000000000]
>>> solve(53.404 - x - 4.19)
[49.2140000000000]
>>> solve(38-x-7.1)
[30.9000000000000]
\end{lstlisting}
Άρα οι απαντήσεις είναι 10,93, 52,79, 49,214, 30,9.

\begin{exercise}
\sel[11]{74}
Ποια είναι η τιμή του x για να ισχύει ; 

(α) $3x = \frac{12}{20}$,

(β) $\frac{5}{7}=\frac{15}{x}$,

(γ) $\frac{35}{40} = \frac{x}{8}$,

(δ) $\frac{49}{5} = x + \frac{4}{5}$.
\end{exercise}

Με την ίδια λογική:

\begin{lstlisting}
>>> solve(3/x-12/20)
[9.00000000000000]
>>> solve(5/7-15/x)
[21.0000000000000]
>>> solve(35/40-x/8)
[7.00000000000000]
>>> solve(49/5-x-4/5)
[9.00000000000000]
\end{lstlisting}
Άρα οι απαντήσεις είναι 9, 21, 7 και 9.

\begin{exercise}
\sel[12]{74}
Λύσε τις εξισώσεις: 

(α) $\ni+3=4$, 

(β) $x – 2=8$, 

(γ) $t+4+1=3+19$,

(δ) $6 – x=5$.
\end{exercise}

\begin{lstlisting}
from sympy import *
x,n,t = symbols('x n t')
print(solve(n+3-4))
print(solve(x-2-8))
print(solve(t+4+1-3-19))
print(solve(6-x-5))
\end{lstlisting}
και το αποτέλεσμα είναι:
\begin{lstlisting}
[1]
[10]
[17]
[1]
\end{lstlisting}

\begin{exercise}
\sel[13]{74}
Ποιον αριθμό πρέπει να προσθέσεις στον $4$, για να προκύψει ο αντίστροφός του 
$\frac{5}{21}$;
\end{exercise}
\begin{lstlisting}
>>> solve(x+4-21/5)
[0.200000000000000]
\end{lstlisting}
\begin{exercise}
\sel[14]{74}
Σε έναν αριθμό προσθέτουμε 5 και παίρνουμε άθροισμα 313. Ποιος είναι ο αριθμός;
\end{exercise}
\begin{lstlisting}
>>> solve(x+5-313)
[308]
\end{lstlisting}
\begin{exercise}
\sel[15]{74}
Τα τετράγωνα που αποτελούν τους “δομικούς λίθους” με τους οποίους κατασκευά-
ζουμε τα παρακάτω σχήματα, έχουν πλευρά ίση με 1 cm.
(α) Bρες την περίμετρο του πέμπτου σχήματος και εξήγησε πώς έφτασες στην απάντησή σου.
(β) Γράψε ένα τύπο με τη βοήθεια του οποίου θα μπορείς να υπολογίσεις την περίμετρο κάθε σχήματος.
(γ) Ποια είναι η σειρά του σχήματος του οποίου η περίμετρος είναι 128 cm;
\end{exercise}

α) Το πέμπτο σχήμα θα έχει περίμετρο $20$cm.

β) $$4x$$

(γ) 
\begin{lstlisting}
>>> solve(4x-128)
[32]
\end{lstlisting}

\begin{exercise}
\sel{75}
Ένα κατάστημα για να προσελκύσει πελατεία ανακοινώνει ότι ο πελάτης που θα αγοράσει τρία
ίδια πακέτα προσφοράς ενός συγκεκριμένου προϊόντος θα έχει έκπτωση 5D. Aν και τα τρία
πακέτα κοστίζουν, με την έκπτωση, συνολικά 85D, ποιά είναι η αρχική αξία του κάθε πακέτου;
\end{exercise}
\begin{lstlisting}
>>> solve(3x-5-85)
[30]
\end{lstlisting}
\begin{exercise}
\sel{75}
Να περιγράψεις κάποιο πρόβλημα, που να λύνεται με τη βοήθεια της εξίσωσης: 2x+800=1000.
\end{exercise}
Είναι δύσκολο να βρούμε με την Python ένα τέτοιο πρόβλημα όμως η λύση του μπορεί να βρεθεί:

\begin{lstlisting}
>>> solve(2x+800-1000)
[100]
\end{lstlisting}
\begin{exercise}
\sel{76}
Η Χριστίνα ξόδεψε τα μισά της χρήματα για να αγοράσει 2 τετράδια και μαρκαδόρους.
Αν είναι γνωστό, ότι κάθε τετράδιο στοιχίζει 1 Q και όλοι οι μαρκαδόροι 3 Q, ποιο είναι
το ποσό των χρημάτων που είχε η Χριστίνα πριν από τις αγορές αυτές;
\end{exercise}
\begin{lstlisting}
>>> solve(x/2-2-3)
[10]
\end{lstlisting}
\begin{exercise}
Η δεξαμενή της κοινότητας χωράει 3.000 m$^3$ νερό. Κάθε μέρα ξοδεύονται 300 m$^3$ από
τα νοικοκυριά και άλλα 200 m$^3$ από τις βιοτεχνίες. Για τη συντήρηση του δικτύου,
σταμάτησε η παροχή νερού προς τη δεξαμενή. Τέσσερις ημέρες μετά την έναρξη των
εργασιών αποφασίζεται να ξοδεύονται μόνο 400 m$^3$ συνολικά κάθε ημέρα. Πόσες
ημέρες ακόμη πρέπει να κρατήσουν τα έργα συντήρησης, ώστε να μη μείνουν χωρίς
νερό οι κάτοικοι της κοινότητας;
\end{exercise}
\begin{lstlisting}
>>> solve((300+200)*4+400*x-3000)
[5/2]
\end{lstlisting}
Δηλαδή 2,5 ημέρες.
\begin{exercise}
Ένας εργάτης για μια εργασία πέντε ημερών συμφώνησε να πάρει προκαταβολή το
μισό της αμοιβής του και το υπόλοιπο αυτής να το πληρωθεί όταν τελειώσει η εργασία.
Αν η προκαταβολή ήταν 180€, ποιό ήταν το μεροκάματό του;
\end{exercise}
\begin{lstlisting}
>>> solve(5/2*x - 180)
[72]
\end{lstlisting}

\begin{exercise}
\sel{76}
Mετά τη συνεδρίαση και τα 10 μέλη του διοικητικού
συμβουλίου μιας εταιρείας ανταλλάσσουν μεταξύ
τους χειραψίες. Πόσες χειραψίες γίνονται συνολικά;
\end{exercise}
\begin{lstlisting}
>>> sum(range(10))
45
\end{lstlisting}
H εντολή sum υπολογίζει το άθροισμα μιας λίστας. Στη συγκεκριμένη περίπτωση η λίστα είναι η range(10) που είναι οι αριθμοί από το 0 μέχρι το 9.

\begin{exercise}
\sel[1]{78}
Η διαφορά της ηλικίας της κόρης από τη μητέρα της είναι 25 χρόνια.
Αν η κόρη είναι 18 ετών, πόσων ετών είναι η μητέρα;
\end{exercise}
$$x-25 = 18$$
\begin{lstlisting}
>>> solve(x-25-18)
[43]
\end{lstlisting}
\begin{exercise}
\sel[2]{78}
Πόσοι μαθητές είναι τα $\frac{7}{10}$ των μαθητών ενός σχολείου, αν τα $\frac{2}{8}$ των
μαθητών, αυτού του σχολείου, είναι $60$ μαθητές.
\end{exercise}
$$\frac{2}{8}\cdot x  =60$$
\begin{lstlisting}
>>> solve(2/8*x-60)
[240]
\end{lstlisting}
Για τα $\frac{7}{10}$ των μαθητών έχουμε
\begin{lstlisting}
>>> 7/10*240
168.0
\end{lstlisting}
\begin{exercise}
\sel[3]{78}
Να βρεις τρεις διαδοχικούς φυσικούς αριθμούς που έχουν άθροισμα 1533.
\end{exercise}
Έστω ότι ο πρώτος αριθμός από αυτούς είναι ο $x$. Τότε:
$$x+(x+1)+x+2 = 1533$$
$$3x+3 = 1533$$
\begin{lstlisting}
>>> solve(3*x+3-1533)
[510]
\end{lstlisting}
Οι αριθμοί είναι 510, 511, 512.
\begin{exercise}
\sel[4]{78}
Βρες το ψηφίο που λείπει από τον αριθμό 75\_3, ώστε αυτός να διαιρείται με το 9.
\end{exercise}
Λύση 1η
Από τα κριτήρια διαιρετότητας ξέρουμε ότι θα πρέπει το άθροισμα των ψηφίων να διαιρείται με το 9 οπότε:

$7+5+x+3$ να είναι πολλαπλάσιο του 9

$x+15$ πολλαπλάσιο του 9 

Τα πολλαπλάσια του 9 είναι 9, 18, 27,...

Ας δούμε τις πιθανότητες:

\begin{lstlisting}
>>> solve(x+15-9)
[-6]
>>> solve(x+15-18)
[3]
>>> solve(x+15-27)
[12]
\end{lstlisting}
Από αυτές τις λύσεις μόνο η 3 είναι αποδεκτή (ένα ψηφίο). Οπότε ο αριθμός είναι 7533 για τον οποίο ισχύει ότι το υπόλοιπο της διαίρεσής του με το 9 είναι 0.
\begin{lstlisting}
>>> 7533%9
0
\end{lstlisting}
Λύση 2η
\begin{lstlisting}
for d in range(10):
    number = 7503 + 10*d
    if number%9 == 0:
        print(d)
\end{lstlisting}
Το αποτέλεσμα της εκτέλεσης είναι
\begin{lstlisting}
3
\end{lstlisting}

\begin{exercise}
\sel[5]{78}
Σε ένα διαγώνισμα, κάθε μαθητής πρέπει να απαντήσει σε 100 ερωτήσεις. Θα πάρει
3 μονάδες, για κάθε σωστή απάντηση και μόνο 1 μονάδα, για κάθε λανθασμένη. Ένας
μαθητής πήρε συνολικά 220 μονάδες. Σε πόσες ερωτήσεις απάντησε σωστά
\end{exercise}
$$3x+(100-x)=220$$
\begin{lstlisting}
>>> solve(3*x+(100-x)-220)
[60]
\end{lstlisting}
Όντως αν απάντησε σε 60 ερωτήσεις σωστά τότε θα έχει απαντήσει 40 λάθος και θα πάρει $60*3+40*1=220$.

\begin{exercise}
\sel[6]{78}
Η ηλικία ενός πατέρα είναι τετραπλάσια από την ηλικία του γιου του. Οι δύο ηλικίες
μαζί συμπληρώνουν μισό αιώνα. Πόσο χρονών είναι ο καθένας;
\end{exercise}
$$4x+x=50$$
\begin{lstlisting}
>>> solve(4*x+x-50)
[10]
\end{lstlisting}
\begin{exercise}
\sel[7]{78}
Τρία αδέλφια μοιράζονται, εξίσου, μια κληρονομιά, που είναι ένα χωράφι και ένα
διαμέρισμα. Ο πρώτος παίρνει το χωράφι. Ο δεύτερος παίρνει το διαμέρισμα, αλλά
δίνει στον πρώτο 600€ και στον τρίτο 15.000€. Ποια ήταν η αξία του χωραφιού και
ποια του διαμερίσματος;
\end{exercise}
Αν $x$ η τιμή του χωραφιού και $y$ η τιμή του διαμερίσματος τότε:
$$x+600=(y-600-15000)=15000$$
Οπότε
\begin{lstlisting}
>>> solve(x+600-15000)
[14400]
>>> solve(y-600-15000-15000)
[30600]
\end{lstlisting}
Οπότε το χωράφι κοστίζει 14.400€ και το διαμέρισμα 30.600€.
