\chapter{Εξισώσεις και προβλήματα}

Σε αυτό το κεφάλαιο θα χρησιμοποιήσουμε τη βιβλιοθήκη sympy.
Υπάρχει ένα περιβάλλον στο οποίο μπορούμε να πληκτρολογούμε εντολές της βιβλιοθήκης ώστε να βλέπουμε τα αποτελέσματα με φιλικό τρόπο στον φυλλομετρήτή μας, συνήθως Chrome, Firefox ή Microsoft Edge. Το περιβάλλον αυτό βρίσκεται στη διεύθυνση https://live.sympy.org/. Μπορούμε να κάνουμε τα ίδια παραδείγματα στον Mu Editor όπως έχουμε συνηθίσει χρησιμοποιώντας την εντολή:
\begin{lstlisting}
from sympy import *
\end{lstlisting}
όμως τα αποτελέσματα δεν θα εμφανίζονται με φιλικό τρόπο αλλά με τον συμβολισμό της Python.
\section{Η έννοια της εξίσωσης}
\begin{exercise}
Γράψε συντομότερα τις εκφράσεις:

(α) $x + x + x + x$, 

(β) $\alpha + \alpha + \alpha + \beta + \beta$, 

(γ) $3\cdot \alpha + 5 \cdot \alpha$, 

(δ) $18 \cdot x + 7 \cdot x + 4 \cdot x$, 

(ε) $15 \cdot \beta – 9 \cdot \beta$.
\end{exercise}

Επειδή τα σύμβολα είναι τα $x, a, b$ θα πρέπει να τα δηλώσουμε στο sympy. Αυτό γίνεται ως εξής:
\begin{lstlisting}
from sympy import *
x,a,b = symbols("x a b")
\end{lstlisting}

Στη συνέχεια όποτε αναφέρουμε τα $x, a, b$ η Python θα καταλαβαίνει ότι πρόκειται για σύμβολα και θα δρα ανάλογα.
Έτσι αν δώσουμε στην Python
\begin{lstlisting}
>>> x + x + x + x
\end{lstlisting}
Θα μας δώσει ως απάντηση
$$4x$$
στο live.sympy.org
και
\begin{lstlisting}
4*x
\end{lstlisting}
στην απλή Python ή στο Mu Editor.
Άρα το 
\begin{lstlisting}
>>> a + a + a + b + b
\end{lstlisting}
Θα μας δώσει σαν απάντηση:
$$3a+2b$$
και τα 
\begin{lstlisting}
>>> 3*a + 5*a 
>>> 18*x + 7*x + 4*x
>>> 15*b - 9b
\end{lstlistling}
$$8a$$
$$29x$$
$$6b$$
αντίστοιχα.
