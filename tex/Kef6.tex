\chapter{Ποσοστά}
Στον διπλανό πίνακα φαίνεται το σύνολο των
πολιτών που ψήφισαν στα χωριά
Α, Β, Γ και Δ και οι ψήφοι που πήραν οι
αντίστοιχοι πρόεδροι που εκλέχτηκαν.
Βρες, ποιος από τους προέδρους που
εκλέχτηκαν, είναι ο πιο δημοφιλής.

\begin{tabular}{ccc}
Κοινότητα & Ψηφίσαντες & Ο πρόεδρος ψηφίστηκε από\\
A& 585& 354\\
B& 3.460& 1.802\\
Γ& 456&312\\
Δ&1.295&823\\
\end{tabular}   

\begin{lstlisting}
>>> 354/585
0.6051282051282051
\end{lstlisting}
Όμως για να το κάνουμε σαν ποσοστό \% τότε θα πρέπει να το πολλαπλασιάσουμε με το 100 οπότε:
\begin{lstlisting}
>>> 354/585*100
60.51282051282051
\end{lstlisting}
Επίσης καλό είναι η στρογγυλοποίηση να γίνει στο δεύτερο δεκαδικό ψηφίο. Οπότε 
\begin{lstlisting}
>>> round(354/585*100,2)
60.51
\end{lstlisting}
Μπορούμε να φτιάξουμε μια μικρή συνάρτηση που να τυπώνει σε ποσοστό έναν αριθμό ως εξής:
\begin{lstlisting}
def pososto(x):
    print(str(round(x*100,2))+'%')

pososto(354/585)
\end{lstlisting}

Που  δίνει το αποτέλεσμα 60.51\%
Οπότε έχουμε
\begin{lstlisting}
>>> pososto(354/585)
60.51%
>>> pososto(1802/3460)
52.08%
>>> pososto(312/456)
68.42%
>>> pososto(823/1295)
63.55%
\end{lstlisting}

Ο πιο δημοφιλής είναι ο Γ που έχει 68.42\%.
Αν θέλουμε όμως η Python να λύσει το πρόβλημα τότε μπορούμε να φτιάξουμε το εξής:
\begin{lstlisting}
class proedros():
    def __init__(self,onoma,psifoi,katoikoi):
        self.onoma = onoma
        self.psifoi = psifoi
        self.katoikoi = katoikoi
    def pososto(self):
        return(round(self.psifoi/self.katoikoi*100,2))

A = proedros('A',354,585)
B = proedros('B',1802,3460)
C = proedros('C',312,456)
D = proedros('D',823,1295)

M = max([A,B,C,D],key=lambda x:x.pososto());
print(M.onoma)
\end{lstlisting}

Που δίνει το αποτέλεσμα C δηλαδή Γ.

\begin{exercise}
Να γραφούν, ως ποσοστά επί τοις εκατό, τα παρακάτω κλάσματα:
(α) $\frac{4}{5}$
(β) $\frac{3}{8}$
(γ) $\frac{84}{91}$

με στρογγυλοποίηση στο εκατοστό.
\end{exercise}

Επειδή η συνάρτηση που έχουμε φτιάξει δεν προσαρμόζει την στρογγυλοποίηση μπορείς να την αλλάξεις ώστε να έχει έξτρα αυτό το δεδομένο. Μάλιστα μπορείς να δηλώσεις στην Python ότι αν δεν γράψεις αυτό το στοιχέιο θα είναι 0.
\begin{lstlisting}
def pososto(x,strog = 2):
    print(str(round(x*100,strog))+'%')

pososto(4/5,strog=0)
pososto(3/8,strog=0)
pososto(84/91,strog=0)
\end{lstlisting}

Έχουμε το αποτέλεσμα:
\begin{lstlisting}
80.0%
38.0%
92.0%
\end{lstlisting}
Μπορείς να αλλάξεις τη συνάρτηση ώστε να κάνει το αποτέλεσμα ακέραιο ειδικά αν το strog είναι 0.
\begin{lstlisting}
def pososto(x,strog = 2):
    if strog == 0:
        print(str(int(round(x*100),0))+'%')
    else:
        print(round(int(x*100),strog)+'%')

pososto(4/5,strog=0)
pososto(3/8,strog=0)
pososto(84/91,strog=0)
\end{lstlisting}

Τότε το αποτέλεσμα θα είναι:
\begin{lstlisting}
80%
37%<--!!!!!!!!!!!!!
92%
\end{lstlisting}
\begin{exercise}
\sel{81}
Να γραφούν, ως κλάσματα, τα ακόλουθα ποσοστά: (α) 12\%, (β) 73\%, (γ) 32,5\%.
\end{exercise}

\begin{lstlisting}
from fractions import Fraction
strx = input('Ποσοστό:')
if strx[-1] == '%':
    strx=strx[:-1]

fx = float(strx)
denom = 100
while int(fx) != fx:
    fx *= 10
    denom *= 10
fx = int(fx)
print(Fraction(fx,denom))
\end{lstlisting}
\begin{lstlisting}
Ποσοστό:12
3/25
Ποσοστό:73
73/100
Ποσοστό:32.5
13/40
\end{lstlisting}
\begin{exercise}
\sel{81}
Ποια θα είναι η τιμή πώλησης ενός πουλόβερ, αξίας 150€, με επιβάρυνση Φ.Π.Α. 19\%;
\end{exercise}
\begin{lstlisting}
>>> 150 + 150*19/100
178.5
\end{lstlisting}
\begin{exercise}
\sel[1]{81}
Γράψε   ως  ποσοστά επί τοις    εκατό,  τα  κλάσματα:   

(α)  $\frac{1}{5}$ , (β)     $\frac{3}{2}$ , (γ)    $\frac{1}{4}$ , (δ) $\frac{3}{4}$,  (ε) $\frac{3}{5}$
\end{exercise}

\begin{lstlisting}
>>> pososto(1/5)
20.0%
>>> pososto(3/2)
150.0%
>>> pososto(1/4)
25.0%
>>> pososto(3/4)
75.0%
>>> pososto(3/5)
60.0%
\end{lstlisting}
\begin{exercise}
\sel[2]{81}
Να  μετατρέψεις σε  ποσοστά επί τοις    εκατό,  τους    δεκαδικούς  αριθμούς:
(α) 0,52    ,           (β) 3,41    ,           (γ) 0,19    ,           (δ) 0,03    ,           (ε) 0,07.
\end{exercise}
\begin{lstlisting}
>>> 0.52*100
52
>>> 3.41*100
341
>>> 0.19*100
19
>>> 0.03*100
3
>>> 0.07*100
7
\end{lstlisting}
Άρα 52\%, 341\%, 19\%, 3\%, 7\%.
\begin{exercise}
\sel[3]{81}
Να  μετατρέψεις σε  δεκαδικά    κλάσματα    τα  ποσοστά:    (α) 15\%,    (β)7\%,  (γ)48\%, (δ) 50\%.    Στη 
συνέχεια,   απλοποίησε  τα  δεκαδικά    κλάσματα,   έως ότου    φτάσεις σε  ανάγωγο κλάσμα.
\end{exercise}
Θα μετατρέψουμε τον προηγούμενο κώδικα σε συνάρτηση:
\begin{lstlisting}
def posostoseklasma(fx):
    fx = float(fx)
    denom = 100
    while int(fx) != fx:
         fx *= 10
         denom *= 10
    fx = int(fx)
    return(Fraction(fx,denom))

print(posostoseklasma(15))
print(posostoseklasma(7))
print(posostoseklasma(48))
print(posostoseklasma(50))
\end{lstlisting}
Και το αποτέλεσμα είναι:
\begin{lstlisting}
3/20
7/100
12/25
1/2
\end{lstlisting}

\begin{exercise}
\sel[4]{81}
Υπολόγισε:  (α) το  10\% των 3000€,  (β) το  45\% της 1   ώρας,   (γ) το  20\% του λίτρου,
(δ) το  50\% των 500 γραμμαρίων, (ε) το  25\% του 1   κιλού.
\end{exercise}
\begin{lstlisting}
>>> 10/100*3000
300.0
\end{lstlisting}
Άρα 300€.

\begin{lstlisting}
>>> 45/100*60
27.0
\end{lstlisting}
Άρα 27 λεπτά.

\begin{lstlisting}
>>> 20/100*1000
200.0
\end{lstlisting}
Άρα 200ml

\begin{lstlisting}
>>> 50/100*500
250.0
\end{lstlisting}
Άρα 250g.

\begin{lstlisting}
>>> 25/100*1000
250.0
\end{lstlisting}
Άρα 250 γραμμάρια.

\begin{exercise}
\sel[5]{81}
Βρες     τι  ποσοστό     είναι:  (α)     τα  50€    για     τα  1.000€,  (β)     οι  30  ημέρες  για  το   1   έτος,
(γ) τα  50  στρέμματα   για τα  2.500   στρέμματα,  (δ) οι  3   παλάμες για τα  10  μέτρα.
\end{exercise}
\begin{lstlisting}
>>> 50/1000 *  100
5
\end{lstlisting}
Άρα 5\%.

\begin{lstlisting}
>>> 30/360 * 100
8.333333333333332
\end{lstlisting}
Άρα 8.33\%.

\begin{lstlisting}
>>> 50/2500 * 100
2.0
\end{lstlisting}
Άρα 2\%.

Παλάμη λέμε το δεκατόμετρο dm οπότε οι 3 παλάμες είναι 3 dm δηλαδή 30cm.
Οπότε:
\begin{lstlisting}
>>> 30 / (10*100) * 100
3
\end{lstlisting}
Άρα 3\%.

\begin{exercise}
\sel[6]{81}
Ένα  μπουκάλι    με  οινόπνευμα  παρέμεινε   ανοικτό και εξατμίστηκε το  22\% του όγκου   
του.    Το  μπουκάλι    περιείχε    αρχικά  0,610   lt. Πόσα    lt  οινοπνεύματος   εξατμίστηκαν;
\end{exercise}

\begin{lstlisting}
>>> 22*0.610 / 100
0.13419999999999999
\end{lstlisting}
Οπότε η Python δίνει μια προσέγγιση της σωστής απάντησης που είναι:
0.1342.

\begin{exercise}
Σε  ένα σημείο  της γήινης  σφαίρας,    ο   φλοιός  έχει    πάχος   50  Km, ο   
μανδύας 2.900   Km  και ο   πυρήνας 3.450   Km. (α) Να  βρεις   το  μήκος   
της ακτίνας της Γης σε  Km. (β) Να  βρεις   ποιο    ποσοστό της ακτίνας 
της Γης κατέχει ο   φλοιός, ο   μανδύας και ο   πυρήνας αντίστοιχα.
\end{exercise}
\begin{lstlisting}
x = [50,2900,3450]
print(sum(x))
for i in x:
    print(100*i/sum(x))
\end{lstlisting}
Το αποτέλεσμα του προγράμματος είναι:
\begin{lstlisting}
6400
0.78125
45.3125
53.90625
\end{lstlisting}
Οπότε το μήκος της ακτίνας της γης είναι 6400Km.
Ο φλοιός είναι το 0,78125\%, o μανδύας το 45,3125\% και ο πυρήνας το 53,90625\%.


\begin{exercise}
\sel{82}
Ένας ηλεκτρολόγος είχε έσοδα 2.856€ το δεύτερο τρίμηνο του έτους. Πόσα χρήματα
πρέπει να αποδώσει στο κράτος, αν ο Φ.Π.Α. που παρακρατά από τους πελάτες του
είναι 19\%.
\end{exercise}
Η σωστή απάντηση είναι:
\begin{lstlisting}
>>> 2856*19/119
456.0
\end{lstlisting}

\begin{exercise}
Στην περίοδο των εκπτώσεων, ένα κατάστημα έκανε έκπτωση 35\% στα είδη ρουχισμού
και 15\% στα παπούτσια. Πόσο θα πληρώσουμε για ένα πουκάμισο και ένα ζευγάρι
παπούτσια που κόστιζαν 58€ και 170€, αντίστοιχα, πριν τις εκπτώσεις.
\end{exercise}

\begin{lstlisting}
>>> 170 * 15/100
25.5
>>> 170 - 25.5
144.5
>>> 58*35/100
20.3
>>> 58-20.3
37.7
>>> 37.7 + 144.5
182.2
\end{lstlisting}
Μπορείς να κάνεις τις πράξεις αυτές σε μία συνάρτηση neatimi:
\begin{lstlisting}
def neatimi(timi,ekpt):
    return(timi-timi*ekpt/100)

neatimi(170,15)
neatimi(58,35)
\end{lstlisting}
Που δίνει σαν αποτέλεσμα:
\begin{lstlisting}
144.5
37.7
\end{lstlisting}
\begin{exercise}
\sel{82}
Ποσό 1.000€ κατατέθηκε σε λογαριασμό ταμιευτηρίου, με επιτόκιο 5\%. Πόσος είναι
ο τόκος που θα αποδώσει το κεφάλαιο αυτό, μετά από 18 μήνες, αν οι τόκοι
προστίθενται στο κεφάλαιο κάθε χρόνο;
\end{exercise}
Στον ένα χρόνο:
\begin{lstlisting}
>>> 1000*5/100
50.0
\end{lstlisting}
Για τους υπόλοιπους έξι μήνες θα είναι τα μισά οπότε:
\begin{lstlisting}
>>> 50.0/2
25.0
\end{lstlisting}
Συνολικά είναι:
\begin{lstlisting}
>>> 50.0 + 25.0
75.0
\end{lstlisting}
Σαν συνάρτηση γίνεται:
\begin{lstlisting}
def tokos(kefalaio,epitokio,mines):
    return(kefalaio*epitokio/ 100*mines/12)

print(tokos(1000,5,18))
\end{lstlisting}
Που δίνει το ίδιο αποτέλεσμα:
\begin{lstlisting}
75
\end{lstlisting}
\begin{exercise}
\sel[1]{82}
Επιχειρηματίας αγόρασε μετοχές μιας εταιρείας, προς 50€ την κάθε μετοχή. Σε ένα
μήνα η μετοχή έπεσε κατά 8\% και το επόμενο δίμηνο ανέβηκε κατά 5\% το μήνα.
(α) Ποια ήταν η τιμή της μετοχής στο τέλος του τρίτου μήνα; (β) Η επένδυση του
επιχειρηματία ήταν κερδοφόρα ή όχι; (γ) Ποιο είναι το ποσοστό κέρδους ή ζημίας του,
επί του αρχικού κεφαλαίου;
\end{exercise}
\begin{lstlisting}
>>> 50 - 8/100*50
46.0
>>> 46+5/100*46
48.3
>>> 48.3+5/100*48.3
50.714999999999996
\end{lstlisting}

Η τιμή της μετοχής είναι 50,715.

Η επένδυση ήταν κερδοφόρα.

Το ποσοστό κέρδους είναι:
\begin{lstlisting}
>>> (50.715 - 50)/50*100
1.4300000000000068
\end{lstlisting}

Άρα το αποτέλεσμα είναι 1,43\%.

\begin{exercise}
\sel[2]{82}
Κεφάλαιο 80.000€ κατατέθηκε, σε λογαριασμό ταμιευτηρίου, με επιτόκιο 4,5\% το χρόνο.
(α) Ποιος θα είναι ο τόκος στο τέλος του πρώτου έτους; (β) Ποιος θα είναι ο τόκος
στο τέλος του δεύτερου έτους, αν ο τόκος του πρώτου έτους κεφαλοποιηθεί;
\end{exercise}

\begin{lstlisting}
>>> 80000*4.5/100
3600
>>> 80000+3600
83600.0
>>> 83600*4.5/100
3762.0
\end{lstlisting}

\begin{exercise}
\sel[3]{82}
Ένα καινούριο αυτοκίνητο κόστιζε 20.000€. Το αγόρασε κάποιος και μετά από 1
χρόνο ήθελε να το πουλήσει, κατά 30\% λιγότερο, από όσο το αγόρασε. Ο υποψήφιος
αγοραστής έμαθε, ότι το ίδιο ακριβώς μοντέλο, καινούριο, κόστιζε 25.000€. (α) Σε
ποια τιμή θα αγόραζε το μεταχειρισμένο αυτοκίνητο; (β) Τι ποσοστό της τιμής του
καινούριου αυτοκινήτου είναι η τιμή του μεταχειρισμένου; (γ) Αν ένα μαγαζί που πουλάει
μεταχειρισμένα αυτοκίνητα δίνει το ίδιο μοντέλο σε τιμή 40\% φτηνότερα από την
τρέχουσα τιμή του καινούριου, από ποιον συμφέρει να αγοράσει το μεταχειρισμένο
αυτοκίνητο ο υποψήφιος αγοραστής;
\end{exercise}

\begin{lstlisting}
>>> 20000-20000*30/100
14000
\end{lstlisting}
Το αυτοκίνητο το πουλάει 14.000€.

\begin{lstlisting}
>>> pososto(14000/25000)
56.00%
\end{lstlisting}

Είναι το 56\%.

\begin{lstlisting}
>>> 25000-25000*40/100
15000
\end{lstlisting}

Άρα το μεταχειρισμένο είνα φτηνότερο.

\begin{exercise}
\sel[4]{82}
Σε ένα προϊόν, έγινε η προσφορά που φαίνεται στην πινακίδα. Στη
συσκευασία του προϊόντος υπήρχε σημειωμένη η συγκεκριμένη, για το
είδος προσφορά, δηλαδή για κάθε 300 κ.εκ., πρόσθεσαν άλλα 100 κ.εκ.
(α) Σύμφωνα, με όσα διαβάζεις, θεωρείς ότι αληθεύουν όσα γράφονται
στην προσφορά; (β) Σε ποια περίπτωση η εταιρεία θα πρόσφερε,
πράγματι, το 50\% του προϊόντος ΔΩΡΕΑΝ;
\end{exercise}

\begin{lstlisting}
>>> pososto(100/300)
33.33%
\end{lstlisting}
Άρα δεν ισχύει. Το 50\% του 300 είναι:
\begin{lstlisting}
>>> 50/100*300
150
\end{lstlisting}
150κ.εκ.
\begin{exercise}
\sel[5]{82}
Τι κεφάλαιο πρέπει να καταθέσουμε στην τράπεζα, για να πάρουμε στο
τέλος ενός έτους 1.000€, αν το επιτόκιο είναι 2\%;
\end{exercise}
$$ x + x*2\% = 1000$$
1ος τρόπος:
Μπορείς να βάλεις την Python να λύσει την εξίσωση:
\begin{lstlisting}
>>> solve(x+x*2/100 - 1000)
[50000/51]
>>> 50000/51
980.3921568627451
\end{lstlisting}
Δηλαδή αν βάλει $980,39$€ θα έχει:
\begin{lstlisting}
>>> 980.39+980.39*2/100
999.9978
\end{lstlisting}
2ος τρόπος:
Μπορείς να λύσεις την εξίσωση ως εξής:
$$ x(1+\frac{2}{100}) = 1000$$
$$x\frac{102}{100}=1000$$
$$x = 1000\cdot\frac{1}{\frac{102}{100}}$$
$$x= 1000\frac{100}{102}$$
Οπότε και:
\begin{lstlisting}
>>> 1000*100/102
980.3921568627451
\end{lstlisting}

Αυτή η λειτουργικότητα μπορεί να υλοποιηθεί και με μια συνάρτηση. Αν θέλουμε ένα ποσό μετά από έναν χρόνο με δοσμένο επιτόκιο  ποιο ποσό πρέπει να επενδύσουμε;
\begin{lstlisting}
def arxiko(teliko,epitokio):
    return(teliko/(1+epitokio/100))

arxiko(1000,2)
\end{lstlisting}
Που δίνει αποτέλεσμα:
\begin{lstlisting}
980.3921568627451
\end{lstlisting}
\begin{exercise}
\sel[6]{83}
Τα βασικά τέλη διμήνου για λογαριασμό του ΟΤΕ είναι 22€ και η χρέωση για
κάθε μονάδα 0,07€. Να βρεις πόσο θα πληρώσει ένας συνδρομητής, αν έχει
κάνει 1.500 μονάδες συνδιαλέξεων και επί του συνόλου υπολογίζεται ΦΠΑ 19\%.
\end{exercise}
\begin{lstlisting}
>>> profpa = 22+1500*0.07
>>> fpa = profpa*19/100
>>> synolo = profpa + fpa
>>> synolo
151.13000000000002
\end{lstlisting}
Άρα 151,13€.
\begin{exercise}
\sel[7]{83}
Ένας έμπορος αγόρασε διάφορα εμπορεύματα συνολικής αξίας 30.000€. Πλήρωσε τοις
μετρητοίς το 40\% και τα υπόλοιπα με συναλλαγματικές, σε 4 μηνιαίες δόσεις με τόκο 1\% τον
μήνα. Να υπολογίσεις: (α) Το συνολικό ποσό της επιβάρυνσης από τους τόκους που θα
πληρώσει. (β) Το ποσοστό της επιβάρυνσης αυτής, επί της αρχικής αξίας των εμπορευμάτων.
\end{exercise}
\begin{lstlisting}
>>> ypoloipo = 30000*60/100
>>> anamina = ypoloipo/4
>>> plirwnei = anamina + 1/100*anamina
>>> epivarinsi = 4*plirwnei-ypoloipo
>>> epivarinsi
180.0
>>> pososto = epivarinsi/30000*100
>>> pososto
0.6
\end{lstlisting}
\begin{exercise}
\sel[8]{83}
Ένας τεχνικός είχε έσοδα σε ένα τρίμηνο 8.330€. Πόσο ΦΠΑ (19\%) πρέπει να αποδώσει
στην εφορία;
\end{exercise}
\begin{lstlisting}
>>> solve(x + 19*x/100-8330)
7000
\end{lstlisting}
Οπότε το ΦΠΑ που θα πρέπει να αποδώσει θα είναι:
\begin{lstlisting}
>>> 19/100*7000
1330.0
\end{lstlisting}
Που μπορεί να βρεθεί και σαν
\begin{lstlisting}
>>> 8330-7000
1330
\end{lstlisting}

\begin{exercise}
\sel[9]{83}
Ένα ψυγείο κοστίζει, τοις μετρητοίς, 1.200€ χωρίς το ΦΠΑ 19\%. Κάποιος το αγόρασε με 50\%
προκαταβολή και το υπόλοιπο, σε 6 μηνιαίες δόσεις με τόκο 3\% το μήνα. (α) Να υπολογίσεις
πόσα χρήματα έδωσε, ως προκαταβολή, αν μαζί με αυτήν κατέβαλε και ολόκληρο το ποσό του
ΦΠΑ. (β) Ποιο ήταν το ποσό της κάθε δόσης; (γ) Πόσο του στοίχισε συνολικά το ψυγείο;
\end{exercise}

\begin{lstlisting}
prokataboli = 50/100*1200+19/100*1200
ypoloipo = (1200+19/100*1200) - prokataboli
dosixwristokous = ypoloipo/6
plirwse = prokataboli
print('Προκαταβολη:',prokataboli)

for i in range(6):
    dosi = dosixwristokous + 3/100*ypoloipo
    plirwse += dosi
    ypoloipo = ypoloipo - dosixwristokous
    print('Δόση '+str(i)+': '+str(dosi))

print('Συνολικα:',plirwse)
\end{lstlisting}

Το αποτέλεσμα είναι:
\begin{lstlisting}
Προκαταβολη: 828.0
Δόση 0: 118.0
Δόση 1: 115.0
Δόση 2: 112.0
Δόση 3: 109.0
Δόση 4: 106.0
Δόση 5: 103.0
Συνολικα: 1491.0
\end{lstlisting}
\begin{exercise}
\sel[10]{83}
Για τη διπλανή διαφήμιση: (α) Πόσο είναι το ΦΠΑ που πρέπει να
πληρώσουμε; (β) Πόσο θα στοιχίσει το ραδιοκασετόφωνο, αν το
αγοράσουμε με δόσεις; (γ) Αν το τραπεζικό επιτόκιο είναι 10\%,
ποια επιλογή αγοράς μας συμφέρει, με την προυπόθεση, ότι
έχουμε όλο το απαιτούμενο ποσό σε λογαριασμό ταμιευτηρίου;
\end{exercise}
\begin{lstlisting}
>>> 350*19/100
66.5
\end{lstlisting}
Με τις δόσεις του καταστήματος θα πληρώσουμε
\begin{lstlisting}
>>> 30*16
480
>>> 480+66.5
546.5
\end{lstlisting}

Τοις μετρητοίς θα πληρώσουμε 
\begin{lstlisting}
>>> 350+66.5
416.5
\end{lstlisting}
Με 10\% επιτόκιο θα πληρώσουμε
\begin{lstlisting}
>>> 416.5+10/100*416.5
458.15
\end{lstlisting}
Αλλά αυτό θα είναι στο τέλος του χρόνου οπότε μένουν άλλοι 4 μήνες:
\begin{lstlisting}
>>> 458.15+4/12*10/100*458.15
473.42
\end{lstlisting}
Άρα συμφέρει να πάρουμε την προσφορά της τράπεζας και να πληρώσουμε μετρητοίς στο κατάστημα.



