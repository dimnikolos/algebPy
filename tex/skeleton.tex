\documentclass[a4paper,12pt]{memoir}
\usepackage{fontspec}
\usepackage{courier}
\usepackage{polyglossia}
\usepackage{hyperref}
\usepackage{graphicx}
\usepackage{listings}
\usepackage{xcolor}
\usepackage{listings}
\usepackage{amsmath}
\usepackage{amssymb}
\usepackage{xparse}
\lstset{basicstyle=\ttfamily\footnotesize,language=Python,backgroundcolor = \color{lightgray},showstringspaces=false,columns=fullflexible,showstringspaces=false, morestring=[b]", morestring=[b]',keepspaces=true,
escapeinside={(*@}{@*)}}
\newtheorem{theorem}{Theorem}[section]
\newtheorem{exercise}[theorem]{Άσκηση}
\setmainlanguage{greek}
\setromanfont{Candara}
\setsansfont{Calibri}
\setmonofont{Consolas}

%\newcommand{\sel}[1] {(Στο βιβλίο βρίσκεται στη Σελ. #1)}
\NewDocumentCommand\sel{om}{%
  \IfNoValueTF{#1}
    {(Στο βιβλίο βρίσκεται στη Σελ. #2)}%
    {(Άσκηση #1 του βιβλίου, Σελ. #2)}
}


\begin{document}
\title{%
%$A\lambda\gamma\epsilon\beta\rho y$ \\ 
Μαθηματικά Γυμνασίου με Python}

\author{Δημήτρης Νικολός}
\maketitle
%\chapter{Τι θα χρησιμοποιήσουμε;}
\section{Η γλώσσα προγραμματισμού Python}
Σε αυτές τις σημειώσεις θα χρησιμοποιήσουμε τη γλώσσα προγραμματισμού Python και μάλιστα την έκδοση 3. Υπάρχει και Python 2 αλλά υπάρχουν σχέδια για την αντικατάστασή της από την Python 3. Για να εγκαταστήσεις την Python 3 θα πρέπει να την κατεβάσεις από το επίσημο site της Python \href{https://www.python.org/}{www.python.org}. Κατεβάστε την πιο πρόσφατη έκδοση που σας προτείνει θα είναι κάτι σαν 3.8.2 ή κάτι 

\section{Ο επεξεργαστής προγραμμάτων Mu}
Μπορείς να γράψεις Python σε οποιοδήποτε πρόγραμμα υποστηρίζει απλό κείμενο, ακόμη και στο Σημειωματάριο, όμως σε αυτές τις σημειώσεις χρησιμοποιούμε τον επεξεργαστή Python, Mu Editor ή πιο απλά Mu που μπορείς να τον κατεβάσεις από τη σελίδα \href{https://codewith.mu/}{codewith.mu}. Μόλις το ανοίξεις θα δεις την εικόνα \ref{Mu}. 
\begin{figure}
\includegraphics[width=\textwidth]{mu.png}
\label{Mu}
\caption{Mu: Ένας επεξεργαστής προγραμμάτων Python}
\end{figure}

Μπορείς να πατήσεις την εκτέλεση (κουμπί Run) και τότε θα δεις ότι το παράθυρο χωρίζεται σε δύο τμήματα (Εικόνα \ref{Mu2}).  Αν θες να δοκιμάσεις ένα ολόκληρο πρόγραμμα μπορείς να το πληκτρολογήσεις στο βασικό παράθυρο (τώρα γράφει `\#Write your code here`). Ενώ αν θες να δοκιμάσεις κάποια εντολή τότε μπορείς να την πληκτρολογήσεις στο κάτω παράθυρο (τώρα γράφει $>>>$).  Το κάτω παράθυρο ονομάζεται REPL, από τα αρχικά των λέξεων Read, Eval, Print, Loop δηλαδή Διάβασε, Εκτέλεσε (την εντολή/έκφραση), Τύπωσε, Επανάλαβε. Το REPL θα διαβάσει την εντολή, θα την εκτελέσει και θα μας δώσει το αποτέλεσμα.

Από εδώ και πέρα όταν βλέπετε στις σημειώσεις τα τρία σύμβολα ``μεγαλύτερο από'' ($>>>$) θα πληκτρολογείτε τις αντίστοιχες εντολές στο κάτω παράθυρο (REPL). Τα μεγαλύτερα προγράμματα που δεν θα έχουν αυτό το σύμβολο θα τα πληκτρολογείτε στο πάνω παράθυρο.

\fbox{
	\parbox{0.8\textwidth}{%
	\textbf{Συμβουλή:} Αν χρησιμοποιείτε την ηλεκτρονική έκδοση αυτών των σημειώσεων, θυμηθείτε να πληκτρολογείτε τις εντολές και να μην τις κάνετε αντιγραφή επικόλληση.
	}
}


Στην αρχή θα δοκιμάσεις κάποια πράγματα στο κάτω παράθυρο, όμως μην ανησυχείς σύντομα θα γράφεις τα δικά σου προγράμματα στο πάνω παράθυρο.

\begin{figure}
\includegraphics[width=\textwidth]{mu2.png}
\label{Mu2}
\caption{Το πρόγραμμα Mu όταν εκτελείτε ένας κώδικας}
\end{figure}

\section{Το βιβλίο μαθηματικών της Α΄ Γυμνασίου}
Σε αυτές τις σημειώσεις οι περισσότερες ασκήσεις είναι από το βιβλίο Μαθηματικών της Α΄ Γυμνασίου των Βανδουλάκη, Καλλιγά, Μαρκάκη και Φερεντίνου (Εικόνα \ref{matha}).

\begin{figure}
\centering
\includegraphics[width=0.8\textwidth]{matha.jpg}
\label{matha}
\caption{Το εξώφυλλο του βιβλίου των Μαθηματικών που θα χρησιμοποιήσουμε}
\end{figure}
%\chapter{Φυσικοί αριθμοί}

\section{Οι αριθμοί και η Python}

Οι φυσικοί αριθμοί είναι οι αριθμοί από 0, 1, 2, 3, 4, 5, 6, \ldots, 98, 99, 100, \ldots, 1999, 2000, 2001, \ldots

Η Python μπορεί να χειριστεί φυσικούς αριθμούς. Δοκιμάστε να γράψετε στο REPL έναν φυσικό αριθμό, θα δείτε ότι η Python θα τον επαναλάβει. Π.χ. δείτε τον αριθμό εκατόν είκοσι τρια (123).
\begin{lstlisting}
>>> 123
123
\end{lstlisting}

Στην Python όμως θα πρέπει να ακολουθείς κάποιους επιπλέον κανόνες. Για παράδειγμα στους αριθμούς δεν πρέπει να βάζεις τελείες στις χιλιάδες όπως στο χαρτί. Αν το κάνεις στην καλύτερη περίπτωση θα προκύψει κάποιο λάθος, στην χειρότερη ο υπολογιστής θα καταλάβει διαφορετικό αριθμό από αυτόν που εννοείς.
Δείτε το παρακάτω παράδειγμα στο REPL.
\begin{lstlisting}
>>> 1.000.000
  File "<stdin>", line 1
    1.000.000
            ^
SyntaxError: invalid syntax
>>> 100.000
100.0
\end{lstlisting}
Σε αυτό το παράδειγμα, η Python δεν καταλαβαίνει καθόλου τον αριθμό 1.000.000 γραμμένο με τελείες ενώ μεταφράζει το 100.000 σε 100.0, που για την Python σημαίνει 100 (εκατό). Γι' αυτόν τον λόγο δεν βάζουμε καθόλου τελείες έτσι αν θέλουμε να γράψουμε το ένα εκατομμύριο θα γράψουμε 1000000.
\begin{lstlisting}
>>> 1000000
1000000
\end{lstlisting}

\section{Πρόσθεση, αφαίρεση και πολλαπλασιασμός φυσικών αριθμών}
Μια γλώσσα προγραμματισμού μπορεί να εκτελέσει απλές πράξεις πολύ εύκολα. Στο βιβλίο των μαθηματικών  σου μπορείς να βρεις πολλές ασκήσεις με πράξεις. Μπορείς να τις λύσεις με την Python.

\begin{exercise}
\sel{16}
Να υπολογιστούν τα γινόμενα: 

(α) $35 \cdot 10$, 

(β) $421 \cdot 100$,

(γ) $5 \cdot 1.000$,

(δ) $27 \cdot 10.000$
\end{exercise}

Η python μπορεί να κάνει αυτές τις πράξεις ως εξής:
\begin{lstlisting}
>>> 35*10
350
>>> 421*100
42100
>>> 5*1000
5000
>>> 27*10000
270000
\end{lstlisting}

Ο τελεστής του πολλαπλασιασμού είναι το αστεράκι * (SHIFT+8) στο πληκτρολόγιο. Εναλλακτικά, μπορείτε να το βρείτε στο αριθμητικό πληκτρολόγιο. 

\begin{exercise}
\sel{16}
Να εκτελεστούν οι ακόλουθες πράξεις:

(α) $89\cdot 7 + 89\cdot 3$

(β) $23 \cdot 49 + 77 \cdot 49$

(γ) $76 \cdot 13 – 76 \cdot 3$

(δ) $284 \cdot 99$
\end{exercise}
\begin{lstlisting}
>>> 89*7+89*3
890
>>> 23*49+77*49
4900
>>> 76*13-76*3
760
>>> 284*99
28116
\end{lstlisting}

Στις παραπάνω περιπτώσεις η python εκτελεί πρώτα τους πολλαπλασιασμούς και μετά τις προσθέσεις/αφαιρέσεις δίνοντας έτσι το αποτέλεσμα που αναμένεται. Για παράδειγμα 89\*7 + 89\*3 = 623 + 267 = 890, που είναι το σωστό αποτέλεσμα.

\begin{exercise}
\sel{18}
Υπολογίστε:

(α)  $157 + 33$ 

(β)  $122 + 25 + 78$

(γ)  $785 - 323$

(δ)  $7.321 - 4.595$

(ε)  $60 - (18 - 2)$

(στ) $52 - 11 -9$

(ζ)  $23 \cdot 10$

(η)  $97 \cdot 100$

(θ)  $879 \cdot 1.000$
\end{exercise}
Σε python τα παραπάνω υπολογίζονται ως εξής:
\begin{lstlisting}
>>> 157+33
190
>>> 122+25+78
225
>>> 785-323
462
>>> 7321-4595
2726
>>> 60-(18-2)
44
>>> 52-11-9
32
>>> 23*10
230
>>> 97*100
9700
>>> 879*1000
879000
\end{lstlisting}
Οι παρενθέσεις (SHIFT+9 και SHIFT+0) αλλάζουν τη σειρά των πράξεων. Οι πράξεις που είναι μέσα στην παρένθεση εκτελούνται πρώτες. Γι' αυτό το λόγο 60-(18-2)=60-16=44.

\begin{exercise}
\sel{18}
Σε ένα αρτοποιείο έφτιαξαν μία μέρα 120 κιλά άσπρο ψωμί, 135 κιλά χωριάτικο, 25 κιλά σικάλεως και 38 κιλά πολύσπορο. Πουλήθηκαν 107 κιλά άσπρο ψωμί, 112 κιλά χωριάτικο, 19 κιλά σικάλεως και 23 κιλά πολύσπορο. Πόσα κιλά ψωμί έμειναν απούλητα;
\end{exercise}
Με τις γνώσεις που έχουμε θα πρέπει να μετατρέψουμε το παραπάνω πρόβλημα σε μια αριθμητική παράσταση ώστε η python να μπορεί να την υπολογίσει, στη συγκεκριμένη περίπτωση η σωστή παράσταση είναι $$(120-107)+(135-112)+(25-19)+(38-23)$$
\begin{lstlisting}
>>> (120-107)+(135-112)+(25-19)+(38-23)
57
\end{lstlisting}
και η απάντηση είναι 57 κιλά ψωμί.

\section{Δυνάμεις φυσικών αριθμών}
Ο τελεστής της python για τις δυνάμεις είναι ο **  (δυο φορές το αστεράκι). Δηλαδή, αν θέλουμε να υπολογίσουμε το $10^2$ θα γράψουμε 10**2, με όμοιο τρόπο μπορούμε να υπολογίσουμε και τις υπόλοιπες δυνάμεις. Δοκίμασε τα παρακάτω στο REPL.
\begin{lstlisting}
>>> 10**2
100
>>> 10**3
1000
>>> 10**4
10000
>>> 10**5
100000
>>> 10**6
1000000
\end{lstlisting}
Στη προτεραιότητα των πράξεων, οι δυνάμεις έχουν μεγλύτερη προτεραιότητα από τον πολλαπλασιασμό και την πρόσθεση. Οπότε όταν έχουμε και δυνάμεις σε μια παράσταση πρώτα γίνονται οι πράξεις στις παρενθέσεις, μετά οι δυνάμεις και μετά οι πολλαπλασιασμοί και οι προσθέσεις.
\begin{exercise}
\sel{21}
Να εκτελεστούν οι πράξεις 

 1. $(2\cdot 5)^4+4\cdot (3+2)^2$

 2. $(2+3)^3 - 8\cdot 3^2$

\end{exercise}
Οι αντίστοιχες εκφράσεις είναι (2*5)**4+4*(3+2)**2 και (2+3)**3 - 8*3**2.

\begin{lstlisting}
>>> (2*5)**4+4*(3+2)**2
10100
>>> (2+3)**3 - 8*3**2
53
\end{lstlisting}
H 8*3**2 υπολογίζεται ως $8\cdot (3^2)$, δηλαδή $8\cdot 9 = 72$, αφού πρώτα γίνεται η δύναμη και μετά οι πολλαπλασιασμοί.

\section{Συγκρίσεις φυσικών αριθμών}
Μπορούμε να συγκρίνουμε αριθμούς στην Python χρησιμοποιώντας τους τελεστές == (πληκτρολογούμε δύο φορές το =) για την \emph{ισότητα}, > για το \emph{μεγαλύτερο} και < για το \emph{μικρότερο}. Επίσης μπορούμε να χρησιμοποιήσουμε >= για το \emph{μεγαλύτερο ή ίσο} και <= για το \emph{μικρότερο ή ίσο}, τέλος υπάρχει το != για το \emph{δεν είναι ίσο}. Μπορείς να δοκιμάσεις τα παρακάτω:
\begin{lstlisting}
>>> 123==123
True
>>> 123>123
False
>>> 123>122
True
>>> 123<123
False
>>> 123<124
True
>>> 123<=123
True
>>> 123<=124
True
>>> 123<=122
False
>>> 123>=123
True
>>> 123>=124
False
>>> 123>=122
True
>>> 122 != 123
True
>>> 122 != 122
False
\end{lstlisting}
Η Python επιστρέφει True (αληθές) όταν μία πρόταση ισχύει και False (ψευδές) όταν δεν ισχύει.

Σκέψου ότι για την Python η σύγκριση είναι και αυτή μια πράξη. Αντί η πράξη αυτή να δίνει σαν αποτέλεσμα έναν αριθμό δίνει σαν αποτέλεσμα το αληθές ή το ψευδές.

Για παράδειγμα:
\begin{exercise}
Να συγκρίνετε τα $3^2$ και $2^3$.
\end{exercise}
Η σύγκριση αυτή μπορεί να γίνει στο REPL. Δοκίμασε:
\begin{lstlisting}
>>> 3**2 > 2**3
True
\end{lstlisting}
Άρα το $3^2$ είναι μεγαλύτερο από το $2^3$. Θυμήσου ότι το $3^2=9$, ενώ $2^3=8$.
\begin{exercise}
\end{exercise}

\section{Η εντολή print}
Ήρθε η ώρα να γράψεις εντολές στο πάνω παράθυρο, δηλαδή να γράψεις το πρώτο σου πρόγραμμα.  Με βάση όσα ξέρεις προσπάθησε να γράψεις μια πράξη στο πάνω παράθυρο, για παράδειγμα $32+35$. Ύστερα πάτησε το κουμπί της εκτέλεσης (Run). Μπορείς να δεις το αποτέλεσμα στην εικόνα \ref{noprint}.
\begin{figure}
\includegraphics[width=\textwidth]{noprint.png}
\caption{Η εκτέλεση δεν δίνει κάποιο αποτέλεσμα}
\end{figure}

Η Python εκτελεί την πράξη $32+35$, και υπολογίζει το αποτέλεσμα. Αν δεν το έκανε και υπήρχε κάποιο πρόβλημα θα εμφάνιζε κάποιο μήνυμα λάθους στο REPL. Το αποτέλεσμα όμως δεν εμφανίζεται. Για να εμφανιστεί το αποτέλεσμα πρέπει να χρησιμοποιήσεις την εντολή print (εκτύπωσε). Η εντολή print εκτελείτε ως εξής:
\begin{lstlisting}
print(32+35)
\end{lstlisting}
Γράφουμε δηλαδή, print ανοίγουμε παρένθεση, γράφουμε αυτό που θέλουμε να εκτυπωθεί και κλείνουμε την παρένθεση. Όταν εκτελέσουμε το πρόγραμμα με την print τότε εμφανίζεται το αποτέλεσμα στο REPL (εικόνα \ref{withprint}).
\begin{figure}
\includegraphics[width=\textwidth]{withprint.png}
\caption{Η εκτέλεση δίνει το αποτέλεσμα της πράξης}
\end{figure}
\emph{Μπράβο!} Μόλις έγραψες το πρώτο σου πρόγραμμα στην Python. Μάλιστα το πρόγραμμά σου κάνει κάτι. Υπολογίζει το αποτέλεσμα της πράξης $32+35$.

\section{Απαρίθμηση}
Είδαμε ότι η Python μπορεί να κάνει πολύ γρήγορα, πολύπλοκες πράξεις ακόμη και με δυνάμεις, αλλά δεν είδαμε ακόμη τις απλές ασκήσεις που υπάρχουν στις πρώτες σελίδες του βιβλίου. Όπως για παράδειγμα ποιοι είναι οι τρεις προηγούμενοι αριθμοί του 289 και ποιο οι δύο επόμενοι (\sel{13}).

Τώρα που μάθαμε να γράφουμε προγράμματα σε Python μπορούμε να αντιμετωπίσουμε αυτό το πρόβλημα με το παρακάτω πρόγραμμα:
\begin{lstlisting}
print(289-3)
print(289-2)
print(289-1)
print(289+1)
print(289+2)
\end{lstlisting}
που δίνει το αποτέλεσμα
\begin{lstlisting}
286
287
288
290
291
\end{lstlisting}

Πιο σωστό θα ήταν να γράψουμε ποιοι αριθμοί είναι οι προηγούμενοι και ποιοι οι επόμενοι. Σε αυτή την περίπτωση θα γράψουμε τις παρακάτω εντολές.
\begin{lstlisting}
print("Οι  προηγούμενοι αριθμοί είναι:")
print(289-3)
print(289-2)
print(289-1)
print("Οι επόμενοι αριθμοί είναι:")
print(289+1)
print(289+2)
\end{lstlisting}

Για να εμφανίσει η print τις λέξεις που θέλουμε πρέπει να τις βάλουμε μέσα σε εισαγωγικά. Η Python υποστηρίζει είτε μονά εισαγωγικά, είτε διπλά. Αυτά εισάγονται συνήθως με το ίδιο κουμπί του πληκτρολογίου (κοντά στο ENTER), είτε με SHIFT ή χωρίς. Θυμήσου να κλείνεις τα εισαγωγικά με τον ίδιο τρόπο που τα άνοιξες. Στο πρόγραμμα Mu τα εισαγωγικά αυτά δεν φαίνονται όπως σε άλλα πρόγραμματα σαν `Εισαγωγικά' ή ``Εισαγωγικά " ή <<Εισαγωγικά>>, αλλά φαίνονται κάπως πιο απλά και ίδια στο άνοιγμα και το κλείσιμο \lstinline{'Εισαγωγικά'} ή  \lstinline{"Εισαγωγικά"}. 

Αν θέλουμε να αλλάξουμε το 289 και να βάλουμε έναν άλλο αριθμό,π.χ. το 132 θα πρέπει να αντικαταστήσουμε το 289 μέσα σε όλες τις εντολές print με το 132.
\begin{lstlisting}
print("Οι προηγούμενοι αριθμοί είναι:")
print(132-3)
print(132-2)
print(132-1)
print("Οι επόμενοι αριθμοί είναι:")
print(132+1)
print(132+2)
\end{lstlisting}

Υπάρχει όμως ένας καλύτερος τρόπος, ο τρόπος αυτός είναι να δώσουμε ένα όνομα στον αριθμό μας. Μπορούμε να πούμε ότι το n είναι το όνομα του αριθμού. Αυτό γίνεται με την εντολή \lstinline{n=132}. Τότε το πρόγραμμά μας γίνεται:
\begin{lstlisting}
n = 132
print("Οι προηγούμενοι αριθμοί είναι:")
print(n-3)
print(n-2)
print(n-1)
print("Οι επόμενοι αριθμοί είναι:")
print(n+1)
print(n+2)
\end{lstlisting}

Μετά την εντολή \lstinline{n=132} η Python ξέρει ότι το n είναι ένα όνομα για το 132 και μπορεί να κάνει πράξεις με αυτό. Για παράδειγμα n+1 κάνει τώρα 133.

Αν θέλουμε να κάνουμε τώρα το ίδιο πρόγραμμα αλλά όχι για το 132 αλλά για το 210, χρειάζεται να αλλάξουμε μόνο μία γραμμή και το πρόγραμμά μας να γίνει ως εξής:
\begin{lstlisting}
n = 210
print("Οι προηγούμενοι αριθμοί είναι:")
print(n-3)
print(n-2)
print(n-1)
print("Οι επόμενοι αριθμοί είναι:")
print(n+1)
print(n+2)
\end{lstlisting}

Στην Python, όταν δίνουμε ένα όνομα σε έναν αριθμό (με τον τελεστή =) τότε δημιουργούμε μια μεταβλητή. Η μεταβλητή έχει ένα όνομα, στην περίπτωσή μας το n, και μια τιμή, στην περίπτωσή μας το 210.

Αν αντί για τους επόμενους δύο αριθμούς θέλαμε τους επόμενους \textbf{δέκα} θα γράφαμε ένα πρόγραμμα όπως το παρακάτω:
\begin{lstlisting}
n = 210
print(n)
print(n+1)
print(n+2)
print(n+3)
print(n+4)
print(n+5)
print(n+6)
print(n+7)
print(n+8)
print(n+9)
print(n+10)
\end{lstlisting}
Το παραπάνω πρόγραμμα εμφανίζει και τον αριθμό μας n, δηλαδή το 210.

Για να μην γράφουμε πολλές εντολές όταν κάνουμε το ίδιο πράγμα χρησιμοποιούμε την εντολή for.
Το πρόγραμμά μας με την for μπορεί να γίνει:
\begin{lstlisting}
n = 210
for i in 0,1,2,3,4,5,6,7,8,9,10:
    print(n+i)
\end{lstlisting}
Όταν γράψεις την for στην Python θα πρέπει να δηλώσεις ποιες εντολές θα εκτελεστούν πολλές φορές. Αυτή η δήλωση γίνεται βάζοντας αυτές τις εντολές λίγο πιο μέσα χρησιμοποιώντας το πλήκτρο κενό ή το πλήκτρο tab. Μια καλή πρακτική είναι να βάζεις τέσσερα κενά. Έτσι, πριν την εντολή \lstinline{print(n+i)} βάζεις τέσσερα κενά δηλαδή \lstinline[showspaces=true]{    print(n+i)}.
Το πρόγραμμα αυτό σημαίνει πως για το i μέσα στο σύνολο 0, 1, 2, 3, \ldots 10 και με αυτή τη σειρά εμφάνισε το n+i. Έτσι το αποτέλεσμα είναι το αναμενόμενο
\begin{lstlisting}
210
211
212
213
214
215
216
217
218
219
220
\end{lstlisting}

Στην Python υπάρχει ένας πιο εύκολος τρόπος να γράψουμε τους αριθμούς από το 0 έως το 10. Αυτός ο τρόπος είναι η εντολή range και συγκεκριμένα η range(11). Η range(11) φτιάχνει τους αριθμούς από το 0 μέχρι το 10 οι οποίοι είναι σε πλήθος 11. 
Έτσι το πρόγραμμά μας γίνεται:
\begin{lstlisting}
n = 210
for i in range(11):
    print(n+i)
\end{lstlisting}

Mπορούμε και να μετρήσουμε τους πρώτους 100 αριθμούς ως εξής:
\begin{lstlisting}
n = 210
for i in range(100):
    print(n+i)
\end{lstlisting}

Σκέψου αν θα δεις τον αριθμό 100 στο αποτέλεσμα του παραπάνω προγράμματος.

\section{Στρογγυλοποίηση}
Το βιβλίο των Μαθηματικών της Α' Γυμνασίου αναφέρει πως
Για να στρογγυλοποιήσουμε έναν φυσικό αριθμό \sel{12}:
\begin{enumerate}
	\item Προσδιορίζουμε την τάξη στην οποία θα γίνει η στρογγυλοποίηση
	\item Εξετάζουμε το ψηφίο της αμέσως μικρότερης τάξης
	\item Αν αυτό το ψηφίο είναι μικρότερο του 5 (δηλαδή 0, 1, 2, 3 ή 4) το ψηφίο αυτό και όλα τα ψηφία των υπόλοιπων τάξεων μηδενίζονται.
	\item Αν είναι μεγαλύτερο ή ίσο του 5 (δηλαδή 5, 6, 7, 8 ή 9) το ψηφίο αυτό και όλα τα ψηφία των υπόλοιπων τάξεων αντικαθίστανται από το 0 και το ψηφίο της τάξης στρογγυλοποίησης αυξάνεται κατά 1.
\end{enumerate}

Ας πούμε ότι θέλουμε να στρογγυλοποιήσουμε τον αριθμό 454.018.512 στα εκατομμύρια. Η απάντηση που περιμένουμε είναι 454 εκατομμύρια.
Για να τα καταφέρουμε θα χρησιμοποιήσουμε την διαίρεση. Όμως στην Python υπάρχουν \emph{δύο} διαιρέσεις μία με το σύμβολο / και μία με το σύμβολο //. Ας δούμε τις διαφορές τους στο REPL.
\begin{lstlisting}
>>> x = 454018512
>>> print(x/1000000)
454.018512
>>> print(x//1000000)
454
\end{lstlisting}
Η <<κανονική>> διαίρεση, με τη μία κάθετο /, δίνει το αποτέλεσμα της διαίρεσης με τα δεκαδικά ψηφία. Η <<ακέραια>> διαίρεση δίνει μόνο τον ακέραιο αριθμό. Δεν μπορούμε να πούμε ότι η ακέραια διαίρεση θα μας δώσει την στρογγυλοποίηση γιατί η ακέραια διαίρεση δεν στρογγυλοποιεί τα δεκαδικά ψηφία αλλά τα απορρίπτει εντελώς. Έτσι, ακόμη και αν είχαμε 454918512 κατοίκους η ακέραια διαίρεση θα δώσει 454 αντί για το στρογγυλοποιημένο που είναι 455.
\begin{lstlisting}
>>> x = 454918512
>>> print(x/1000000)
454.918512
>>> print(x//1000000)
454
\end{lstlisting}

Χρειάζεται επομένως να δούμε το ψηφίο της αμέσως χαμηλότερης τάξης το οποίο είναι το πρώτο δεκαδικό της κανονικής διαίρεσης. Για να το απομονώσουμε αφαιρούμε από το αποτέλεσμα της κανονικής διαίρεσης το ακέραιο μέρος.
\begin{lstlisting}
>>> x = 454018512
>>> x / 1000000 - x // 1000000
0.018511999999986983
\end{lstlisting}
Οπότε τώρα έχουμε δύο ενδεχόμενα αν το αποτέλεσμα αυτής της πράξης είναι μικρότερο από 0.5 όπως παραπάνω τότε το αποτέλεσμα που ψάχνουμε είναι το αποτέλεσμα της ακέραιας διαίρεσης. Αλλιώς πρέπει να προσθέσουμε ένα στο αποτέλεσμα της ακέραιας διαίρεσης.
Αυτό γίνεται με την εντολή if, που σημαίνει στα αγγλικά αν. Για ευκολία μπορούμε να ονομάσουμε d την διαφορά των δύο διαιρέσεων με την εντολή:
\begin{lstlisting}
d = x / 1000000 - x // 1000000
\end{lstlisting}
Επειδή το πρόγραμμα γίνεται μεγαλύτερο τώρα θα το γράψουμε στο πάνω παράθυρο του Mu.
\begin{lstlisting}
x = 454018512
d = x / 1000000 - x // 1000000
if d < 0.5:
    print(x // 1000000)
else:
    print(x // 1000000 + 1)
\end{lstlisting}
Την \lstinline{if} την γράφουμε ως εξής:
\begin{lstlisting}
if συνθήκη:
    εντολές που εκτελούνται
    αν ισχύει η συνθήκη
else:
    εντολές που εκτελούνται
    αν δεν ισχύει η συνθήκη
\end{lstlisting}
Θυμήσου να βάζεις την άνω κάτω τελεία μετά τη συνθήκη και μετά τη λέξη else που σημαίνει αλλιώς.

Αν στο ίδιο πρόγραμμα και βάλεις αντί για 454.018.512 τον αριθμό 454.918.512 θα δεις ότι θα εμφανιστεί το σωστό αποτέλεσμα (455).

Αν θέλεις στρογγυλοποίηση στις χιλιάδες τότε το πρόγραμμά σου γίνεται:
\begin{lstlisting}
x = 454018512
d = x / 1000 - x // 1000
if d < 0.5:
    print(x // 1000)
else:
    print(x // 1000 + 1)
\end{lstlisting}
και το αποτέλεσμα είναι 454019.

\begin{exercise}
Στο βιβλίο \sel{12} η στρογγυλοποίηση γίνεται στις δεκάδες των εκατομμυρίων. Μπορείς να γράψεις ένα πρόγραμμα που να στρογγυλοποιεί αριθμούς στις δεκάδες των εκατομμυρίων;
\end{exercise}

\egyptify{1}{1}{1}{1}{1}{1}{1}


Ανάπτυγμα
%\input{Kef3.tex}
%\chapter{Δεκαδικοί αριθμοί}
\section{Εισαγωγή}
Αν χωρίσουμε τη μονάδα σε 10 ίσα μέρη τότε μπορούμε να πάρουμε κλάσματα της μονάδας όπως $\frac{3}{10}$, $\frac{5}{10}$ κλπ. Τα κλάσματα είναι ομώνυμα συγκρίνονται εύκολα και βοηθάνε στις πράξεις. 
Γενικότερα, ονομάζουμε δεκαδικό κλάσμα οποιδήποτε κλάσμα έχει παρονομαστή μια δύναμη του 10. Κάθε δεκαδικό κλάσμα γράφεται σαν δεκαδικός αριθμός με τόσα δεκαδικά ψηφία όσα μηδενικά έχει ο παρονομαστής του.
Η Python χειρίζεται τους δεκαδικούς αριθμούς όπως και τους υπόλοιπους.
Δοκίμασε:
\begin{lstlisting}
>>> 0.3 + 0.5
0.8
>>> type(0.7)
<class 'float'>
\end{lstlisting}

Βλέπουμε ότι οι δεκαδικοί αριθμοί δεν είναι int, όπως οι ακέραιο αλλά float. Το όνομα float έχει να κάνει με τον τρόπο με τον οποίο ο υπολογιστής αποθηκεύει αποδοτικά αυτούς τους αριθμούς. 

Ας συνδυάσουμε τις γνώσεις από τα κλάσματα με τα κλάσματα που μάθαμε στο προηγούμενο κεφάλαιο.
\begin{lstlisting}
>>> from fractions import Fraction
>>> x = Fraction(3,10)
>>> float(x)
0.3
\end{lstlisting}

Το \lstinline{Fraction(3,10)} εννοεί το κλάσμα $\frac{3}{10}$ που είναι ίσο με 0,3. Όμως στην Python το 0,3 θα το γράφουμε με 0.3. Με τη συνάρτηση float μετατρέπουμε το $\frac{3}{10}$ σε δεκαδικό αριθμό.

\begin{exercise}
\sel{56} Γράψτε τους αριθμούς $\frac{3}{10}$, $\frac{825}{1000}$, $\frac{53}{1000}$, $\frac{1004}{10000}$.
\end{exercise}
\begin{lstlisting}
>>> float(Fraction(3,10))
0.625
>>> float(Fraction(825,100))
8.25
>>> float(Fraction(53,1000))
0.053
>>> float(Fraction(1004,10000))
0.1004
\end{lstlisting}

Η Python μπορεί να μετατρέψει τα κλάσματα σε δεκαδικό αριθμό ανεξάρτητα από τον παρονομαστή.
\begin{exercise}
\sel{59} Γράψε καθένα από τα παρακάτω κλάσματα, ως δεκαδικό αριθμό: (i) με προσέγγιση
εκατοστού και (ii) με προσέγγιση χιλιοστού: 

(α) $\frac{7}{16}$

(β) $\frac{21}{17}$

(γ) $\frac{20}{95}$
\end{exercise}
\begin{lstlisting}
>>> x = Fraction(7,16)
>>> float(x)
0.4375
>>> round(float(x),2)
0.44
>>> round(float(x),3)
0.438
>>> x = Fraction(21,17)
>>> float(x)
1.2352941176470589
>>> round(float(x),2)
1.24
>>> round(float(x),3)
1.235
>>> x = Fraction(20,95)
>>> float(x)
0.21052631578947367
>>> round(float(x),2)
0.21
>>> round(float(x),3)
0.211
\end{lstlisting}


Η στρογγυλοποίηση των δεκαδικών υλοποιείται στην Python με τη συνάρτηση round. Οπότε μπορείς να στρογγυλοποιήσεις εύκολα δεκαδικούς αριθμούς ως εξής:
\begin{exercise}
Να στρογγυλοποιήσεις τους παρακάτω δεκαδικούς αριθμούς στο δέκατο, εκατοστό και
χιλιοστό: 

(α) 9876,008, 

(β) 67,8956, 

(γ) 0,001, 

(δ) 8,239, 

(ε) 23,7048.
\end{exercise}
Θυμόμαστε να αλλάζουμε την υποδιαστολή από κόμμα σε τελεία:
\begin{lstlisting}
def roundall(x):
    print(round(x,1))
    print(round(x,2))
    print(round(x,3))

roundall(9876.008)
roundall(67.8956)
roundall(0.001)
roundall(8.239)
roundall(23.7048)
\end{lstlisting}

To αποτέλεσμα είναι:
\begin{lstlisting}
67.9
67.9
67.896
0.0
0.0
0.001
8.2
8.24
8.239
23.7
23.7
23.705
\end{lstlisting}

\begin{exercise}
\sel{59} Στον αριθμό $34,\square\square\square$ λείπουν τα τελευταία τρία ψηφία του. Να συμπληρώσεις τον
αριθμό με τα ψηφία 9, 5 και 2, έτσι ώστε κάθε ψηφίο να γράφεται μία μόνο φορά. Να γράψεις όλους τους δεκαδικούς που μπορείς να βρεις και να τους διατάξεις σε φθίνουσα σειρά.
\end{exercise}

Πώς μπορεί η Python να βρει όλους τους πιθανούς συνδυασμούς του 9,5,2;
Δοκίμασε τη βιβλιοθήκη itertools και συγκεκριμένα τη συνάρτηση permutations.
\begin{lstlisting}
>>> from itertools import permutations
>>> x = permutations([1,2,3])
>>> print(x)
<itertools.permutations object at 0x012BE1B0>
>>> print(list(x))
[(1, 2, 3), (1, 3, 2), (2, 1, 3), (2, 3, 1), (3, 1, 2), (3, 2, 1)]
\end{lstlisting}
Έτσι με την permutations μπορείς να βρεις όλες τις αναδιατάξεις των αριθμών. Οπότε τώρα το πρόγραμμα μπορεί να γίνει ως εξής:
\begin{lstlisting}
lista = []
from itertools import permutations
for p in permutations([9,5,2]):
    lista.append(34+p[0]/10+p[1]/100+p[2]/1000)
print(lista)
\end{lstlisting}
Που δίνει το αποτέλεσμα:
\begin{lstlisting}
[34.952, 34.925000000000004, 34.592000000000006, 34.529, 
34.29500000000001, 34.259]
\end{lstlisting}
Τα ψηφία που εμφανίζονται στο τέλος των αριθμών προκύπτουν από την αναπαράσταση των δεκαδικών στον υπολογιστή που υπόκειται σε κάποιους περιορισμούς.
Αν δεν θέλουμε να εμφανίζονται μπορούμε να αλλάξουμε το for σε:
\begin{lstlisting}
for p in permutations([9,5,2]):
    ar = 34+p[0]/10+p[1]/100+p[2]/1000
    lista.append(round(ar,3))
\end{lstlisting}
Τώρα για να γράψουμε τους αριθμούς με φθίνουσα σειρά θα δοκιμάσουμε τη sorted. Η sorted ταξινομεί τους αριθμούς που δίνονται σε μια λίστα. Δοκίμασε:
\begin{lstlisting}
>>> sorted([4,2,3])
[2, 3, 4]
\end{lstlisting}
Έτσι το συνολικό πρόγραμμα γίνεται:
\begin{lstlisting}
lista = []
from itertools import permutations
for p in permutations([9,5,2]):
    ar = 34+p[0]/10+p[1]/100+p[2]/1000
    lista.append(round(ar,3))
print(sorted(lista))
\end{lstlisting}

Που δίνει το αποτέλεσμα:
\begin{lstlisting}
[34.259, 34.295, 34.529, 34.592, 34.925, 34.952]
\end{lstlisting}

Όμως η άσκηση μας ζητάει να τυπώσουμε τη λίστα με φθίνουσα σειρά. Αυτό μπορεί να γίνει δηλώνοντας στη sorted ότι θέλουμε αντίστροφη σειρά γράφοντας \lstinline{reverse=True}. Το τελικό πρόγραμμα είναι το εξής:
\begin{lstlisting}
lista = []
from itertools import permutations
for p in permutations([9,5,2]):
    ar = 34+p[0]/10+p[1]/100+p[2]/1000
    lista.append(round(ar,3))
print(sorted(lista,reverse=True))
\end{lstlisting}

Μια μικρή τροποποίηση που μπορεί να γίνει για να εμφανιστούν οι αριθμοί σε διαφορετικές γραμμές είναι να τυπώσουμε τη λίστα με μια for.
\begin{lstlisting}
lista = []
from itertools import permutations
for p in permutations([9,5,2]):
    ar = 34+p[0]/10+p[1]/100+p[2]/1000
    lista.append(round(ar,3))

for x in sorted(lista,reverse=True):
    print(x)
\end{lstlisting}

\begin{exercise}
\sel{61} Να υπολογίσεις τα αθροίσματα:

(α) $48,18 + 3,256 + 7,129$

(β) $3,59 + 7,13 + 8,195$
\end{exercise}

\begin{lstlisting}
>>> 48.18+3.256+7.129
58.565
>>> 3.59 + 7.13 + 8.195
18.915
\end{lstlisting}
\begin{exercise}
\sel{61}
Να υπολογίσεις το μήκος της περιμέτρου των οικοπέδων:
(Σχήμα ---)
\end{exercise}
\begin{lstlisting}
>>> 26.14 + 80.19 + 29.13+38.13+23.24+57.89+80.19
334.91
>>> 39.93+80.19+57.89+47.73+44.75+48.9+47.19
366.58
\end{lstlisting}
\begin{exercise}
\sel{61} Να κάνεις τις διαρέσεις:
(α) $579:48$

(β) $314:25$

(γ) $520:5,14$

(δ) $49,35:7$

\end{exercise}
\begin{lstlisting}
>>> 579/48
12.0625
>>> 314/25
12.56
>>> 520/5.14
101.16731517509729
>>> 49.35/7
7.05
\end{lstlisting}
\begin{exercise}
\sel{61}
Να κάνεις τις πράξεις: 

(α) $520 \cdot 0,1 + 0,32 \cdot 100 $

(β) $4,91 \cdot 0,01 + 0,819 \cdot 10$

\end{exercise}

\begin{lstlisting}
>>> 520*0.1 + 0.32*100
84.0
>>> 4.91*0.01 + 0.819*10
8.239099999999999
\end
\end{lstlisting}

Σε αυτή την άσκηση βλέπουμε ότι ο υπολογιστής προσεγγίζει τα αποτελέσματα με τον δικό του τρόπο.
Δοκίμασε:
\begin{lstlisting}
>>> x = 520*0.1 + 0.32*100
>>> x
84.0
>>> type(x)
<class 'float'>
>>> y = int(x)
>>> type(y)
<class 'int'>
>>> x == y
True
\end{lstlisting}
Αυτό σημαίνει πως ο ακέραιος αριθμός 84, και κάθε ακέραιος, στην Python μπορεί να αναπαρασταθεί σαν ακέραιος αλλά και σαν float με μηδενικά δεκαδικά ψηφία.
Στην δεύτερη πράξη παρατηρούμε ότι αντί για το σωστό αποτέλεσμα που είναι $0,0491+8,19=8,2391$ η Python εμφανίζει μια προσέγγιση που είναι $8.239099999999999$. Η διαφορά είναι πολύ μικρή. Ωστόσο οι δύο ποσότητες δεν είναι ίσες.
Δοκίμασε:
\begin{lstlisting}
>>> 4.91*0.01 + 0.819*10 == 8.2391
False
>>> 8.2391 - 4.91*0.01 + 0.819*10 
1.7763568394002505e-15
\end{lstlisting}
Ο αριθμός \lstinline{1.7763568394002505e-15} σημαίνει πως η διαφορά είναι περίπου $1.77\cot 10^{-15}$ που είναι πάρα πολύ μικρή και προκύπτει από τον τρόπο με τον οποίο η Python αποθηκεύει τους αριθμούς.

\begin{exercise}
\sel{61}
Να κάνεις τις πράξεις:

(α) $4,7:0,1-45:10$

(β) $0,98:0,0001 - 6785:1000$
\end{exercise}

\begin{lstlisting}
>>> 4.7/0.1 - 45/10
42.5
>>> 0.98/0.0001 - 6785/1000
9793.215
\end{lstlisting}
Βλέπουμε ότι η Python υπολογίζει σωστά πρώτα τη διαίρεση και μετά την αφαίρεση.

\begin{exercise}
\sel{61}
Η περίμετρος ενός τετραγώνου είναι 20,2. Να υπολογίσεις την πλευρά του.
\end{exercise}
\begin{lstlisting}
>>> 20.2/4
5.05
\end{lstlisting}

\begin{exercise}
\sel{61}
Η περίμετρος ενός ισοσκελούς τριγώνου είναι 48,52. Αν η βάση του είναι 10,7, πόσο είναι η κάθε μία από τις ίσες πλευρές του;
\end{exercise}
Αφαιρούμε πρώτα από το 48,52 το 10,7. Το αποτελέσμα το διαιρούμε με το δυο.
\begin{lstlisting}
>>> 48.52-10.7
37.82000000000001
>>> 37.82/2
18.91
\end{lstlisting}

\begin{exercise}
\sel{61}
Να υπολογίσεις τις τιμές των αριθμητικών παραστάσεων:

(α) $24\cdot 5 - 2 + 3 \cdot 5$

(β) $3\cdot 11 -2 + 45,1 : 2$
\end{exercise}
\begin{lstlisting}
>>> 24*5 - 2 +3*5
133
>>> 3*11 - 2 + 54.1/2
58.05
\end{lstlisting}

\begin{exercise}
\sel{61}
Να υπολογίσεις τις δυνάμεις:
(α) $3,1^2$, (β) $7,01^2$, (γ) $4,5^2$, (δ) $0,5^2$, (ε) $0,2^2$, (στ) $0,3^3$
\end{exercise}
\begin{lstlisting}
>>> 3.1**2
9.610000000000001
>>> 7.01**2
49.1401
>>> 4.5**2
20.25
>>> 0.5**2
0.25
>>> 0.2**2
0.04000000000000001
>>> 0.3*3
0.8999999999999999
\end{lstlisting}
Πάλι κάνουν την εμφάνισή τους μικρές προσεγγίσεις.

\begin{exercise}
Τοποθέτησε ένα ``x'' στην αντίστοιχη θέση (ΣΩΣΤΟ ΛΑΘΟΣ)
(α) $2,75 + 0,05 + 1,40 + 16,80 = 21$
(β) $420,510 + 72,490 + 45,19 + 11,81 = 500$
(γ) $4 – 3,852 = 1,148$
(δ) $32,01 – 4,001 = 28,01$
(ε) $41900 \cdot 0,0001 – 0,0419 \cdot 1000 = 0$
(στ) $56,89 \cdot 0,01 + 4311 : 10000 = 1$
(ζ) $(3,2 + 7,2 \cdot 2 + 24 \cdot 0,1) : 100 = 0,2$
\end{exercise}

(α)
\begin{lstlisting}
>>> 2.75 + 0.05 + 1.40 + 16.80 == 21
True
>>> 2.75 + 0.05 + 1.40 + 16.80
21.0
\end{lstlisting}
Άρα Σωστό

(β)
\begin{lstlisting}
>>> 420.510 + 72.490 + 45.19 + 11.81 == 500
False
>>> 420.510 + 72.490 + 45.19 + 11.81
550.0
\end{lstlisting}
Άρα Λάθος

(γ)
\begin{lstlisting}
>>> 4 - 3.852 == 1.148
False
>>> 4 - 3.852
0.14800000000000013
\end{lstlisting}
Άρα Λάθος

(δ)
\begin{lstlisting}
>>> 32.01 - 4.001 == 28.01
False
>>> 32.01 - 4.001
28.008999999999997
\end{lstlisting}
Άρα Λάθος

(ε)
\begin{lstlisting}
>>> 41900*0.0001 - 0.0419*1000 == 0
False
>>> 41900*0.0001 - 0.0419*1000
-37.71
\end{lstlisting}
Άρα Λάθος

(στ)
\begin{lstlisting}
>>> 56.89*0.01 + 4311 / 10000 == 1
True
>>> 56.89*0.01 + 4311 / 10000
1.0
\end{lstlisting}
Άρα Σωστό

και 

(ζ)
\begin{lstlisting}
>>> (3.2 + 7.2*2 + 24*0.1) / 100 == 0.2
True
>>> (3.2 + 7.2*2 + 24*0.1) / 100
0.2
\end{lstlisting}

Άρα Σωστό.
\chapter{Εξισώσεις και προβλήματα}

Σε αυτό το κεφάλαιο θα χρησιμοποιήσουμε τη βιβλιοθήκη sympy.
Υπάρχει ένα περιβάλλον στο οποίο μπορούμε να πληκτρολογούμε εντολές της βιβλιοθήκης ώστε να βλέπουμε τα αποτελέσματα με φιλικό τρόπο στον φυλλομετρήτή μας, συνήθως Chrome, Firefox ή Microsoft Edge. Το περιβάλλον αυτό βρίσκεται στη διεύθυνση https://live.sympy.org/. Μπορούμε να κάνουμε τα ίδια παραδείγματα στον Mu Editor όπως έχουμε συνηθίσει χρησιμοποιώντας την εντολή:
\begin{lstlisting}
from sympy import *
\end{lstlisting}
όμως τα αποτελέσματα δεν θα εμφανίζονται με φιλικό τρόπο αλλά με τον συμβολισμό της Python.
\section{Η έννοια της εξίσωσης}
\begin{exercise}
Γράψε συντομότερα τις εκφράσεις:

(α) $x + x + x + x$, 

(β) $\alpha + \alpha + \alpha + \beta + \beta$, 

(γ) $3\cdot \alpha + 5 \cdot \alpha$, 

(δ) $18 \cdot x + 7 \cdot x + 4 \cdot x$, 

(ε) $15 \cdot \beta – 9 \cdot \beta$.
\end{exercise}

Επειδή τα σύμβολα είναι τα $x, a, b$ θα πρέπει να τα δηλώσουμε στο sympy. Αυτό γίνεται ως εξής:
\begin{lstlisting}
from sympy import *
x,a,b = symbols("x a b")
\end{lstlisting}

Στη συνέχεια όποτε αναφέρουμε τα $x, a, b$ η Python θα καταλαβαίνει ότι πρόκειται για σύμβολα και θα δρα ανάλογα.
Έτσι αν δώσουμε στην Python
\begin{lstlisting}
>>> x + x + x + x
\end{lstlisting}
Θα μας δώσει ως απάντηση
$$4x$$
στο live.sympy.org
και
\begin{lstlisting}
4*x
\end{lstlisting}
στην απλή Python ή στο Mu Editor.
Άρα το 
\begin{lstlisting}
>>> a + a + a + b + b
\end{lstlisting}
Θα μας δώσει σαν απάντηση:
$$3a+2b$$
και τα 
\begin{lstlisting}
>>> 3*a + 5*a 
>>> 18*x + 7*x + 4*x
>>> 15*b - 9b
\end{lstlisting}
$$8a$$
$$29x$$
$$6b$$
αντίστοιχα.
\begin{exercise}
Να αντικαταστήσεις το x, με τους αριθμούς 1, 3, 4, 5, 6 και 11, σε κάθε ισότητα της πρώτης στήλης, του παρακάτω πίνακα. Βρες ποιος από αυτούς την επαληθεύει και ποιος όχι.

\begin{tabular}{|c|c|c|}
Εξίσωση            &Αριθμοί που την επαληθεύουν     &Αριθμοί που δεν την επαληθεύουν\\
$x – 4 = 1$        &                             &                                \\
$5 – x = 4$        &                             &                                \\
$2x = 8$           &                             &                                \\
$\frac{6}{x} = 2$  &                             &                                \\
$\frac{x}{2} = 3$  &                             &                                \\
$x + 7 = 30$       &                             &                                \\
\end{tabular}
\end{exercise}

\begin{lstlisting}
>>> e = x - 4
>>> e.subs(x,1)
\end{lstlisting}
$$−3$$
\begin{lstlisting}
>>> e.subs(x,3)
\end{lstlisting}
$$−1$$
\begin{lstlisting}
>>> e.subs(x,4)
\end{lstlisting}
$$0$$
\begin{lstlisting}
>>> e.subs(x,5)
\end{lstlisting}
$$1$$
\begin{lstlisting}
>>> e.subs(x,6)
\end{lstlisting}
$$2$$
\begin{lstlisting}
>>> e.subs(x,11)
\end{lstlisting}
$$7$$

Οπότε ο αριθμός που την επαληθεύει είναι ο 5 και όλοι οι υπόλοιποι δεν την επαληθεύουν.

\begin{lstlisting}
>>> e = 5 - x
>>> e.subs(x,1)
\end{lstlisting}
$$4$$
\begin{lstlisting}
>>> e.subs(x,3)
\end{lstlisting}
$$2$$
\begin{lstlisting}
>>> e.subs(x,4)
\end{lstlisting}
$$1$$
\begin{lstlisting}
>>> e.subs(x,5)
\end{lstlisting}
$$0$$
\begin{lstlisting}
>>> e.subs(x,6)
\end{lstlisting}
$$-1$$
\begin{lstlisting}
>>> e.subs(x,11)
\end{lstlisting}
$$-6$$

Οπότε ο αριθμός που την επαληθεύει είναι ο 1 και όλοι οι υπόλοιποι δεν την επαληθεύουν.

\begin{lstlisting}
>>> e = 2*x
>>> e.subs(x,1)
\end{lstlisting}
$$2$$
\begin{lstlisting}
>>> e.subs(x,3)
\end{lstlisting}
$$6$$
\begin{lstlisting}
>>> e.subs(x,4)
\end{lstlisting}
$$8$$
\begin{lstlisting}
>>> e.subs(x,5)
\end{lstlisting}
$$10$$
\begin{lstlisting}
>>> e.subs(x,6)
\end{lstlisting}
$$12$$
\begin{lstlisting}
>>> e.subs(x,11)
\end{lstlisting}
$$22$$

Οπότε ο αριθμός που την επαληθεύει είναι ο 4 και όλοι οι υπόλοιποι δεν την επαληθεύουν.

\begin{lstlisting}
>>> e = 6/x
>>> e.subs(x,1)
\end{lstlisting}
$$6$$
\begin{lstlisting}
>>> e.subs(x,3)
\end{lstlisting}
$$2$$
\begin{lstlisting}
>>> e.subs(x,4)
\end{lstlisting}
$$\frac{3}{2}$$
\begin{lstlisting}
>>> e.subs(x,5)
\end{lstlisting}
$$\frac{6}{5}$$
\begin{lstlisting}
>>> e.subs(x,6)
\end{lstlisting}
$$1$$
\begin{lstlisting}
>>> e.subs(x,11)
\end{lstlisting}
$$\frac{6}{11}$$

Οπότε ο αριθμός 3 επαληθεύει την εξίσωση και όλοι οι υπόλοιποι δεν την επαληθεύουν:


\begin{lstlisting}
>>> e = x/2
>>> e.subs(x,1)
\end{lstlisting}
$$6$$
\begin{lstlisting}
>>> e.subs(x,3)
\end{lstlisting}
$$2$$
\begin{lstlisting}
>>> e.subs(x,4)
\end{lstlisting}
$$\frac{3}{2}$$
\begin{lstlisting}
>>> e.subs(x,5)
\end{lstlisting}
$$\frac{6}{5}$$
\begin{lstlisting}
>>> e.subs(x,6)
\end{lstlisting}
$$1$$
\begin{lstlisting}
>>> e.subs(x,11)
\end{lstlisting}
$$\frac{6}{11}$$

\begin{lstlisting}
>>> e = x/2
>>> e.subs(x,1)
\end{lstlisting}
$$\frac{1}{2}$$

\begin{lstlisting}
>>> e.subs(x,3)
\end{lstlisting}
$$\frac{3}{2}$$

\begin{lstlisting}
>>> e.subs(x,4)
\end{lstlisting}
$$2$$

\begin{lstlisting}
>>> e.subs(x,5)
\end{lstlisting}
$$\frac{5}{2}$$

\begin{lstlisting}
>>> e.subs(x,6)
\end{lstlisting}
$$3$$
\begin{lstlisting}
>>> e.subs(x,11)
\end{lstlisting}
$$\frac{11}{2}$$

Ο αριθμός που επαληθεύει την εξίσωση είναι ο 6, οι υπόλοιποι αριθμοί δεν την επαληθεύουν.

\begin{lstlisting}
>>> e = x + 7
>>> e.subs(x,1)
\end{lstlisting}
$$8$$

\begin{lstlisting}
>>> e.subs(x,3)
\end{lstlisting}
$$10$$

\begin{lstlisting}
>>> e.subs(x,4)
\end{lstlisting}
$$11$$

\begin{lstlisting}
>>> e.subs(x,5)
\end{lstlisting}
$$12$$

\begin{lstlisting}
>>> e.subs(x,6)
\end{lstlisting}
$$13$$
\begin{lstlisting}
>>> e.subs(x,11)
\end{lstlisting}
$$18$$

Κανένας από αυτούς τους αριθμούς δεν επαληθεύει την εξίσωση, οπότε:

\begin{tabular}{|c|c|c|}
Εξίσωση            &Αριθμοί που την επαληθεύουν  &Αριθμοί που δεν την επαληθεύουν\\
$x - 4 = 1$        &            5                &          1, 3, 4, 6 και 11    \\
$5 - x = 4$        &            1                &          3, 4, 5, 6 και 11    \\
$2x = 8$           &            4                &          1, 3, 5, 6 και 11    \\
$\frac{6}{x} = 2$  &            3                &          1, 4, 5, 6 και 11    \\
$\frac{x}{2} = 3$  &            6                &          1, 3, 4, 5 και 11    \\
$x + 7 = 30$       &                             &          1, 3, 4, 5, 6 και 11 \\
\end{tabular}

Ένας καλύτερος τρόπος για να έχουμε το ίδιο αποτέλεσμα είναι να γραφτεί ένα πρόγραμμα που να υπολογίζει τα αποτελέσματα για όλους τους αριθμούς και να συγκρίνει το αποτέλεσμα με το αναμενόμενο. Η enumerate μετράει τη λίστα και δημιουγεί έναν μετρητή με όνομα i που μπορούμε να τον χρησιμοποιήσουμε για να μετρήσουμε τα αναμενόμενα αποτελέσματα:
\begin{lstlisting}
for e in exprs: 
    for (i,xi) in enumerate([1,3,4,5,6,11]):
        print(e,xi,e.subs(x,xi),res[i])
        print(e.subs(x,xi)==res[i])
\end{lstlisting}
\begin{exercise}
\sel{73}
Να λυθούν οι εξισώσεις:
$$x+5=12$$
$$y-2=3$$
$$10-z =1$$
$$7\cdot phi = 14$$
$$w:5 = 4$$
$$24:\psi = 6$$
\end{exercise}
Η βιβλιοθήκη sympy έχει συνάρτηση solve για να λύνει εξισώσεις όταν το δεξί μέρος της εξίσωσης είναι 0 οπότε οι εξισώσεις πρέπει να μετατραπούν με το χέρι σε:
$$x+5-12 = 0$$
$$y-2-3 = 0$$
$$10-z -1 = 0$$
$$7\cdot phi - 14 = 0$$
$$w:5 - 4 = 0$$
$$24:\psi - 6 = 0$$

\begin{lstlisting}
>>> from sympy import *
>>> x,y,z,f,w,psi = symbols('x y z f w psi')
>>> solve(x+5-12)
[7]
>>> solve(y-2-3)
[5]
>>> solve(10-z -1)
[9]
>>> solve(7* f - 14)
[2]
>>> solve(w/5 - 4)
[20]
>>> solve(24/psi - 6)
[4]
\end{lstlisting}
Η συνάρτηση solve επιστρέφει μια λίστα με τις τιμές που επαληθεύουν την εξίσωση. Επειδή υπάρχει μόνο μία τιμή που επαληθεύει την εξίσωση για αυτό το λόγο υπάρχει μόνο μία τιμή στην κάθε λίστα.

\begin{exercise}
\sel{63}
Μια δεξαμενή χωρητικότητας 6m$^3$ που έχει μήκος 1,5m και πλάτος 2m, έχει
ύψος (α) 1,5m ή (β) 3m ή (γ) 2m;
\end{exercise}

\begin{lstlisting}
>>> solve(2*1.5*x - 6)
[2.0]
\end{lstlisting}
\begin{exercise}
\sel[4]{74}Γράψε με απλούστερο τρόπο τις μαθηματικές εκφράσεις: 

(α) $x+x$,

(β) $\alpha+\alpha+\alpha$,

(γ) $3\cdot \alpha+52\cdot \alpha$, 

(δ) $2\cdot \beta+\beta+3\cdot \alpha+2\cdot \alpha$, 

(ε) $4\cdot x+8\cdot x–3\cdot x$, 

(στ) $7\cdot \omega+4\cdot \omega–10\cdot \omega$

\end{exercise}

\begin{lstlisting}
>>> x+x
\end{lstlisting}

$$2x$$

\begin{lstlisting}
>>> a = symbols('a')
>>> a+a+a
\end{lstlisting}

$$3a$$

\begin{lstlisting}
>>>  3*a  + 52 * a
\end{lstlisting}

$$55a$$

\begin{lstlisting}
>>> a,b = symbols('a b')
>>> 2*b+b+3*a+2*a 
\end{lstlisting}

$$5a+3b$$

\begin{lstlisting}
>>> 4*x+8*x–3*x 
\end{lstlisting}

$$9x$$

\begin{lstlisting}
>>> w = symbols('w')
>>> 7*w+4*w–10*w
\end{lstlisting}

$$w$$

\begin{exercise}
\sel[6]{74}
Στην εξίσωση 2 + α = x, το α και το x είναι φυσικοί αριθμοί. Ποια από τις τιμές
0, 3, 1 μπορεί να πάρει το x ;
\end{exercise}
Θα λύσουμε την $$2+a-x=0$$ για αυτές τις τιμές:
\begin{lstlisting}
>>> solve(2+a-0)
[-2]
>>> solve(2+a-3)
[1]
>>> solve(2+a-1)
[-1]
\end{lstlisting}
Από αυτές τις λύσεις συμπεραίνουμε ότι μόνο η $2+a-3$ μπορεί να ισχύει για φυσικό αριθμό και άρα μόνο την τιμή $3$ μπορεί να πάρει το $x$.
\begin{exercise}
\sel[7]{74}
Να εξετάσεις, αν ο αριθμός 12 είναι η λύση της εξίσωσης: x + 13 = 25
\end{exercise}
\begin{lstlisting}
>>> e = x + 13
>>> e.subs(e,x,12)
\end{lstlisting}
$$25$$

\end{document}