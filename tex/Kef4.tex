\chapter{Δεκαδικοί αριθμοί}
\section{Εισαγωγή}
Αν χωρίσουμε τη μονάδα σε 10 ίσα μέρη τότε μπορούμε να πάρουμε κλάσματα της μονάδας όπως $\frac{3}{10}$, $\frac{5}{10}$ κλπ. Τα κλάσματα είναι ομώνυμα συγκρίνονται εύκολα και βοηθάνε στις πράξεις. 
Γενικότερα, ονομάζουμε δεκαδικό κλάσμα οποιδήποτε κλάσμα έχει παρονομαστή μια δύναμη του 10. Κάθε δεκαδικό κλάσμα γράφεται σαν δεκαδικός αριθμός με τόσα δεκαδικά ψηφία όσα μηδενικά έχει ο παρονομαστής του.
Η Python χειρίζεται τους δεκαδικούς αριθμούς όπως και τους υπόλοιπους.
Δοκίμασε:
\begin{lstlisting}
>>> 0.3 + 0.5
0.8
>>> type(0.7)
<class 'float'>
\end{lstlisting}

Βλέπουμε ότι οι δεκαδικοί αριθμοί δεν είναι int, όπως οι ακέραιο αλλά float. Το όνομα float έχει να κάνει με τον τρόπο με τον οποίο ο υπολογιστής αποθηκεύει αποδοτικά αυτούς τους αριθμούς. 

Ας συνδυάσουμε τις γνώσεις από τα κλάσματα με τα κλάσματα που μάθαμε στο προηγούμενο κεφάλαιο.
\begin{lstlisting}
>>> from fractions import Fraction
>>> x = Fraction(3,10)
>>> float(x)
0.3
\end{lstlisting}

Το \lstinline{Fraction(3,10)} εννοεί το κλάσμα $\frac{3}{10}$ που είναι ίσο με 0,3. Όμως στην Python το 0,3 θα το γράφουμε με 0.3. Με τη συνάρτηση float μετατρέπουμε το $\frac{3}{10}$ σε δεκαδικό αριθμό.

\begin{exercise}
\sel{56} Γράψτε τους αριθμούς $\frac{3}{10}$, $\frac{825}{1000}$, $\frac{53}{1000}$, $\frac{1004}{10000}$.
\end{exercise}
\begin{lstlisting}
>>> float(Fraction(3,10))
0.625
>>> float(Fraction(825,100))
8.25
>>> float(Fraction(53,1000))
0.053
>>> float(Fraction(1004,10000))
0.1004
\end{lstlisting}

Η Python μπορεί να μετατρέψει τα κλάσματα σε δεκαδικό αριθμό ανεξάρτητα από τον παρονομαστή.
\begin{exercise}
\sel{59} Γράψε καθένα από τα παρακάτω κλάσματα, ως δεκαδικό αριθμό: (i) με προσέγγιση
εκατοστού και (ii) με προσέγγιση χιλιοστού: 

(α) $\frac{7}{16}$

(β) $\frac{21}{17}$

(γ) $\frac{20}{95}$
\end{exercise}
\begin{lstlisting}
>>> x = Fraction(7,16)
>>> float(x)
0.4375
>>> round(float(x),2)
0.44
>>> round(float(x),3)
0.438
>>> x = Fraction(21,17)
>>> float(x)
1.2352941176470589
>>> round(float(x),2)
1.24
>>> round(float(x),3)
1.235
>>> x = Fraction(20,95)
>>> float(x)
0.21052631578947367
>>> round(float(x),2)
0.21
>>> round(float(x),3)
0.211
\end{lstlisting}


Η στρογγυλοποίηση των δεκαδικών υλοποιείται στην Python με τη συνάρτηση round. Οπότε μπορείς να στρογγυλοποιήσεις εύκολα δεκαδικούς αριθμούς ως εξής:
\begin{exercise}
Να στρογγυλοποιήσεις τους παρακάτω δεκαδικούς αριθμούς στο δέκατο, εκατοστό και
χιλιοστό: 

(α) 9876,008, 

(β) 67,8956, 

(γ) 0,001, 

(δ) 8,239, 

(ε) 23,7048.
\end{exercise}
Θυμόμαστε να αλλάζουμε την υποδιαστολή από κόμμα σε τελεία:
\begin{lstlisting}
def roundall(x):
    print(round(x,1))
    print(round(x,2))
    print(round(x,3))

roundall(9876.008)
roundall(67.8956)
roundall(0.001)
roundall(8.239)
roundall(23.7048)
\end{lstlisting}

To αποτέλεσμα είναι:
\begin{lstlisting}
67.9
67.9
67.896
0.0
0.0
0.001
8.2
8.24
8.239
23.7
23.7
23.705
\end{lstlisting}

\begin{exercise}
\sel{59} Στον αριθμό $34,\square\square\square$ λείπουν τα τελευταία τρία ψηφία του. Να συμπληρώσεις τον
αριθμό με τα ψηφία 9, 5 και 2, έτσι ώστε κάθε ψηφίο να γράφεται μία μόνο φορά. Να γράψεις όλους τους δεκαδικούς που μπορείς να βρεις και να τους διατάξεις σε φθίνουσα σειρά.
\end{exercise}

Πώς μπορεί η Python να βρει όλους τους πιθανούς συνδυασμούς του 9,5,2;
Δοκίμασε τη βιβλιοθήκη itertools και συγκεκριμένα τη συνάρτηση permutations.
\begin{lstlisting}
>>> from itertools import permutations
>>> x = permutations([1,2,3])
>>> print(x)
<itertools.permutations object at 0x012BE1B0>
>>> print(list(x))
[(1, 2, 3), (1, 3, 2), (2, 1, 3), (2, 3, 1), (3, 1, 2), (3, 2, 1)]
\end{lstlisting}
Έτσι με την permutations μπορείς να βρεις όλες τις αναδιατάξεις των αριθμών. Οπότε τώρα το πρόγραμμα μπορεί να γίνει ως εξής:
\begin{lstlisting}
lista = []
from itertools import permutations
for p in permutations([9,5,2]):
    lista.append(34+p[0]/10+p[1]/100+p[2]/1000)
print(lista)
\end{lstlisting}
Που δίνει το αποτέλεσμα:
\begin{lstlisting}
[34.952, 34.925000000000004, 34.592000000000006, 34.529, 
34.29500000000001, 34.259]
\end{lstlisting}
Τα ψηφία που εμφανίζονται στο τέλος των αριθμών προκύπτουν από την αναπαράσταση των δεκαδικών στον υπολογιστή που υπόκειται σε κάποιους περιορισμούς.
Αν δεν θέλουμε να εμφανίζονται μπορούμε να αλλάξουμε το for σε:
\begin{lstlisting}
for p in permutations([9,5,2]):
    ar = 34+p[0]/10+p[1]/100+p[2]/1000
    lista.append(round(ar,3))
\end{lstlisting}
Τώρα για να γράψουμε τους αριθμούς με φθίνουσα σειρά θα δοκιμάσουμε τη sorted. Η sorted ταξινομεί τους αριθμούς που δίνονται σε μια λίστα. Δοκίμασε:
\begin{lstlisting}
>>> sorted([4,2,3])
[2, 3, 4]
\end{lstlisting}
Έτσι το συνολικό πρόγραμμα γίνεται:
\begin{lstlisting}
lista = []
from itertools import permutations
for p in permutations([9,5,2]):
    ar = 34+p[0]/10+p[1]/100+p[2]/1000
    lista.append(round(ar,3))
print(sorted(lista))
\end{lstlisting}

Που δίνει το αποτέλεσμα:
\begin{lstlisting}
[34.259, 34.295, 34.529, 34.592, 34.925, 34.952]
\end{lstlisting}

Όμως η άσκηση μας ζητάει να τυπώσουμε τη λίστα με φθίνουσα σειρά. Αυτό μπορεί να γίνει δηλώνοντας στη sorted ότι θέλουμε αντίστροφη σειρά γράφοντας \lstinline{reverse=True}. Το τελικό πρόγραμμα είναι το εξής:
\begin{lstlisting}
lista = []
from itertools import permutations
for p in permutations([9,5,2]):
    ar = 34+p[0]/10+p[1]/100+p[2]/1000
    lista.append(round(ar,3))
print(sorted(lista,reverse=True))
\end{lstlisting}

Μια μικρή τροποποίηση που μπορεί να γίνει για να εμφανιστούν οι αριθμοί σε διαφορετικές γραμμές είναι να τυπώσουμε τη λίστα με μια for.
\begin{lstlisting}
lista = []
from itertools import permutations
for p in permutations([9,5,2]):
    ar = 34+p[0]/10+p[1]/100+p[2]/1000
    lista.append(round(ar,3))

for x in sorted(lista,reverse=True):
    print(x)
\end{lstlisting}

\begin{exercise}
\sel{61} Να υπολογίσεις τα αθροίσματα:

(α) $48,18 + 3,256 + 7,129$

(β) $3,59 + 7,13 + 8,195$
\end{exercise}

\begin{lstlisting}
>>> 48.18+3.256+7.129
58.565
>>> 3.59 + 7.13 + 8.195
18.915
\end{lstlisting}
\begin{exercise}
\sel{61}
Να υπολογίσεις το μήκος της περιμέτρου των οικοπέδων:
(Σχήμα ---)
\end{exercise}
\begin{lstlisting}
>>> 26.14 + 80.19 + 29.13+38.13+23.24+57.89+80.19
334.91
>>> 39.93+80.19+57.89+47.73+44.75+48.9+47.19
366.58
\end{lstlisting}
\begin{exercise}
\sel{61} Να κάνεις τις διαρέσεις:
(α) $579:48$

(β) $314:25$

(γ) $520:5,14$

(δ) $49,35:7$

\end{exercise}
\begin{lstlisting}
>>> 579/48
12.0625
>>> 314/25
12.56
>>> 520/5.14
101.16731517509729
>>> 49.35/7
7.05
\end{lstlisting}
\begin{exercise}
\sel{61}
Να κάνεις τις πράξεις: 

(α) $520 \cdot 0,1 + 0,32 \cdot 100 $

(β) $4,91 \cdot 0,01 + 0,819 \cdot 10$

\end{exercise}

\begin{lstlisting}
>>> 520*0.1 + 0.32*100
84.0
>>> 4.91*0.01 + 0.819*10
8.239099999999999
\end
\end{lstlisting}

Σε αυτή την άσκηση βλέπουμε ότι ο υπολογιστής προσεγγίζει τα αποτελέσματα με τον δικό του τρόπο.
Δοκίμασε:
\begin{lstlisting}
>>> x = 520*0.1 + 0.32*100
>>> x
84.0
>>> type(x)
<class 'float'>
>>> y = int(x)
>>> type(y)
<class 'int'>
>>> x == y
True
\end{lstlisting}
Αυτό σημαίνει πως ο ακέραιος αριθμός 84, και κάθε ακέραιος, στην Python μπορεί να αναπαρασταθεί σαν ακέραιος αλλά και σαν float με μηδενικά δεκαδικά ψηφία.
Στην δεύτερη πράξη παρατηρούμε ότι αντί για το σωστό αποτέλεσμα που είναι $0,0491+8,19=8,2391$ η Python εμφανίζει μια προσέγγιση που είναι $8.239099999999999$. Η διαφορά είναι πολύ μικρή. Ωστόσο οι δύο ποσότητες δεν είναι ίσες.
Δοκίμασε:
\begin{lstlisting}
>>> 4.91*0.01 + 0.819*10 == 8.2391
False
>>> 8.2391 - 4.91*0.01 + 0.819*10 
1.7763568394002505e-15
\end{lstlisting}
Ο αριθμός \lstinline{1.7763568394002505e-15} σημαίνει πως η διαφορά είναι περίπου $1.77\cot 10^{-15}$ που είναι πάρα πολύ μικρή και προκύπτει από τον τρόπο με τον οποίο η Python αποθηκεύει τους αριθμούς.

\begin{exercise}
\sel{61}
Να κάνεις τις πράξεις:

(α) $4,7:0,1-45:10$

(β) $0,98:0,0001 - 6785:1000$
\end{exercise}

\begin{lstlisting}
>>> 4.7/0.1 - 45/10
42.5
>>> 0.98/0.0001 - 6785/1000
9793.215
\end{lstlisting}
Βλέπουμε ότι η Python υπολογίζει σωστά πρώτα τη διαίρεση και μετά την αφαίρεση.

\begin{exercise}
\sel{61}
Η περίμετρος ενός τετραγώνου είναι 20,2. Να υπολογίσεις την πλευρά του.
\end{exercise}
\begin{lstlisting}
>>> 20.2/4
5.05
\end{lstlisting}

\begin{exercise}
\sel{61}
Η περίμετρος ενός ισοσκελούς τριγώνου είναι 48,52. Αν η βάση του είναι 10,7, πόσο είναι η κάθε μία από τις ίσες πλευρές του;
\end{exercise}
Αφαιρούμε πρώτα από το 48,52 το 10,7. Το αποτελέσμα το διαιρούμε με το δυο.
\begin{lstlisting}
>>> 48.52-10.7
37.82000000000001
>>> 37.82/2
18.91
\end{lstlisting}

\begin{exercise}
\sel{61}
Να υπολογίσεις τις τιμές των αριθμητικών παραστάσεων:

(α) $24\cdot 5 - 2 + 3 \cdot 5$

(β) $3\cdot 11 -2 + 45,1 : 2$
\end{exercise}
\begin{lstlisting}
>>> 24*5 - 2 +3*5
133
>>> 3*11 - 2 + 54.1/2
58.05
\end{lstlisting}

\begin{exercise}
\sel{61}
Να υπολογίσεις τις δυνάμεις:
(α) $3,1^2$, (β) $7,01^2$, (γ) $4,5^2$, (δ) $0,5^2$, (ε) $0,2^2$, (στ) $0,3^3$
\end{exercise}
\begin{lstlisting}
>>> 3.1**2
9.610000000000001
>>> 7.01**2
49.1401
>>> 4.5**2
20.25
>>> 0.5**2
0.25
>>> 0.2**2
0.04000000000000001
>>> 0.3*3
0.8999999999999999
\end{lstlisting}
Πάλι κάνουν την εμφάνισή τους μικρές προσεγγίσεις.

\begin{exercise}
Τοποθέτησε ένα ``x'' στην αντίστοιχη θέση (ΣΩΣΤΟ ΛΑΘΟΣ)
(α) $2,75 + 0,05 + 1,40 + 16,80 = 21$
(β) $420,510 + 72,490 + 45,19 + 11,81 = 500$
(γ) $4 – 3,852 = 1,148$
(δ) $32,01 – 4,001 = 28,01$
(ε) $41900 \cdot 0,0001 – 0,0419 \cdot 1000 = 0$
(στ) $56,89 \cdot 0,01 + 4311 : 10000 = 1$
(ζ) $(3,2 + 7,2 \cdot 2 + 24 \cdot 0,1) : 100 = 0,2$
\end{exercise}

(α)
\begin{lstlisting}
>>> 2.75 + 0.05 + 1.40 + 16.80 == 21
True
>>> 2.75 + 0.05 + 1.40 + 16.80
21.0
\end{lstlisting}
Άρα Σωστό

(β)
\begin{lstlisting}
>>> 420.510 + 72.490 + 45.19 + 11.81 == 500
False
>>> 420.510 + 72.490 + 45.19 + 11.81
550.0
\end{lstlisting}
Άρα Λάθος

(γ)
\begin{lstlisting}
>>> 4 - 3.852 == 1.148
False
>>> 4 - 3.852
0.14800000000000013
\end{lstlisting}
Άρα Λάθος

(δ)
\begin{lstlisting}
>>> 32.01 - 4.001 == 28.01
False
>>> 32.01 - 4.001
28.008999999999997
\end{lstlisting}
Άρα Λάθος

(ε)
\begin{lstlisting}
>>> 41900*0.0001 - 0.0419*1000 == 0
False
>>> 41900*0.0001 - 0.0419*1000
-37.71
\end{lstlisting}
Άρα Λάθος

(στ)
\begin{lstlisting}
>>> 56.89*0.01 + 4311 / 10000 == 1
True
>>> 56.89*0.01 + 4311 / 10000
1.0
\end{lstlisting}
Άρα Σωστό

και 

(ζ)
\begin{lstlisting}
>>> (3.2 + 7.2*2 + 24*0.1) / 100 == 0.2
True
>>> (3.2 + 7.2*2 + 24*0.1) / 100
0.2
\end{lstlisting}

Άρα Σωστό.