\chapter{Δεκαδικοί αριθμοί}
\section{Εισαγωγή}
Αν χωρίσουμε τη μονάδα σε 10 ίσα μέρη τότε μπορούμε να πάρουμε κλάσματα της μονάδας όπως $\frac{3}{10}$, $\frac{5}{10}$ κλπ. Τα κλάσματα είναι ομώνυμα συγκρίνονται εύκολα και βοηθάνε στις πράξεις. 
Γενικότερα, ονομάζουμε δεκαδικό κλάσμα οποιδήποτε κλάσμα έχει παρονομαστή μια δύναμη του 10. Κάθε δεκαδικό κλάσμα γράφεται σαν δεκαδικός αριθμός με τόσα δεκαδικά ψηφία όσα μηδενικά έχει ο παρονομαστής του.
Η Python χειρίζεται τους δεκαδικούς αριθμούς όπως και τους υπόλοιπους.
Δοκίμασε:
\begin{lstlisting}
>>> 0.3 + 0.5
0.8
>>> type(0.7)
<class 'float'>
\end{lstlisting}

Βλέπουμε ότι οι δεκαδικοί αριθμοί δεν είναι int, όπως οι ακέραιο αλλά float. Το όνομα float έχει να κάνει με τον τρόπο με τον οποίο ο υπολογιστής αποθηκεύει αποδοτικά αυτούς τους αριθμούς. 

Ας συνδυάσουμε τις γνώσεις από τα κλάσματα με τα κλάσματα που μάθαμε στο προηγούμενο κεφάλαιο.
\begin{lstlisting}
>>> from fractions import Fraction
>>> x = Fraction(3,10)
>>> float(x)
0.3
\end{lstlisting}

Το \lstinline{Fraction(3,10)} εννοεί το κλάσμα $\frac{3}{10}$ που είναι ίσο με 0,3. Όμως στην Python το 0,3 θα το γράφουμε με 0.3. Με τη συνάρτηση float μετατρέπουμε το $\frac{3}{10}$ σε δεκαδικό αριθμό.

\begin{exercise}
\sel{56} Γράψτε τους αριθμούς $\frac{3}{10}$, $\frac{825}{1000}$, $\frac{53}{1000}$, $\frac{1004}{10000}$.
\end{exercise}
\begin{lstlisting}
>>> float(Fraction(3,10))
0.625
>>> float(Fraction(825,100))
8.25
>>> float(Fraction(53,1000))
0.053
>>> float(Fraction(1004,10000))
0.1004
\end{lstlisting}

Η Python μπορεί να μετατρέψει τα κλάσματα σε δεκαδικό αριθμό ανεξάρτητα από τον παρονομαστή.
\begin{exercise}
\sel{59} Γράψε καθένα από τα παρακάτω κλάσματα, ως δεκαδικό αριθμό: (i) με προσέγγιση
εκατοστού και (ii) με προσέγγιση χιλιοστού: 

(α) $\frac{7}{16}$

(β) $\frac{21}{17}$

(γ) $\frac{20}{95}$
\end{exercise}
\begin{lstlisting}
>>> x = Fraction(7,16)
>>> float(x)
0.4375
>>> round(float(x),2)
0.44
>>> round(float(x),3)
0.438
>>> x = Fraction(21,17)
>>> float(x)
1.2352941176470589
>>> round(float(x),2)
1.24
>>> round(float(x),3)
1.235
>>> x = Fraction(20,95)
>>> float(x)
0.21052631578947367
>>> round(float(x),2)
0.21
>>> round(float(x),3)
0.211
\end{lstlisting}


Η στρογγυλοποίηση των δεκαδικών υλοποιείται στην Python με τη συνάρτηση round. Οπότε μπορείς να στρογγυλοποιήσεις εύκολα δεκαδικούς αριθμούς ως εξής:
\begin{exercise}
Να στρογγυλοποιήσεις τους παρακάτω δεκαδικούς αριθμούς στο δέκατο, εκατοστό και
χιλιοστό: 

(α) 9876,008, 

(β) 67,8956, 

(γ) 0,001, 

(δ) 8,239, 

(ε) 23,7048.
\end{exercise}
Θυμόμαστε να αλλάζουμε την υποδιαστολή από κόμμα σε τελεία:
\begin{lstlisting}
def roundall(x):
    print(round(x,1))
    print(round(x,2))
    print(round(x,3))

roundall(9876.008)
roundall(67.8956)
roundall(0.001)
roundall(8.239)
roundall(23.7048)
\end{lstlisting}

To αποτέλεσμα είναι:
\begin{lstlisting}
67.9
67.9
67.896
0.0
0.0
0.001
8.2
8.24
8.239
23.7
23.7
23.705
\end{lstlisting}
