\chapter{Ανάλογα ποσά - Αντιστρόφως ανάλογα ποσά}
\begin{exercise}
Να σχεδιάσεις ένα ορθοκανονικό σύστημα ημιαξόνων, με μονάδα το 1 cm και να
τοποθετήσεις τα σημεία Α(2,3), Β(3,2), Γ(4,5), Δ(5,5), Ε(1,4), Z(7,3), Η(7,2), Θ(6,2),
Ι(6,0), Κ(0,5). Τι παρατηρείς για τα σημεία Ι και Κ; Πού βρίσκονται αυτά; Μπορείς να
γενικεύσεις τις παρατηρήσεις σου για τα σημεία που έχουν τετμημένη ή τεταγμένη το
μηδέν;
\end{exercise}
\begin{lstlisting}
import matplotlib.pyplot as plt

plt.clf()
points = [(2,3), (3,2), (4,5), (5,5), (1,4), (7,3), (7,2), (6,2), (6,0), (0,5)]
pointName = ['Α','Β','Γ','Δ','Δ','Ε','Ζ','Η','Θ','Ι','Κ']
x = [p[0] for p in points]
y = [p[1] for p in points]
color=['m','g','r','b']
plt.grid()
plt.scatter(x,y, s=100 ,marker='o', c=color)
for (i,p) in enumerate(points):
    plt.annotate(pointName[i],(p[0],p[1]))

plt.show()
\end{lstlisting}
\begin{figure}
\includegraphics{graph1.png}
\end{figure}
\begin{exercise}
\sel[2]{89}
Σε ορθοκανονικό σύστημα ημιαξόνων να τοποθετήσεις τα σημεία Α(2,1), Β(1,2), Γ(2,3)
και Δ(3,2). Τι σχήμα είναι το ΑΒΓΔ; Αν τα ευθύγραμμα τμήματα ΑΓ και ΒΔ τέμνονται
στο σημείο Κ, ποιες είναι οι συντεταγμένες του Κ;
\end{exercise}
\begin{lstlisting}
import matplotlib.pyplot as plt

plt.clf()
points = [(2,1), (1,2), (2,3), (3,2)]
pointName = ['Α','Β','Γ','Δ']
x = [p[0] for p in points]
y = [p[1] for p in points]
color=['m','g','r','b']
plt.grid()
plt.scatter(x,y, s=100 ,marker='o', c=color)
for (i,p) in enumerate(points):
    plt.annotate(pointName[i],(p[0],p[1]))

x = [points[0][0],points[2][0]]
y = [points[0][1],points[2][1]]
plt.plot(x,y)
x = [points[1][0],points[3][0]]
y = [points[3][1],points[3][1]]
plt.plot(x,y)

plt.show()
\end{lstlisting}
\begin{figure}
\includegraphics{graph2.png}
\end{figure}

\begin{exercise}
\sel[3]{89}
Γράψε πέντε διατεταγμένα ζεύγη σημείων, των οποίων η τετμημένη τους είναι ίση με
την τεταγμένη τους. Μπορείς να τα
τοποθετήσεις, σε ένα ορθοκανονικό
σύστημα ημιαξόνων; Τι παρατηρείς;
\end{exercise}
\begin{lstlisting}
import matplotlib.pyplot as plt

plt.clf()
points = [(1,1), (2,2), (5,5), (10,10), (15,15)]
pointName = ['Α','Β','Γ','Δ','Ε']
x = [p[0] for p in points]
y = [p[1] for p in points]
color=['m','g','r','b']
plt.grid()
plt.scatter(x,y, s=100 ,marker='o', c=color)
for (i,p) in enumerate(points):
    plt.annotate(pointName[i],(p[0],p[1]))

plt.show()
\end{lstlisting}
\begin{figure}
\includegraphics{graph3.png}
\end{figure}
