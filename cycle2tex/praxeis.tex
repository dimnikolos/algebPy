\chapter{Πράξεις}

\section{Οι αριθμοί και η Python}

Οι φυσικοί αριθμοί είναι οι αριθμοί από 0, 1, 2, 3, 4, 5, 6, \ldots, 98, 99, 100, \ldots, 1999, 2000, 2001, \ldots

Η Python μπορεί να χειριστεί φυσικούς αριθμούς. Δοκιμάστε να γράψετε στο Shell του IDLE έναν φυσικό αριθμό, θα δείτε ότι η Python θα τον επαναλάβει. Π.χ. δείτε τον αριθμό εκατόν είκοσι τρια (123).
\begin{lstlisting}
>>> 123
123
\end{lstlisting}

Πρέπει να γράφεις τους αριθμούς λίγο διαφορετικά από ότι στο χαρτί. Στους αριθμούς δεν πρέπει να βάζεις τελείες στις χιλιάδες όπως στο χαρτί. Αν το κάνεις, μπορεί η Python να σου πει ότι έκανες κάποιο λάθος, όμως μπορεί ο υπολογιστής θα καταλάβει διαφορετικό αριθμό από αυτόν που εννοείς.
Δείτε το παρακάτω παράδειγμα στο IDLE.
\begin{lstlisting}
>>> 1.000.000
  File "<stdin>", line 1
    1.000.000
            ^
SyntaxError: invalid syntax
>>> 100.000
100.0
\end{lstlisting}
Σε αυτό το παράδειγμα, η Python δεν καταλαβαίνει καθόλου τον αριθμό 1.000.000 γραμμένο με τελείες ενώ μεταφράζει το 100.000 σε 100.0, που για την Python σημαίνει 100 (εκατό). Γι' αυτόν τον λόγο δεν βάζουμε καθόλου τελείες έτσι αν θέλεις να γράψεις τον αριθμό ένα εκατομμύριο θα γράψεις 1000000.
\begin{lstlisting}
>>> 1000000
1000000
\end{lstlisting}

\section{Πρόσθεση, αφαίρεση και πολλαπλασιασμός φυσικών αριθμών}
Μια γλώσσα προγραμματισμού μπορεί να εκτελέσει απλές πράξεις πολύ εύκολα. Στο βιβλίο των μαθηματικών  σου μπορείς να βρεις πολλές ασκήσεις με πράξεις. Μπορείς να τις λύσεις με την Python.
Το σύμβολο του πολλαπλασιασμού στην Python είναι το αστεράκι * (SHIFT+8) στο πληκτρολόγιο. Εναλλακτικά, μπορείς να το βρείς στο αριθμητικό πληκτρολόγιο. 

Για παράδειγμα αν θες να υπολογίσεις το $27 \cdot 10.000$ τότε θα γράψεις \lstinline{27*10000}:
\begin{lstlisting}
>>> 27*10000
270000
\end{lstlisting}

Αν θες να υπολογίσεις το $89\cdot 7 + 89\cdot 3$ τότε θα γράψεις \lstinline{89*7+89*3}:
\begin{lstlisting}
>>> 89*7+89*3
890
\end{lstlisting}

Η Python εκτελεί πρώτα τους πολλαπλασιασμούς και μετά τις προσθέσεις/αφαιρέσεις δίνοντας έτσι το σωστό αποτέλεσμα. Για παράδειγμα 89\*7 + 89\*3 = 623 + 267 = 890, που είναι το σωστό αποτέλεσμα.

Ο μαθηματικός τρόπος σκέψης είναι η πράξη να γίνει ως εξής $89(7+3) = 89\cdot 10 = 890$. Ο σκοπός των σημειώσεων αυτών δεν είναι να ξεχάσεις τον μαθηματικό τρόπο σκέψης αλλά να κατακτήσεις και τον υπολογιστικό τρόπο σκέψης.



\section{Λυμένες ασκήσεις βιβλίου}
%Kef2.tex
\begin{exercise}
\sel{16}
Να υπολογιστούν τα γινόμενα: 

(α) $35 \cdot 10$, 

(β) $421 \cdot 100$,

(γ) $5 \cdot 1.000$,

(δ) $27 \cdot 10.000$
\end{exercise}

Η python μπορεί να κάνει αυτές τις πράξεις ως εξής:
\begin{lstlisting}
>>> 35*10
350
>>> 421*100
42100
>>> 5*1000
5000
>>> 27*10000
270000
\end{lstlisting}



\begin{exercise}
\sel{16}
Να εκτελεστούν οι ακόλουθες πράξεις:

(α) $89\cdot 7 + 89\cdot 3$

(β) $23 \cdot 49 + 77 \cdot 49$

(γ) $76 \cdot 13 – 76 \cdot 3$

(δ) $284 \cdot 99$
\end{exercise}
\begin{lstlisting}
>>> 89*7+89*3
890
>>> 23*49+77*49
4900
>>> 76*13-76*3
760
>>> 284*99
28116
\end{lstlisting}

Στις παραπάνω περιπτώσεις η python εκτελεί πρώτα τους πολλαπλασιασμούς και μετά τις προσθέσεις/αφαιρέσεις δίνοντας έτσι το σωστό αποτέλεσμα. Για παράδειγμα 89\*7 + 89\*3 = 623 + 267 = 890, που είναι το σωστό αποτέλεσμα.

\begin{exercise}
\sel{18}
Υπολογίστε:

(α)  $157 + 33$ 

(β)  $122 + 25 + 78$

(γ)  $785 - 323$

(δ)  $7.321 - 4.595$

(ε)  $60 - (18 - 2)$

(στ) $52 - 11 -9$

(ζ)  $23 \cdot 10$

(η)  $97 \cdot 100$

(θ)  $879 \cdot 1.000$
\end{exercise}
Σε python τα παραπάνω υπολογίζονται ως εξής:
\begin{lstlisting}
>>> 157+33
190
>>> 122+25+78
225
>>> 785-323
462
>>> 7321-4595
2726
>>> 60-(18-2)
44
>>> 52-11-9
32
>>> 23*10
230
>>> 97*100
9700
>>> 879*1000
879000
\end{lstlisting}
Οι παρενθέσεις (SHIFT+9 και SHIFT+0) αλλάζουν τη σειρά των πράξεων. Οι πράξεις που είναι μέσα στην παρένθεση εκτελούνται πρώτες. Γι' αυτό το λόγο 60-(18-2)=60-16=44.

\begin{exercise}
\sel{18}
Σε ένα αρτοποιείο έφτιαξαν μία μέρα 120 κιλά άσπρο ψωμί, 135 κιλά χωριάτικο, 25 κιλά σικάλεως και 38 κιλά πολύσπορο. Πουλήθηκαν 107 κιλά άσπρο ψωμί, 112 κιλά χωριάτικο, 19 κιλά σικάλεως και 23 κιλά πολύσπορο. Πόσα κιλά ψωμί έμειναν απούλητα;
\end{exercise}
Με τις γνώσεις που έχουμε θα πρέπει να μετατρέψουμε το παραπάνω πρόβλημα σε μια αριθμητική παράσταση ώστε η python να μπορεί να την υπολογίσει, στη συγκεκριμένη περίπτωση η σωστή παράσταση είναι $$(120-107)+(135-112)+(25-19)+(38-23)$$
\begin{lstlisting}
>>> (120-107)+(135-112)+(25-19)+(38-23)
57
\end{lstlisting}
και η απάντηση είναι 57 κιλά ψωμί.

\section{Δυνάμεις φυσικών αριθμών}
Ο τελεστής της python για τις δυνάμεις είναι ο **  (δυο φορές το αστεράκι). Δηλαδή, αν θέλουμε να υπολογίσουμε το $10^2$ θα γράψουμε 10**2, με όμοιο τρόπο μπορούμε να υπολογίσουμε και τις υπόλοιπες δυνάμεις. Δοκίμασε τα παρακάτω στο IDLE.
\begin{lstlisting}
>>> 10**2
100
>>> 10**3
1000
>>> 10**4
10000
>>> 10**5
100000
>>> 10**6
1000000
\end{lstlisting}
Στη προτεραιότητα των πράξεων, οι δυνάμεις έχουν μεγλύτερη προτεραιότητα από τον πολλαπλασιασμό και την πρόσθεση. Οπότε όταν έχουμε και δυνάμεις σε μια παράσταση πρώτα γίνονται οι πράξεις στις παρενθέσεις, μετά οι δυνάμεις και μετά οι πολλαπλασιασμοί και οι διαιρέσεις και τέλος οι προσθέσεις και οι αφαιρέσεις. Την ίδια σειρά ακολουθεί και η Python για τον υπολογισμό των πράξεων.
\begin{exercise}
\sel{21}
Να εκτελεστούν οι πράξεις 

 1. $(2\cdot 5)^4+4\cdot (3+2)^2$

 2. $(2+3)^3 - 8\cdot 3^2$

\end{exercise}
Οι αντίστοιχες εκφράσεις είναι (2*5)**4+4*(3+2)**2 και (2+3)**3 - 8*3**2.

\begin{lstlisting}
>>> (2*5)**4+4*(3+2)**2
10100
>>> (2+3)**3 - 8*3**2
53
\end{lstlisting}
H 8*3**2 υπολογίζεται ως $8\cdot (3^2)$, δηλαδή $8\cdot 9 = 72$, αφού πρώτα γίνεται η δύναμη και μετά οι πολλαπλασιασμοί.

\begin{exercise}
Κάνε τις πράξεις: 
(α) $3\cdot 5^2$, 

(β) $3\cdot 5^2 + 2$, 

(γ) $3\cdot5^2 + 2^2$, 

(δ) $3\cdot 5 + 2^2$, 

(ε) $3\cdot(5 + 2)^2$.
\end{exercise}

Αυτές οι πράξεις μπορούν να γίνουν στο IDLE.
\begin{lstlisting}
>>> 3*5**2
75
>>> 3*5**2 + 2
77
>>> 3*5**2 + 2**2
79
>>> 3*5 +2**2
19
>>> 3*(5 + 2)**2
147
\end{lstlisting}

\begin{exercise}
Κάνε τις πράξεις: 
(α) $3^2 +3^3 +2^3 +2^4$, 

(β) $(13-2)^ 4 + 5\cdot 3^2$
\end{exercise}

\begin{lstlisting}
>>> 3**2 +3**3 +2**3 +2**4
60
>>> (13-2)**4 + 5*3**2
14686
\end{lstlisting}

\begin{exercise}
Βρες τις τιμές των παραστάσεων: 

(α) $(6+5)^2$ και $6^2+5^2$, 

(β) $(3+6)^2$ και $3^2+6^2$.
\end{exercise}
\begin{lstlisting}
>>> (6+5)**2
121
>>> 6**2+5**2
61
>>> (3+6)**2
81
>>> 3**2+6**2
45
\end{lstlisting}