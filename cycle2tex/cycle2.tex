\documentclass[b5paper,11pt,twoside,openleft]{memoir}
\usepackage[left=1in,right=1.15in,top=1.25in,bottom=1.15in]{geometry}
\usepackage{xltxtra}
\usepackage{unicode-math}
\begin{document}
\setmainfont[Mapping=tex-text,Ligatures=Common]{Arno Pro}
\setsansfont[Mapping=tex-text,Scale=MatchLowercase]{Myriad Pro}
\setmonofont[Mapping=tex-text,Scale=MatchLowercase]{Consolas}
\setmathfont{Asana-Math}

\title{%
%$A\lambda\gamma\epsilon\beta\rho y$ \\ 
Μαθηματικά Γυμνασίου με Python}

\author{}
\maketitle
\thispagestyle{empty}
\chapter*	{Εισαγωγή}
Το βιβλίο αυτό είναι ένας συνδυασμός των μαθηματικών που έμαθες στην Α΄ Γυμνασίου με τη γλώσσα προγραμματισμού Python. Θα θυμηθείς όσα έμαθες στην Α΄ Γυμνασίου και θα μάθεις και τα βασικά μιας σύγχρονης γλώσσας προγραμματισμού που χρησιμοποιείται από πολλούς προγραμματιστές σε όλον τον κόσμο.

Για να εγκαταστήσεις την Python στον υπολογιστή σου πήγαινε στη σελίδα https://www.python.org/ και κατέβασε την τελευταία έκδοση της Python 3 (Latest). Αφού κάνεις εγκατάσταση θα βρεις στον υπολογιστή σου το πρόγραμμα IDLE με το οποίο μπορείς να δουλέψεις αυτές τις σημειώσεις.


%\chapter{Τι θα χρησιμοποιήσουμε;}
\section{Η γλώσσα προγραμματισμού Python}
Σε αυτές τις σημειώσεις θα χρησιμοποιήσουμε τη γλώσσα προγραμματισμού Python και μάλιστα την έκδοση 3. Υπάρχει και Python 2 αλλά υπάρχουν σχέδια για την αντικατάστασή της από την Python 3. Για να εγκαταστήσεις την Python 3 θα πρέπει να την κατεβάσεις από το επίσημο site της Python \href{https://www.python.org/}{www.python.org}. Κατεβάστε την πιο πρόσφατη έκδοση που σας προτείνει θα είναι κάτι σαν 3.8.2 ή κάτι 

\section{Ο επεξεργαστής προγραμμάτων Mu}
Μπορείς να γράψεις Python σε οποιοδήποτε πρόγραμμα υποστηρίζει απλό κείμενο, ακόμη και στο Σημειωματάριο, όμως σε αυτές τις σημειώσεις χρησιμοποιούμε τον επεξεργαστή Python, Mu Editor ή πιο απλά Mu που μπορείς να τον κατεβάσεις από τη σελίδα \href{https://codewith.mu/}{codewith.mu}. Μόλις το ανοίξεις θα δεις την εικόνα \ref{Mu}. 
\begin{figure}
\includegraphics[width=\textwidth]{mu.png}
\label{Mu}
\caption{Mu: Ένας επεξεργαστής προγραμμάτων Python}
\end{figure}

Μπορείς να πατήσεις την εκτέλεση (κουμπί Run) και τότε θα δεις ότι το παράθυρο χωρίζεται σε δύο τμήματα (Εικόνα \ref{Mu2}).  Αν θες να δοκιμάσεις ένα ολόκληρο πρόγραμμα μπορείς να το πληκτρολογήσεις στο βασικό παράθυρο (τώρα γράφει `\#Write your code here`). Ενώ αν θες να δοκιμάσεις κάποια εντολή τότε μπορείς να την πληκτρολογήσεις στο κάτω παράθυρο (τώρα γράφει $>>>$).  Το κάτω παράθυρο ονομάζεται REPL, από τα αρχικά των λέξεων Read, Eval, Print, Loop δηλαδή Διάβασε, Εκτέλεσε (την εντολή/έκφραση), Τύπωσε, Επανάλαβε. Το REPL θα διαβάσει την εντολή, θα την εκτελέσει και θα μας δώσει το αποτέλεσμα.

Από εδώ και πέρα όταν βλέπετε στις σημειώσεις τα τρία σύμβολα ``μεγαλύτερο από'' ($>>>$) θα πληκτρολογείτε τις αντίστοιχες εντολές στο κάτω παράθυρο (REPL). Τα μεγαλύτερα προγράμματα που δεν θα έχουν αυτό το σύμβολο θα τα πληκτρολογείτε στο πάνω παράθυρο.

\fbox{
	\parbox{0.8\textwidth}{%
	\textbf{Συμβουλή:} Αν χρησιμοποιείτε την ηλεκτρονική έκδοση αυτών των σημειώσεων, θυμηθείτε να πληκτρολογείτε τις εντολές και να μην τις κάνετε αντιγραφή επικόλληση.
	}
}


Στην αρχή θα δοκιμάσεις κάποια πράγματα στο κάτω παράθυρο, όμως μην ανησυχείς σύντομα θα γράφεις τα δικά σου προγράμματα στο πάνω παράθυρο.

\begin{figure}
\includegraphics[width=\textwidth]{mu2.png}
\label{Mu2}
\caption{Το πρόγραμμα Mu όταν εκτελείτε ένας κώδικας}
\end{figure}

\section{Το βιβλίο μαθηματικών της Α΄ Γυμνασίου}
Σε αυτές τις σημειώσεις οι περισσότερες ασκήσεις είναι από το βιβλίο Μαθηματικών της Α΄ Γυμνασίου των Βανδουλάκη, Καλλιγά, Μαρκάκη και Φερεντίνου (Εικόνα \ref{matha}).

\begin{figure}
\centering
\includegraphics[width=0.8\textwidth]{matha.jpg}
\label{matha}
\caption{Το εξώφυλλο του βιβλίου των Μαθηματικών που θα χρησιμοποιήσουμε}
\end{figure}
%\chapter{Φυσικοί αριθμοί}

\section{Οι αριθμοί και η Python}

Οι φυσικοί αριθμοί είναι οι αριθμοί από 0, 1, 2, 3, 4, 5, 6, \ldots, 98, 99, 100, \ldots, 1999, 2000, 2001, \ldots

Η Python μπορεί να χειριστεί φυσικούς αριθμούς. Δοκιμάστε να γράψετε στο REPL έναν φυσικό αριθμό, θα δείτε ότι η Python θα τον επαναλάβει. Π.χ. δείτε τον αριθμό εκατόν είκοσι τρια (123).
\begin{lstlisting}
>>> 123
123
\end{lstlisting}

Στην Python όμως θα πρέπει να ακολουθείς κάποιους επιπλέον κανόνες. Για παράδειγμα στους αριθμούς δεν πρέπει να βάζεις τελείες στις χιλιάδες όπως στο χαρτί. Αν το κάνεις στην καλύτερη περίπτωση θα προκύψει κάποιο λάθος, στην χειρότερη ο υπολογιστής θα καταλάβει διαφορετικό αριθμό από αυτόν που εννοείς.
Δείτε το παρακάτω παράδειγμα στο REPL.
\begin{lstlisting}
>>> 1.000.000
  File "<stdin>", line 1
    1.000.000
            ^
SyntaxError: invalid syntax
>>> 100.000
100.0
\end{lstlisting}
Σε αυτό το παράδειγμα, η Python δεν καταλαβαίνει καθόλου τον αριθμό 1.000.000 γραμμένο με τελείες ενώ μεταφράζει το 100.000 σε 100.0, που για την Python σημαίνει 100 (εκατό). Γι' αυτόν τον λόγο δεν βάζουμε καθόλου τελείες έτσι αν θέλουμε να γράψουμε το ένα εκατομμύριο θα γράψουμε 1000000.
\begin{lstlisting}
>>> 1000000
1000000
\end{lstlisting}

\section{Πρόσθεση, αφαίρεση και πολλαπλασιασμός φυσικών αριθμών}
Μια γλώσσα προγραμματισμού μπορεί να εκτελέσει απλές πράξεις πολύ εύκολα. Στο βιβλίο των μαθηματικών  σου μπορείς να βρεις πολλές ασκήσεις με πράξεις. Μπορείς να τις λύσεις με την Python.

\begin{exercise}
\sel{16}
Να υπολογιστούν τα γινόμενα: 

(α) $35 \cdot 10$, 

(β) $421 \cdot 100$,

(γ) $5 \cdot 1.000$,

(δ) $27 \cdot 10.000$
\end{exercise}

Η python μπορεί να κάνει αυτές τις πράξεις ως εξής:
\begin{lstlisting}
>>> 35*10
350
>>> 421*100
42100
>>> 5*1000
5000
>>> 27*10000
270000
\end{lstlisting}

Ο τελεστής του πολλαπλασιασμού είναι το αστεράκι * (SHIFT+8) στο πληκτρολόγιο. Εναλλακτικά, μπορείτε να το βρείτε στο αριθμητικό πληκτρολόγιο. 

\begin{exercise}
\sel{16}
Να εκτελεστούν οι ακόλουθες πράξεις:

(α) $89\cdot 7 + 89\cdot 3$

(β) $23 \cdot 49 + 77 \cdot 49$

(γ) $76 \cdot 13 – 76 \cdot 3$

(δ) $284 \cdot 99$
\end{exercise}
\begin{lstlisting}
>>> 89*7+89*3
890
>>> 23*49+77*49
4900
>>> 76*13-76*3
760
>>> 284*99
28116
\end{lstlisting}

Στις παραπάνω περιπτώσεις η python εκτελεί πρώτα τους πολλαπλασιασμούς και μετά τις προσθέσεις/αφαιρέσεις δίνοντας έτσι το αποτέλεσμα που αναμένεται. Για παράδειγμα 89\*7 + 89\*3 = 623 + 267 = 890, που είναι το σωστό αποτέλεσμα.

\begin{exercise}
\sel{18}
Υπολογίστε:

(α)  $157 + 33$ 

(β)  $122 + 25 + 78$

(γ)  $785 - 323$

(δ)  $7.321 - 4.595$

(ε)  $60 - (18 - 2)$

(στ) $52 - 11 -9$

(ζ)  $23 \cdot 10$

(η)  $97 \cdot 100$

(θ)  $879 \cdot 1.000$
\end{exercise}
Σε python τα παραπάνω υπολογίζονται ως εξής:
\begin{lstlisting}
>>> 157+33
190
>>> 122+25+78
225
>>> 785-323
462
>>> 7321-4595
2726
>>> 60-(18-2)
44
>>> 52-11-9
32
>>> 23*10
230
>>> 97*100
9700
>>> 879*1000
879000
\end{lstlisting}
Οι παρενθέσεις (SHIFT+9 και SHIFT+0) αλλάζουν τη σειρά των πράξεων. Οι πράξεις που είναι μέσα στην παρένθεση εκτελούνται πρώτες. Γι' αυτό το λόγο 60-(18-2)=60-16=44.

\begin{exercise}
\sel{18}
Σε ένα αρτοποιείο έφτιαξαν μία μέρα 120 κιλά άσπρο ψωμί, 135 κιλά χωριάτικο, 25 κιλά σικάλεως και 38 κιλά πολύσπορο. Πουλήθηκαν 107 κιλά άσπρο ψωμί, 112 κιλά χωριάτικο, 19 κιλά σικάλεως και 23 κιλά πολύσπορο. Πόσα κιλά ψωμί έμειναν απούλητα;
\end{exercise}
Με τις γνώσεις που έχουμε θα πρέπει να μετατρέψουμε το παραπάνω πρόβλημα σε μια αριθμητική παράσταση ώστε η python να μπορεί να την υπολογίσει, στη συγκεκριμένη περίπτωση η σωστή παράσταση είναι $$(120-107)+(135-112)+(25-19)+(38-23)$$
\begin{lstlisting}
>>> (120-107)+(135-112)+(25-19)+(38-23)
57
\end{lstlisting}
και η απάντηση είναι 57 κιλά ψωμί.

\section{Δυνάμεις φυσικών αριθμών}
Ο τελεστής της python για τις δυνάμεις είναι ο **  (δυο φορές το αστεράκι). Δηλαδή, αν θέλουμε να υπολογίσουμε το $10^2$ θα γράψουμε 10**2, με όμοιο τρόπο μπορούμε να υπολογίσουμε και τις υπόλοιπες δυνάμεις. Δοκίμασε τα παρακάτω στο REPL.
\begin{lstlisting}
>>> 10**2
100
>>> 10**3
1000
>>> 10**4
10000
>>> 10**5
100000
>>> 10**6
1000000
\end{lstlisting}
Στη προτεραιότητα των πράξεων, οι δυνάμεις έχουν μεγλύτερη προτεραιότητα από τον πολλαπλασιασμό και την πρόσθεση. Οπότε όταν έχουμε και δυνάμεις σε μια παράσταση πρώτα γίνονται οι πράξεις στις παρενθέσεις, μετά οι δυνάμεις και μετά οι πολλαπλασιασμοί και οι προσθέσεις.
\begin{exercise}
\sel{21}
Να εκτελεστούν οι πράξεις 

 1. $(2\cdot 5)^4+4\cdot (3+2)^2$

 2. $(2+3)^3 - 8\cdot 3^2$

\end{exercise}
Οι αντίστοιχες εκφράσεις είναι (2*5)**4+4*(3+2)**2 και (2+3)**3 - 8*3**2.

\begin{lstlisting}
>>> (2*5)**4+4*(3+2)**2
10100
>>> (2+3)**3 - 8*3**2
53
\end{lstlisting}
H 8*3**2 υπολογίζεται ως $8\cdot (3^2)$, δηλαδή $8\cdot 9 = 72$, αφού πρώτα γίνεται η δύναμη και μετά οι πολλαπλασιασμοί.

\section{Συγκρίσεις φυσικών αριθμών}
Μπορούμε να συγκρίνουμε αριθμούς στην Python χρησιμοποιώντας τους τελεστές == (πληκτρολογούμε δύο φορές το =) για την \emph{ισότητα}, > για το \emph{μεγαλύτερο} και < για το \emph{μικρότερο}. Επίσης μπορούμε να χρησιμοποιήσουμε >= για το \emph{μεγαλύτερο ή ίσο} και <= για το \emph{μικρότερο ή ίσο}, τέλος υπάρχει το != για το \emph{δεν είναι ίσο}. Μπορείς να δοκιμάσεις τα παρακάτω:
\begin{lstlisting}
>>> 123==123
True
>>> 123>123
False
>>> 123>122
True
>>> 123<123
False
>>> 123<124
True
>>> 123<=123
True
>>> 123<=124
True
>>> 123<=122
False
>>> 123>=123
True
>>> 123>=124
False
>>> 123>=122
True
>>> 122 != 123
True
>>> 122 != 122
False
\end{lstlisting}
Η Python επιστρέφει True (αληθές) όταν μία πρόταση ισχύει και False (ψευδές) όταν δεν ισχύει.

Σκέψου ότι για την Python η σύγκριση είναι και αυτή μια πράξη. Αντί η πράξη αυτή να δίνει σαν αποτέλεσμα έναν αριθμό δίνει σαν αποτέλεσμα το αληθές ή το ψευδές.

Για παράδειγμα:
\begin{exercise}
Να συγκρίνετε τα $3^2$ και $2^3$.
\end{exercise}
Η σύγκριση αυτή μπορεί να γίνει στο REPL. Δοκίμασε:
\begin{lstlisting}
>>> 3**2 > 2**3
True
\end{lstlisting}
Άρα το $3^2$ είναι μεγαλύτερο από το $2^3$. Θυμήσου ότι το $3^2=9$, ενώ $2^3=8$.
\begin{exercise}
\end{exercise}

\section{Η εντολή print}
Ήρθε η ώρα να γράψεις εντολές στο πάνω παράθυρο, δηλαδή να γράψεις το πρώτο σου πρόγραμμα.  Με βάση όσα ξέρεις προσπάθησε να γράψεις μια πράξη στο πάνω παράθυρο, για παράδειγμα $32+35$. Ύστερα πάτησε το κουμπί της εκτέλεσης (Run). Μπορείς να δεις το αποτέλεσμα στην εικόνα \ref{noprint}.
\begin{figure}
\includegraphics[width=\textwidth]{noprint.png}
\caption{Η εκτέλεση δεν δίνει κάποιο αποτέλεσμα}
\end{figure}

Η Python εκτελεί την πράξη $32+35$, και υπολογίζει το αποτέλεσμα. Αν δεν το έκανε και υπήρχε κάποιο πρόβλημα θα εμφάνιζε κάποιο μήνυμα λάθους στο REPL. Το αποτέλεσμα όμως δεν εμφανίζεται. Για να εμφανιστεί το αποτέλεσμα πρέπει να χρησιμοποιήσεις την εντολή print (εκτύπωσε). Η εντολή print εκτελείτε ως εξής:
\begin{lstlisting}
print(32+35)
\end{lstlisting}
Γράφουμε δηλαδή, print ανοίγουμε παρένθεση, γράφουμε αυτό που θέλουμε να εκτυπωθεί και κλείνουμε την παρένθεση. Όταν εκτελέσουμε το πρόγραμμα με την print τότε εμφανίζεται το αποτέλεσμα στο REPL (εικόνα \ref{withprint}).
\begin{figure}
\includegraphics[width=\textwidth]{withprint.png}
\caption{Η εκτέλεση δίνει το αποτέλεσμα της πράξης}
\end{figure}
\emph{Μπράβο!} Μόλις έγραψες το πρώτο σου πρόγραμμα στην Python. Μάλιστα το πρόγραμμά σου κάνει κάτι. Υπολογίζει το αποτέλεσμα της πράξης $32+35$.

\section{Απαρίθμηση}
Είδαμε ότι η Python μπορεί να κάνει πολύ γρήγορα, πολύπλοκες πράξεις ακόμη και με δυνάμεις, αλλά δεν είδαμε ακόμη τις απλές ασκήσεις που υπάρχουν στις πρώτες σελίδες του βιβλίου. Όπως για παράδειγμα ποιοι είναι οι τρεις προηγούμενοι αριθμοί του 289 και ποιο οι δύο επόμενοι (\sel{13}).

Τώρα που μάθαμε να γράφουμε προγράμματα σε Python μπορούμε να αντιμετωπίσουμε αυτό το πρόβλημα με το παρακάτω πρόγραμμα:
\begin{lstlisting}
print(289-3)
print(289-2)
print(289-1)
print(289+1)
print(289+2)
\end{lstlisting}
που δίνει το αποτέλεσμα
\begin{lstlisting}
286
287
288
290
291
\end{lstlisting}

Πιο σωστό θα ήταν να γράψουμε ποιοι αριθμοί είναι οι προηγούμενοι και ποιοι οι επόμενοι. Σε αυτή την περίπτωση θα γράψουμε τις παρακάτω εντολές.
\begin{lstlisting}
print("Οι  προηγούμενοι αριθμοί είναι:")
print(289-3)
print(289-2)
print(289-1)
print("Οι επόμενοι αριθμοί είναι:")
print(289+1)
print(289+2)
\end{lstlisting}

Για να εμφανίσει η print τις λέξεις που θέλουμε πρέπει να τις βάλουμε μέσα σε εισαγωγικά. Η Python υποστηρίζει είτε μονά εισαγωγικά, είτε διπλά. Αυτά εισάγονται συνήθως με το ίδιο κουμπί του πληκτρολογίου (κοντά στο ENTER), είτε με SHIFT ή χωρίς. Θυμήσου να κλείνεις τα εισαγωγικά με τον ίδιο τρόπο που τα άνοιξες. Στο πρόγραμμα Mu τα εισαγωγικά αυτά δεν φαίνονται όπως σε άλλα πρόγραμματα σαν `Εισαγωγικά' ή ``Εισαγωγικά " ή <<Εισαγωγικά>>, αλλά φαίνονται κάπως πιο απλά και ίδια στο άνοιγμα και το κλείσιμο \lstinline{'Εισαγωγικά'} ή  \lstinline{"Εισαγωγικά"}. 

Αν θέλουμε να αλλάξουμε το 289 και να βάλουμε έναν άλλο αριθμό,π.χ. το 132 θα πρέπει να αντικαταστήσουμε το 289 μέσα σε όλες τις εντολές print με το 132.
\begin{lstlisting}
print("Οι προηγούμενοι αριθμοί είναι:")
print(132-3)
print(132-2)
print(132-1)
print("Οι επόμενοι αριθμοί είναι:")
print(132+1)
print(132+2)
\end{lstlisting}

Υπάρχει όμως ένας καλύτερος τρόπος, ο τρόπος αυτός είναι να δώσουμε ένα όνομα στον αριθμό μας. Μπορούμε να πούμε ότι το n είναι το όνομα του αριθμού. Αυτό γίνεται με την εντολή \lstinline{n=132}. Τότε το πρόγραμμά μας γίνεται:
\begin{lstlisting}
n = 132
print("Οι προηγούμενοι αριθμοί είναι:")
print(n-3)
print(n-2)
print(n-1)
print("Οι επόμενοι αριθμοί είναι:")
print(n+1)
print(n+2)
\end{lstlisting}

Μετά την εντολή \lstinline{n=132} η Python ξέρει ότι το n είναι ένα όνομα για το 132 και μπορεί να κάνει πράξεις με αυτό. Για παράδειγμα n+1 κάνει τώρα 133.

Αν θέλουμε να κάνουμε τώρα το ίδιο πρόγραμμα αλλά όχι για το 132 αλλά για το 210, χρειάζεται να αλλάξουμε μόνο μία γραμμή και το πρόγραμμά μας να γίνει ως εξής:
\begin{lstlisting}
n = 210
print("Οι προηγούμενοι αριθμοί είναι:")
print(n-3)
print(n-2)
print(n-1)
print("Οι επόμενοι αριθμοί είναι:")
print(n+1)
print(n+2)
\end{lstlisting}

Στην Python, όταν δίνουμε ένα όνομα σε έναν αριθμό (με τον τελεστή =) τότε δημιουργούμε μια μεταβλητή. Η μεταβλητή έχει ένα όνομα, στην περίπτωσή μας το n, και μια τιμή, στην περίπτωσή μας το 210.

Αν αντί για τους επόμενους δύο αριθμούς θέλαμε τους επόμενους \textbf{δέκα} θα γράφαμε ένα πρόγραμμα όπως το παρακάτω:
\begin{lstlisting}
n = 210
print(n)
print(n+1)
print(n+2)
print(n+3)
print(n+4)
print(n+5)
print(n+6)
print(n+7)
print(n+8)
print(n+9)
print(n+10)
\end{lstlisting}
Το παραπάνω πρόγραμμα εμφανίζει και τον αριθμό μας n, δηλαδή το 210.

Για να μην γράφουμε πολλές εντολές όταν κάνουμε το ίδιο πράγμα χρησιμοποιούμε την εντολή for.
Το πρόγραμμά μας με την for μπορεί να γίνει:
\begin{lstlisting}
n = 210
for i in 0,1,2,3,4,5,6,7,8,9,10:
    print(n+i)
\end{lstlisting}
Όταν γράψεις την for στην Python θα πρέπει να δηλώσεις ποιες εντολές θα εκτελεστούν πολλές φορές. Αυτή η δήλωση γίνεται βάζοντας αυτές τις εντολές λίγο πιο μέσα χρησιμοποιώντας το πλήκτρο κενό ή το πλήκτρο tab. Μια καλή πρακτική είναι να βάζεις τέσσερα κενά. Έτσι, πριν την εντολή \lstinline{print(n+i)} βάζεις τέσσερα κενά δηλαδή \lstinline[showspaces=true]{    print(n+i)}.
Το πρόγραμμα αυτό σημαίνει πως για το i μέσα στο σύνολο 0, 1, 2, 3, \ldots 10 και με αυτή τη σειρά εμφάνισε το n+i. Έτσι το αποτέλεσμα είναι το αναμενόμενο
\begin{lstlisting}
210
211
212
213
214
215
216
217
218
219
220
\end{lstlisting}

Στην Python υπάρχει ένας πιο εύκολος τρόπος να γράψουμε τους αριθμούς από το 0 έως το 10. Αυτός ο τρόπος είναι η εντολή range και συγκεκριμένα η range(11). Η range(11) φτιάχνει τους αριθμούς από το 0 μέχρι το 10 οι οποίοι είναι σε πλήθος 11. 
Έτσι το πρόγραμμά μας γίνεται:
\begin{lstlisting}
n = 210
for i in range(11):
    print(n+i)
\end{lstlisting}

Mπορούμε και να μετρήσουμε τους πρώτους 100 αριθμούς ως εξής:
\begin{lstlisting}
n = 210
for i in range(100):
    print(n+i)
\end{lstlisting}

Σκέψου αν θα δεις τον αριθμό 100 στο αποτέλεσμα του παραπάνω προγράμματος.

\section{Στρογγυλοποίηση}
Το βιβλίο των Μαθηματικών της Α' Γυμνασίου αναφέρει πως
Για να στρογγυλοποιήσουμε έναν φυσικό αριθμό \sel{12}:
\begin{enumerate}
	\item Προσδιορίζουμε την τάξη στην οποία θα γίνει η στρογγυλοποίηση
	\item Εξετάζουμε το ψηφίο της αμέσως μικρότερης τάξης
	\item Αν αυτό το ψηφίο είναι μικρότερο του 5 (δηλαδή 0, 1, 2, 3 ή 4) το ψηφίο αυτό και όλα τα ψηφία των υπόλοιπων τάξεων μηδενίζονται.
	\item Αν είναι μεγαλύτερο ή ίσο του 5 (δηλαδή 5, 6, 7, 8 ή 9) το ψηφίο αυτό και όλα τα ψηφία των υπόλοιπων τάξεων αντικαθίστανται από το 0 και το ψηφίο της τάξης στρογγυλοποίησης αυξάνεται κατά 1.
\end{enumerate}

Ας πούμε ότι θέλουμε να στρογγυλοποιήσουμε τον αριθμό 454.018.512 στα εκατομμύρια. Η απάντηση που περιμένουμε είναι 454 εκατομμύρια.
Για να τα καταφέρουμε θα χρησιμοποιήσουμε την διαίρεση. Όμως στην Python υπάρχουν \emph{δύο} διαιρέσεις μία με το σύμβολο / και μία με το σύμβολο //. Ας δούμε τις διαφορές τους στο REPL.
\begin{lstlisting}
>>> x = 454018512
>>> print(x/1000000)
454.018512
>>> print(x//1000000)
454
\end{lstlisting}
Η <<κανονική>> διαίρεση, με τη μία κάθετο /, δίνει το αποτέλεσμα της διαίρεσης με τα δεκαδικά ψηφία. Η <<ακέραια>> διαίρεση δίνει μόνο τον ακέραιο αριθμό. Δεν μπορούμε να πούμε ότι η ακέραια διαίρεση θα μας δώσει την στρογγυλοποίηση γιατί η ακέραια διαίρεση δεν στρογγυλοποιεί τα δεκαδικά ψηφία αλλά τα απορρίπτει εντελώς. Έτσι, ακόμη και αν είχαμε 454918512 κατοίκους η ακέραια διαίρεση θα δώσει 454 αντί για το στρογγυλοποιημένο που είναι 455.
\begin{lstlisting}
>>> x = 454918512
>>> print(x/1000000)
454.918512
>>> print(x//1000000)
454
\end{lstlisting}

Χρειάζεται επομένως να δούμε το ψηφίο της αμέσως χαμηλότερης τάξης το οποίο είναι το πρώτο δεκαδικό της κανονικής διαίρεσης. Για να το απομονώσουμε αφαιρούμε από το αποτέλεσμα της κανονικής διαίρεσης το ακέραιο μέρος.
\begin{lstlisting}
>>> x = 454018512
>>> x / 1000000 - x // 1000000
0.018511999999986983
\end{lstlisting}
Οπότε τώρα έχουμε δύο ενδεχόμενα αν το αποτέλεσμα αυτής της πράξης είναι μικρότερο από 0.5 όπως παραπάνω τότε το αποτέλεσμα που ψάχνουμε είναι το αποτέλεσμα της ακέραιας διαίρεσης. Αλλιώς πρέπει να προσθέσουμε ένα στο αποτέλεσμα της ακέραιας διαίρεσης.
Αυτό γίνεται με την εντολή if, που σημαίνει στα αγγλικά αν. Για ευκολία μπορούμε να ονομάσουμε d την διαφορά των δύο διαιρέσεων με την εντολή:
\begin{lstlisting}
d = x / 1000000 - x // 1000000
\end{lstlisting}
Επειδή το πρόγραμμα γίνεται μεγαλύτερο τώρα θα το γράψουμε στο πάνω παράθυρο του Mu.
\begin{lstlisting}
x = 454018512
d = x / 1000000 - x // 1000000
if d < 0.5:
    print(x // 1000000)
else:
    print(x // 1000000 + 1)
\end{lstlisting}
Την \lstinline{if} την γράφουμε ως εξής:
\begin{lstlisting}
if συνθήκη:
    εντολές που εκτελούνται
    αν ισχύει η συνθήκη
else:
    εντολές που εκτελούνται
    αν δεν ισχύει η συνθήκη
\end{lstlisting}
Θυμήσου να βάζεις την άνω κάτω τελεία μετά τη συνθήκη και μετά τη λέξη else που σημαίνει αλλιώς.

Αν στο ίδιο πρόγραμμα και βάλεις αντί για 454.018.512 τον αριθμό 454.918.512 θα δεις ότι θα εμφανιστεί το σωστό αποτέλεσμα (455).

Αν θέλεις στρογγυλοποίηση στις χιλιάδες τότε το πρόγραμμά σου γίνεται:
\begin{lstlisting}
x = 454018512
d = x / 1000 - x // 1000
if d < 0.5:
    print(x // 1000)
else:
    print(x // 1000 + 1)
\end{lstlisting}
και το αποτέλεσμα είναι 454019.

\begin{exercise}
Στο βιβλίο \sel{12} η στρογγυλοποίηση γίνεται στις δεκάδες των εκατομμυρίων. Μπορείς να γράψεις ένα πρόγραμμα που να στρογγυλοποιεί αριθμούς στις δεκάδες των εκατομμυρίων;
\end{exercise}

\egyptify{1}{1}{1}{1}{1}{1}{1}


Ανάπτυγμα
%\input{Kef3.tex}
%\chapter{Δεκαδικοί αριθμοί}
\section{Εισαγωγή}
Αν χωρίσουμε τη μονάδα σε 10 ίσα μέρη τότε μπορούμε να πάρουμε κλάσματα της μονάδας όπως $\frac{3}{10}$, $\frac{5}{10}$ κλπ. Τα κλάσματα είναι ομώνυμα συγκρίνονται εύκολα και βοηθάνε στις πράξεις. 
Γενικότερα, ονομάζουμε δεκαδικό κλάσμα οποιδήποτε κλάσμα έχει παρονομαστή μια δύναμη του 10. Κάθε δεκαδικό κλάσμα γράφεται σαν δεκαδικός αριθμός με τόσα δεκαδικά ψηφία όσα μηδενικά έχει ο παρονομαστής του.
Η Python χειρίζεται τους δεκαδικούς αριθμούς όπως και τους υπόλοιπους.
Δοκίμασε:
\begin{lstlisting}
>>> 0.3 + 0.5
0.8
>>> type(0.7)
<class 'float'>
\end{lstlisting}

Βλέπουμε ότι οι δεκαδικοί αριθμοί δεν είναι int, όπως οι ακέραιο αλλά float. Το όνομα float έχει να κάνει με τον τρόπο με τον οποίο ο υπολογιστής αποθηκεύει αποδοτικά αυτούς τους αριθμούς. 

Ας συνδυάσουμε τις γνώσεις από τα κλάσματα με τα κλάσματα που μάθαμε στο προηγούμενο κεφάλαιο.
\begin{lstlisting}
>>> from fractions import Fraction
>>> x = Fraction(3,10)
>>> float(x)
0.3
\end{lstlisting}

Το \lstinline{Fraction(3,10)} εννοεί το κλάσμα $\frac{3}{10}$ που είναι ίσο με 0,3. Όμως στην Python το 0,3 θα το γράφουμε με 0.3. Με τη συνάρτηση float μετατρέπουμε το $\frac{3}{10}$ σε δεκαδικό αριθμό.

\begin{exercise}
\sel{56} Γράψτε τους αριθμούς $\frac{3}{10}$, $\frac{825}{1000}$, $\frac{53}{1000}$, $\frac{1004}{10000}$.
\end{exercise}
\begin{lstlisting}
>>> float(Fraction(3,10))
0.625
>>> float(Fraction(825,100))
8.25
>>> float(Fraction(53,1000))
0.053
>>> float(Fraction(1004,10000))
0.1004
\end{lstlisting}

Η Python μπορεί να μετατρέψει τα κλάσματα σε δεκαδικό αριθμό ανεξάρτητα από τον παρονομαστή.
\begin{exercise}
\sel{59} Γράψε καθένα από τα παρακάτω κλάσματα, ως δεκαδικό αριθμό: (i) με προσέγγιση
εκατοστού και (ii) με προσέγγιση χιλιοστού: 

(α) $\frac{7}{16}$

(β) $\frac{21}{17}$

(γ) $\frac{20}{95}$
\end{exercise}
\begin{lstlisting}
>>> x = Fraction(7,16)
>>> float(x)
0.4375
>>> round(float(x),2)
0.44
>>> round(float(x),3)
0.438
>>> x = Fraction(21,17)
>>> float(x)
1.2352941176470589
>>> round(float(x),2)
1.24
>>> round(float(x),3)
1.235
>>> x = Fraction(20,95)
>>> float(x)
0.21052631578947367
>>> round(float(x),2)
0.21
>>> round(float(x),3)
0.211
\end{lstlisting}


Η στρογγυλοποίηση των δεκαδικών υλοποιείται στην Python με τη συνάρτηση round. Οπότε μπορείς να στρογγυλοποιήσεις εύκολα δεκαδικούς αριθμούς ως εξής:
\begin{exercise}
Να στρογγυλοποιήσεις τους παρακάτω δεκαδικούς αριθμούς στο δέκατο, εκατοστό και
χιλιοστό: 

(α) 9876,008, 

(β) 67,8956, 

(γ) 0,001, 

(δ) 8,239, 

(ε) 23,7048.
\end{exercise}
Θυμόμαστε να αλλάζουμε την υποδιαστολή από κόμμα σε τελεία:
\begin{lstlisting}
def roundall(x):
    print(round(x,1))
    print(round(x,2))
    print(round(x,3))

roundall(9876.008)
roundall(67.8956)
roundall(0.001)
roundall(8.239)
roundall(23.7048)
\end{lstlisting}

To αποτέλεσμα είναι:
\begin{lstlisting}
67.9
67.9
67.896
0.0
0.0
0.001
8.2
8.24
8.239
23.7
23.7
23.705
\end{lstlisting}

\begin{exercise}
\sel{59} Στον αριθμό $34,\square\square\square$ λείπουν τα τελευταία τρία ψηφία του. Να συμπληρώσεις τον
αριθμό με τα ψηφία 9, 5 και 2, έτσι ώστε κάθε ψηφίο να γράφεται μία μόνο φορά. Να γράψεις όλους τους δεκαδικούς που μπορείς να βρεις και να τους διατάξεις σε φθίνουσα σειρά.
\end{exercise}

Πώς μπορεί η Python να βρει όλους τους πιθανούς συνδυασμούς του 9,5,2;
Δοκίμασε τη βιβλιοθήκη itertools και συγκεκριμένα τη συνάρτηση permutations.
\begin{lstlisting}
>>> from itertools import permutations
>>> x = permutations([1,2,3])
>>> print(x)
<itertools.permutations object at 0x012BE1B0>
>>> print(list(x))
[(1, 2, 3), (1, 3, 2), (2, 1, 3), (2, 3, 1), (3, 1, 2), (3, 2, 1)]
\end{lstlisting}
Έτσι με την permutations μπορείς να βρεις όλες τις αναδιατάξεις των αριθμών. Οπότε τώρα το πρόγραμμα μπορεί να γίνει ως εξής:
\begin{lstlisting}
lista = []
from itertools import permutations
for p in permutations([9,5,2]):
    lista.append(34+p[0]/10+p[1]/100+p[2]/1000)
print(lista)
\end{lstlisting}
Που δίνει το αποτέλεσμα:
\begin{lstlisting}
[34.952, 34.925000000000004, 34.592000000000006, 34.529, 
34.29500000000001, 34.259]
\end{lstlisting}
Τα ψηφία που εμφανίζονται στο τέλος των αριθμών προκύπτουν από την αναπαράσταση των δεκαδικών στον υπολογιστή που υπόκειται σε κάποιους περιορισμούς.
Αν δεν θέλουμε να εμφανίζονται μπορούμε να αλλάξουμε το for σε:
\begin{lstlisting}
for p in permutations([9,5,2]):
    ar = 34+p[0]/10+p[1]/100+p[2]/1000
    lista.append(round(ar,3))
\end{lstlisting}
Τώρα για να γράψουμε τους αριθμούς με φθίνουσα σειρά θα δοκιμάσουμε τη sorted. Η sorted ταξινομεί τους αριθμούς που δίνονται σε μια λίστα. Δοκίμασε:
\begin{lstlisting}
>>> sorted([4,2,3])
[2, 3, 4]
\end{lstlisting}
Έτσι το συνολικό πρόγραμμα γίνεται:
\begin{lstlisting}
lista = []
from itertools import permutations
for p in permutations([9,5,2]):
    ar = 34+p[0]/10+p[1]/100+p[2]/1000
    lista.append(round(ar,3))
print(sorted(lista))
\end{lstlisting}

Που δίνει το αποτέλεσμα:
\begin{lstlisting}
[34.259, 34.295, 34.529, 34.592, 34.925, 34.952]
\end{lstlisting}

Όμως η άσκηση μας ζητάει να τυπώσουμε τη λίστα με φθίνουσα σειρά. Αυτό μπορεί να γίνει δηλώνοντας στη sorted ότι θέλουμε αντίστροφη σειρά γράφοντας \lstinline{reverse=True}. Το τελικό πρόγραμμα είναι το εξής:
\begin{lstlisting}
lista = []
from itertools import permutations
for p in permutations([9,5,2]):
    ar = 34+p[0]/10+p[1]/100+p[2]/1000
    lista.append(round(ar,3))
print(sorted(lista,reverse=True))
\end{lstlisting}

Μια μικρή τροποποίηση που μπορεί να γίνει για να εμφανιστούν οι αριθμοί σε διαφορετικές γραμμές είναι να τυπώσουμε τη λίστα με μια for.
\begin{lstlisting}
lista = []
from itertools import permutations
for p in permutations([9,5,2]):
    ar = 34+p[0]/10+p[1]/100+p[2]/1000
    lista.append(round(ar,3))

for x in sorted(lista,reverse=True):
    print(x)
\end{lstlisting}

\begin{exercise}
\sel{61} Να υπολογίσεις τα αθροίσματα:

(α) $48,18 + 3,256 + 7,129$

(β) $3,59 + 7,13 + 8,195$
\end{exercise}

\begin{lstlisting}
>>> 48.18+3.256+7.129
58.565
>>> 3.59 + 7.13 + 8.195
18.915
\end{lstlisting}
\begin{exercise}
\sel{61}
Να υπολογίσεις το μήκος της περιμέτρου των οικοπέδων:
(Σχήμα ---)
\end{exercise}
\begin{lstlisting}
>>> 26.14 + 80.19 + 29.13+38.13+23.24+57.89+80.19
334.91
>>> 39.93+80.19+57.89+47.73+44.75+48.9+47.19
366.58
\end{lstlisting}
\begin{exercise}
\sel{61} Να κάνεις τις διαρέσεις:
(α) $579:48$

(β) $314:25$

(γ) $520:5,14$

(δ) $49,35:7$

\end{exercise}
\begin{lstlisting}
>>> 579/48
12.0625
>>> 314/25
12.56
>>> 520/5.14
101.16731517509729
>>> 49.35/7
7.05
\end{lstlisting}
\begin{exercise}
\sel{61}
Να κάνεις τις πράξεις: 

(α) $520 \cdot 0,1 + 0,32 \cdot 100 $

(β) $4,91 \cdot 0,01 + 0,819 \cdot 10$

\end{exercise}

\begin{lstlisting}
>>> 520*0.1 + 0.32*100
84.0
>>> 4.91*0.01 + 0.819*10
8.239099999999999
\end
\end{lstlisting}

Σε αυτή την άσκηση βλέπουμε ότι ο υπολογιστής προσεγγίζει τα αποτελέσματα με τον δικό του τρόπο.
Δοκίμασε:
\begin{lstlisting}
>>> x = 520*0.1 + 0.32*100
>>> x
84.0
>>> type(x)
<class 'float'>
>>> y = int(x)
>>> type(y)
<class 'int'>
>>> x == y
True
\end{lstlisting}
Αυτό σημαίνει πως ο ακέραιος αριθμός 84, και κάθε ακέραιος, στην Python μπορεί να αναπαρασταθεί σαν ακέραιος αλλά και σαν float με μηδενικά δεκαδικά ψηφία.
Στην δεύτερη πράξη παρατηρούμε ότι αντί για το σωστό αποτέλεσμα που είναι $0,0491+8,19=8,2391$ η Python εμφανίζει μια προσέγγιση που είναι $8.239099999999999$. Η διαφορά είναι πολύ μικρή. Ωστόσο οι δύο ποσότητες δεν είναι ίσες.
Δοκίμασε:
\begin{lstlisting}
>>> 4.91*0.01 + 0.819*10 == 8.2391
False
>>> 8.2391 - 4.91*0.01 + 0.819*10 
1.7763568394002505e-15
\end{lstlisting}
Ο αριθμός \lstinline{1.7763568394002505e-15} σημαίνει πως η διαφορά είναι περίπου $1.77\cot 10^{-15}$ που είναι πάρα πολύ μικρή και προκύπτει από τον τρόπο με τον οποίο η Python αποθηκεύει τους αριθμούς.

\begin{exercise}
\sel{61}
Να κάνεις τις πράξεις:

(α) $4,7:0,1-45:10$

(β) $0,98:0,0001 - 6785:1000$
\end{exercise}

\begin{lstlisting}
>>> 4.7/0.1 - 45/10
42.5
>>> 0.98/0.0001 - 6785/1000
9793.215
\end{lstlisting}
Βλέπουμε ότι η Python υπολογίζει σωστά πρώτα τη διαίρεση και μετά την αφαίρεση.

\begin{exercise}
\sel{61}
Η περίμετρος ενός τετραγώνου είναι 20,2. Να υπολογίσεις την πλευρά του.
\end{exercise}
\begin{lstlisting}
>>> 20.2/4
5.05
\end{lstlisting}

\begin{exercise}
\sel{61}
Η περίμετρος ενός ισοσκελούς τριγώνου είναι 48,52. Αν η βάση του είναι 10,7, πόσο είναι η κάθε μία από τις ίσες πλευρές του;
\end{exercise}
Αφαιρούμε πρώτα από το 48,52 το 10,7. Το αποτελέσμα το διαιρούμε με το δυο.
\begin{lstlisting}
>>> 48.52-10.7
37.82000000000001
>>> 37.82/2
18.91
\end{lstlisting}

\begin{exercise}
\sel{61}
Να υπολογίσεις τις τιμές των αριθμητικών παραστάσεων:

(α) $24\cdot 5 - 2 + 3 \cdot 5$

(β) $3\cdot 11 -2 + 45,1 : 2$
\end{exercise}
\begin{lstlisting}
>>> 24*5 - 2 +3*5
133
>>> 3*11 - 2 + 54.1/2
58.05
\end{lstlisting}

\begin{exercise}
\sel{61}
Να υπολογίσεις τις δυνάμεις:
(α) $3,1^2$, (β) $7,01^2$, (γ) $4,5^2$, (δ) $0,5^2$, (ε) $0,2^2$, (στ) $0,3^3$
\end{exercise}
\begin{lstlisting}
>>> 3.1**2
9.610000000000001
>>> 7.01**2
49.1401
>>> 4.5**2
20.25
>>> 0.5**2
0.25
>>> 0.2**2
0.04000000000000001
>>> 0.3*3
0.8999999999999999
\end{lstlisting}
Πάλι κάνουν την εμφάνισή τους μικρές προσεγγίσεις.

\begin{exercise}
Τοποθέτησε ένα ``x'' στην αντίστοιχη θέση (ΣΩΣΤΟ ΛΑΘΟΣ)
(α) $2,75 + 0,05 + 1,40 + 16,80 = 21$
(β) $420,510 + 72,490 + 45,19 + 11,81 = 500$
(γ) $4 – 3,852 = 1,148$
(δ) $32,01 – 4,001 = 28,01$
(ε) $41900 \cdot 0,0001 – 0,0419 \cdot 1000 = 0$
(στ) $56,89 \cdot 0,01 + 4311 : 10000 = 1$
(ζ) $(3,2 + 7,2 \cdot 2 + 24 \cdot 0,1) : 100 = 0,2$
\end{exercise}

(α)
\begin{lstlisting}
>>> 2.75 + 0.05 + 1.40 + 16.80 == 21
True
>>> 2.75 + 0.05 + 1.40 + 16.80
21.0
\end{lstlisting}
Άρα Σωστό

(β)
\begin{lstlisting}
>>> 420.510 + 72.490 + 45.19 + 11.81 == 500
False
>>> 420.510 + 72.490 + 45.19 + 11.81
550.0
\end{lstlisting}
Άρα Λάθος

(γ)
\begin{lstlisting}
>>> 4 - 3.852 == 1.148
False
>>> 4 - 3.852
0.14800000000000013
\end{lstlisting}
Άρα Λάθος

(δ)
\begin{lstlisting}
>>> 32.01 - 4.001 == 28.01
False
>>> 32.01 - 4.001
28.008999999999997
\end{lstlisting}
Άρα Λάθος

(ε)
\begin{lstlisting}
>>> 41900*0.0001 - 0.0419*1000 == 0
False
>>> 41900*0.0001 - 0.0419*1000
-37.71
\end{lstlisting}
Άρα Λάθος

(στ)
\begin{lstlisting}
>>> 56.89*0.01 + 4311 / 10000 == 1
True
>>> 56.89*0.01 + 4311 / 10000
1.0
\end{lstlisting}
Άρα Σωστό

και 

(ζ)
\begin{lstlisting}
>>> (3.2 + 7.2*2 + 24*0.1) / 100 == 0.2
True
>>> (3.2 + 7.2*2 + 24*0.1) / 100
0.2
\end{lstlisting}

Άρα Σωστό.
%\chapter{Εξισώσεις και προβλήματα}

Σε αυτό το κεφάλαιο θα χρησιμοποιήσουμε τη βιβλιοθήκη sympy.
Υπάρχει ένα περιβάλλον στο οποίο μπορούμε να πληκτρολογούμε εντολές της βιβλιοθήκης ώστε να βλέπουμε τα αποτελέσματα με φιλικό τρόπο στον φυλλομετρήτή μας, συνήθως Chrome, Firefox ή Microsoft Edge. Το περιβάλλον αυτό βρίσκεται στη διεύθυνση https://live.sympy.org/. Μπορούμε να κάνουμε τα ίδια παραδείγματα στον Mu Editor όπως έχουμε συνηθίσει χρησιμοποιώντας την εντολή:
\begin{lstlisting}
from sympy import *
\end{lstlisting}
όμως τα αποτελέσματα δεν θα εμφανίζονται με φιλικό τρόπο αλλά με τον συμβολισμό της Python.
\section{Η έννοια της εξίσωσης}
\begin{exercise}
Γράψε συντομότερα τις εκφράσεις:

(α) $x + x + x + x$, 

(β) $\alpha + \alpha + \alpha + \beta + \beta$, 

(γ) $3\cdot \alpha + 5 \cdot \alpha$, 

(δ) $18 \cdot x + 7 \cdot x + 4 \cdot x$, 

(ε) $15 \cdot \beta – 9 \cdot \beta$.
\end{exercise}

Επειδή τα σύμβολα είναι τα $x, a, b$ θα πρέπει να τα δηλώσουμε στο sympy. Αυτό γίνεται ως εξής:
\begin{lstlisting}
from sympy import *
x,a,b = symbols("x a b")
\end{lstlisting}

Στη συνέχεια όποτε αναφέρουμε τα $x, a, b$ η Python θα καταλαβαίνει ότι πρόκειται για σύμβολα και θα δρα ανάλογα.
Έτσι αν δώσουμε στην Python
\begin{lstlisting}
>>> x + x + x + x
\end{lstlisting}
Θα μας δώσει ως απάντηση
$$4x$$
στο live.sympy.org
και
\begin{lstlisting}
4*x
\end{lstlisting}
στην απλή Python ή στο Mu Editor.
Άρα το 
\begin{lstlisting}
>>> a + a + a + b + b
\end{lstlisting}
Θα μας δώσει σαν απάντηση:
$$3a+2b$$
και τα 
\begin{lstlisting}
>>> 3*a + 5*a 
>>> 18*x + 7*x + 4*x
>>> 15*b - 9b
\end{lstlisting}
$$8a$$
$$29x$$
$$6b$$
αντίστοιχα.
\begin{exercise}
Να αντικαταστήσεις το x, με τους αριθμούς 1, 3, 4, 5, 6 και 11, σε κάθε ισότητα της πρώτης στήλης, του παρακάτω πίνακα. Βρες ποιος από αυτούς την επαληθεύει και ποιος όχι.

\begin{tabular}{|c|c|c|}
Εξίσωση            &Αριθμοί που την επαληθεύουν     &Αριθμοί που δεν την επαληθεύουν\\
$x – 4 = 1$        &                             &                                \\
$5 – x = 4$        &                             &                                \\
$2x = 8$           &                             &                                \\
$\frac{6}{x} = 2$  &                             &                                \\
$\frac{x}{2} = 3$  &                             &                                \\
$x + 7 = 30$       &                             &                                \\
\end{tabular}
\end{exercise}

\begin{lstlisting}
>>> e = x - 4
>>> e.subs(x,1)
\end{lstlisting}
$$−3$$
\begin{lstlisting}
>>> e.subs(x,3)
\end{lstlisting}
$$−1$$
\begin{lstlisting}
>>> e.subs(x,4)
\end{lstlisting}
$$0$$
\begin{lstlisting}
>>> e.subs(x,5)
\end{lstlisting}
$$1$$
\begin{lstlisting}
>>> e.subs(x,6)
\end{lstlisting}
$$2$$
\begin{lstlisting}
>>> e.subs(x,11)
\end{lstlisting}
$$7$$

Οπότε ο αριθμός που την επαληθεύει είναι ο 5 και όλοι οι υπόλοιποι δεν την επαληθεύουν.

\begin{lstlisting}
>>> e = 5 - x
>>> e.subs(x,1)
\end{lstlisting}
$$4$$
\begin{lstlisting}
>>> e.subs(x,3)
\end{lstlisting}
$$2$$
\begin{lstlisting}
>>> e.subs(x,4)
\end{lstlisting}
$$1$$
\begin{lstlisting}
>>> e.subs(x,5)
\end{lstlisting}
$$0$$
\begin{lstlisting}
>>> e.subs(x,6)
\end{lstlisting}
$$-1$$
\begin{lstlisting}
>>> e.subs(x,11)
\end{lstlisting}
$$-6$$

Οπότε ο αριθμός που την επαληθεύει είναι ο 1 και όλοι οι υπόλοιποι δεν την επαληθεύουν.

\begin{lstlisting}
>>> e = 2*x
>>> e.subs(x,1)
\end{lstlisting}
$$2$$
\begin{lstlisting}
>>> e.subs(x,3)
\end{lstlisting}
$$6$$
\begin{lstlisting}
>>> e.subs(x,4)
\end{lstlisting}
$$8$$
\begin{lstlisting}
>>> e.subs(x,5)
\end{lstlisting}
$$10$$
\begin{lstlisting}
>>> e.subs(x,6)
\end{lstlisting}
$$12$$
\begin{lstlisting}
>>> e.subs(x,11)
\end{lstlisting}
$$22$$

Οπότε ο αριθμός που την επαληθεύει είναι ο 4 και όλοι οι υπόλοιποι δεν την επαληθεύουν.

\begin{lstlisting}
>>> e = 6/x
>>> e.subs(x,1)
\end{lstlisting}
$$6$$
\begin{lstlisting}
>>> e.subs(x,3)
\end{lstlisting}
$$2$$
\begin{lstlisting}
>>> e.subs(x,4)
\end{lstlisting}
$$\frac{3}{2}$$
\begin{lstlisting}
>>> e.subs(x,5)
\end{lstlisting}
$$\frac{6}{5}$$
\begin{lstlisting}
>>> e.subs(x,6)
\end{lstlisting}
$$1$$
\begin{lstlisting}
>>> e.subs(x,11)
\end{lstlisting}
$$\frac{6}{11}$$

Οπότε ο αριθμός 3 επαληθεύει την εξίσωση και όλοι οι υπόλοιποι δεν την επαληθεύουν:


\begin{lstlisting}
>>> e = x/2
>>> e.subs(x,1)
\end{lstlisting}
$$6$$
\begin{lstlisting}
>>> e.subs(x,3)
\end{lstlisting}
$$2$$
\begin{lstlisting}
>>> e.subs(x,4)
\end{lstlisting}
$$\frac{3}{2}$$
\begin{lstlisting}
>>> e.subs(x,5)
\end{lstlisting}
$$\frac{6}{5}$$
\begin{lstlisting}
>>> e.subs(x,6)
\end{lstlisting}
$$1$$
\begin{lstlisting}
>>> e.subs(x,11)
\end{lstlisting}
$$\frac{6}{11}$$

\begin{lstlisting}
>>> e = x/2
>>> e.subs(x,1)
\end{lstlisting}
$$\frac{1}{2}$$

\begin{lstlisting}
>>> e.subs(x,3)
\end{lstlisting}
$$\frac{3}{2}$$

\begin{lstlisting}
>>> e.subs(x,4)
\end{lstlisting}
$$2$$

\begin{lstlisting}
>>> e.subs(x,5)
\end{lstlisting}
$$\frac{5}{2}$$

\begin{lstlisting}
>>> e.subs(x,6)
\end{lstlisting}
$$3$$
\begin{lstlisting}
>>> e.subs(x,11)
\end{lstlisting}
$$\frac{11}{2}$$

Ο αριθμός που επαληθεύει την εξίσωση είναι ο 6, οι υπόλοιποι αριθμοί δεν την επαληθεύουν.

\begin{lstlisting}
>>> e = x + 7
>>> e.subs(x,1)
\end{lstlisting}
$$8$$

\begin{lstlisting}
>>> e.subs(x,3)
\end{lstlisting}
$$10$$

\begin{lstlisting}
>>> e.subs(x,4)
\end{lstlisting}
$$11$$

\begin{lstlisting}
>>> e.subs(x,5)
\end{lstlisting}
$$12$$

\begin{lstlisting}
>>> e.subs(x,6)
\end{lstlisting}
$$13$$
\begin{lstlisting}
>>> e.subs(x,11)
\end{lstlisting}
$$18$$

Κανένας από αυτούς τους αριθμούς δεν επαληθεύει την εξίσωση, οπότε:

\begin{tabular}{|c|c|c|}
Εξίσωση            &Αριθμοί που την επαληθεύουν  &Αριθμοί που δεν την επαληθεύουν\\
$x - 4 = 1$        &            5                &          1, 3, 4, 6 και 11    \\
$5 - x = 4$        &            1                &          3, 4, 5, 6 και 11    \\
$2x = 8$           &            4                &          1, 3, 5, 6 και 11    \\
$\frac{6}{x} = 2$  &            3                &          1, 4, 5, 6 και 11    \\
$\frac{x}{2} = 3$  &            6                &          1, 3, 4, 5 και 11    \\
$x + 7 = 30$       &                             &          1, 3, 4, 5, 6 και 11 \\
\end{tabular}

Ένας καλύτερος τρόπος για να έχουμε το ίδιο αποτέλεσμα είναι να γραφτεί ένα πρόγραμμα που να υπολογίζει τα αποτελέσματα για όλους τους αριθμούς και να συγκρίνει το αποτέλεσμα με το αναμενόμενο. Η enumerate μετράει τη λίστα και δημιουγεί έναν μετρητή με όνομα i που μπορούμε να τον χρησιμοποιήσουμε για να μετρήσουμε τα αναμενόμενα αποτελέσματα:
\begin{lstlisting}
for e in exprs: 
    for (i,xi) in enumerate([1,3,4,5,6,11]):
        print(e,xi,e.subs(x,xi),res[i])
        print(e.subs(x,xi)==res[i])
\end{lstlisting}
\begin{exercise}
\sel{73}
Να λυθούν οι εξισώσεις:
$$x+5=12$$
$$y-2=3$$
$$10-z =1$$
$$7\cdot phi = 14$$
$$w:5 = 4$$
$$24:\psi = 6$$
\end{exercise}
Η βιβλιοθήκη sympy έχει συνάρτηση solve για να λύνει εξισώσεις όταν το δεξί μέρος της εξίσωσης είναι 0 οπότε οι εξισώσεις πρέπει να μετατραπούν με το χέρι σε:
$$x+5-12 = 0$$
$$y-2-3 = 0$$
$$10-z -1 = 0$$
$$7\cdot phi - 14 = 0$$
$$w:5 - 4 = 0$$
$$24:\psi - 6 = 0$$

\begin{lstlisting}
>>> from sympy import *
>>> x,y,z,f,w,psi = symbols('x y z f w psi')
>>> solve(x+5-12)
[7]
>>> solve(y-2-3)
[5]
>>> solve(10-z -1)
[9]
>>> solve(7* f - 14)
[2]
>>> solve(w/5 - 4)
[20]
>>> solve(24/psi - 6)
[4]
\end{lstlisting}
Η συνάρτηση solve επιστρέφει μια λίστα με τις τιμές που επαληθεύουν την εξίσωση. Επειδή υπάρχει μόνο μία τιμή που επαληθεύει την εξίσωση για αυτό το λόγο υπάρχει μόνο μία τιμή στην κάθε λίστα.

\begin{exercise}
\sel{63}
Μια δεξαμενή χωρητικότητας 6m$^3$ που έχει μήκος 1,5m και πλάτος 2m, έχει
ύψος (α) 1,5m ή (β) 3m ή (γ) 2m;
\end{exercise}

\begin{lstlisting}
>>> solve(2*1.5*x - 6)
[2.0]
\end{lstlisting}
\begin{exercise}
\sel[4]{74}Γράψε με απλούστερο τρόπο τις μαθηματικές εκφράσεις: 

(α) $x+x$,

(β) $\alpha+\alpha+\alpha$,

(γ) $3\cdot \alpha+52\cdot \alpha$, 

(δ) $2\cdot \beta+\beta+3\cdot \alpha+2\cdot \alpha$, 

(ε) $4\cdot x+8\cdot x–3\cdot x$, 

(στ) $7\cdot \omega+4\cdot \omega–10\cdot \omega$

\end{exercise}

\begin{lstlisting}
>>> x+x
\end{lstlisting}

$$2x$$

\begin{lstlisting}
>>> a = symbols('a')
>>> a+a+a
\end{lstlisting}

$$3a$$

\begin{lstlisting}
>>>  3*a  + 52 * a
\end{lstlisting}

$$55a$$

\begin{lstlisting}
>>> a,b = symbols('a b')
>>> 2*b+b+3*a+2*a 
\end{lstlisting}

$$5a+3b$$

\begin{lstlisting}
>>> 4*x+8*x–3*x 
\end{lstlisting}

$$9x$$

\begin{lstlisting}
>>> w = symbols('w')
>>> 7*w+4*w–10*w
\end{lstlisting}

$$w$$

\begin{exercise}
\sel[6]{74}
Στην εξίσωση 2 + α = x, το α και το x είναι φυσικοί αριθμοί. Ποια από τις τιμές
0, 3, 1 μπορεί να πάρει το x ;
\end{exercise}
Θα λύσουμε την $$2+a-x=0$$ για αυτές τις τιμές:
\begin{lstlisting}
>>> solve(2+a-0)
[-2]
>>> solve(2+a-3)
[1]
>>> solve(2+a-1)
[-1]
\end{lstlisting}
Από αυτές τις λύσεις συμπεραίνουμε ότι μόνο η $2+a-3$ μπορεί να ισχύει για φυσικό αριθμό και άρα μόνο την τιμή $3$ μπορεί να πάρει το $x$.
\begin{exercise}
\sel[7]{74}
Να εξετάσεις, αν ο αριθμός 12 είναι η λύση της εξίσωσης: x + 13 = 25
\end{exercise}
\begin{lstlisting}
>>> e = x + 13
>>> e.subs(e,x,12)
\end{lstlisting}
$$25$$

%\chapter{Ποσοστά}
Στον διπλανό πίνακα φαίνεται το σύνολο των
πολιτών που ψήφισαν στα χωριά
Α, Β, Γ και Δ και οι ψήφοι που πήραν οι
αντίστοιχοι πρόεδροι που εκλέχτηκαν.
Βρες, ποιος από τους προέδρους που
εκλέχτηκαν, είναι ο πιο δημοφιλής.

\begin{tabular}{ccc}
Κοινότητα & Ψηφίσαντες & Ο πρόεδρος ψηφίστηκε από\\
A& 585& 354\\
B& 3.460& 1.802\\
Γ& 456&312\\
Δ&1.295&823\\
\end{tabular}   

\begin{lstlisting}
>>> 354/585
0.6051282051282051
\end{lstlisting}
Όμως για να το κάνουμε σαν ποσοστό \% τότε θα πρέπει να το πολλαπλασιάσουμε με το 100 οπότε:
\begin{lstlisting}
>>> 354/585*100
60.51282051282051
\end{lstlisting}
Επίσης καλό είναι η στρογγυλοποίηση να γίνει στο δεύτερο δεκαδικό ψηφίο. Οπότε 
\begin{lstlisting}
>>> round(354/585*100,2)
60.51
\end{lstlisting}
Μπορούμε να φτιάξουμε μια μικρή συνάρτηση που να τυπώνει σε ποσοστό έναν αριθμό ως εξής:
\begin{lstlisting}
def pososto(x):
    print(str(round(x*100,2))+'%')

pososto(354/585)
\end{lstlisting}

Που  δίνει το αποτέλεσμα 60.51\%
Οπότε έχουμε
\begin{lstlisting}
>>> pososto(354/585)
60.51%
>>> pososto(1802/3460)
52.08%
>>> pososto(312/456)
68.42%
>>> pososto(823/1295)
63.55%
\end{lstlisting}

Ο πιο δημοφιλής είναι ο Γ που έχει 68.42\%.
Αν θέλουμε όμως η Python να λύσει το πρόβλημα τότε μπορούμε να φτιάξουμε το εξής:
\begin{lstlisting}
class proedros():
    def __init__(self,onoma,psifoi,katoikoi):
        self.onoma = onoma
        self.psifoi = psifoi
        self.katoikoi = katoikoi
    def pososto(self):
        return(round(self.psifoi/self.katoikoi*100,2))

A = proedros('A',354,585)
B = proedros('B',1802,3460)
C = proedros('C',312,456)
D = proedros('D',823,1295)

M = max([A,B,C,D],key=lambda x:x.pososto());
print(M.onoma)
\end{lstlisting}

Που δίνει το αποτέλεσμα C δηλαδή Γ.

\begin{exercise}
Να γραφούν, ως ποσοστά επί τοις εκατό, τα παρακάτω κλάσματα:
(α) $\frac{4}{5}$
(β) $\frac{3}{8}$
(γ) $\frac{84}{91}$

με στρογγυλοποίηση στο εκατοστό.
\end{exercise}

Επειδή η συνάρτηση που έχουμε φτιάξει δεν προσαρμόζει την στρογγυλοποίηση μπορείς να την αλλάξεις ώστε να έχει έξτρα αυτό το δεδομένο. Μάλιστα μπορείς να δηλώσεις στην Python ότι αν δεν γράψεις αυτό το στοιχέιο θα είναι 0.
\begin{lstlisting}
def pososto(x,strog = 2):
    print(str(round(x*100,strog))+'%')

pososto(4/5,strog=0)
pososto(3/8,strog=0)
pososto(84/91,strog=0)
\end{lstlisting}

Έχουμε το αποτέλεσμα:
\begin{lstlisting}
80.0%
38.0%
92.0%
\end{lstlisting}
Μπορείς να αλλάξεις τη συνάρτηση ώστε να κάνει το αποτέλεσμα ακέραιο ειδικά αν το strog είναι 0.
\begin{lstlisting}
def pososto(x,strog = 2):
    if strog == 0:
        print(str(int(round(x*100),0))+'%')
    else:
        print(round(int(x*100),strog)+'%')

pososto(4/5,strog=0)
pososto(3/8,strog=0)
pososto(84/91,strog=0)
\end{lstlisting}

Τότε το αποτέλεσμα θα είναι:
\begin{lstlisting}
80%
37%<--!!!!!!!!!!!!!
92%
\end{lstlisting}
\begin{exercise}
\sel{81}
Να γραφούν, ως κλάσματα, τα ακόλουθα ποσοστά: (α) 12\%, (β) 73\%, (γ) 32,5\%.
\end{exercise}

\begin{lstlisting}
from fractions import Fraction
strx = input('Ποσοστό:')
if strx[-1] == '%':
    strx=strx[:-1]

fx = float(strx)
denom = 100
while int(fx) != fx:
    fx *= 10
    denom *= 10
fx = int(fx)
print(Fraction(fx,denom))
\end{lstlisting}
\begin{lstlisting}
Ποσοστό:12
3/25
Ποσοστό:73
73/100
Ποσοστό:32.5
13/40
\end{lstlisting}
\begin{exercise}
\sel{81}
Ποια θα είναι η τιμή πώλησης ενός πουλόβερ, αξίας 150€, με επιβάρυνση Φ.Π.Α. 19\%;
\end{exercise}
\begin{lstlisting}
>>> 150 + 150*19/100
178.5
\end{lstlisting}
\begin{exercise}
\sel[1]{81}
Γράψε   ως  ποσοστά επί τοις    εκατό,  τα  κλάσματα:   

(α)  $\frac{1}{5}$ , (β)     $\frac{3}{2}$ , (γ)    $\frac{1}{4}$ , (δ) $\frac{3}{4}$,  (ε) $\frac{3}{5}$
\end{exercise}

\begin{lstlisting}
>>> pososto(1/5)
20.0%
>>> pososto(3/2)
150.0%
>>> pososto(1/4)
25.0%
>>> pososto(3/4)
75.0%
>>> pososto(3/5)
60.0%
\end{lstlisting}
\begin{exercise}
\sel[2]{81}
Να  μετατρέψεις σε  ποσοστά επί τοις    εκατό,  τους    δεκαδικούς  αριθμούς:
(α) 0,52    ,           (β) 3,41    ,           (γ) 0,19    ,           (δ) 0,03    ,           (ε) 0,07.
\end{exercise}
\begin{lstlisting}
>>> 0.52*100
52
>>> 3.41*100
341
>>> 0.19*100
19
>>> 0.03*100
3
>>> 0.07*100
7
\end{lstlisting}
Άρα 52\%, 341\%, 19\%, 3\%, 7\%.
\begin{exercise}
\sel[3]{81}
Να  μετατρέψεις σε  δεκαδικά    κλάσματα    τα  ποσοστά:    (α) 15\%,    (β)7\%,  (γ)48\%, (δ) 50\%.    Στη 
συνέχεια,   απλοποίησε  τα  δεκαδικά    κλάσματα,   έως ότου    φτάσεις σε  ανάγωγο κλάσμα.
\end{exercise}
Θα μετατρέψουμε τον προηγούμενο κώδικα σε συνάρτηση:
\begin{lstlisting}
def posostoseklasma(fx):
    fx = float(fx)
    denom = 100
    while int(fx) != fx:
         fx *= 10
         denom *= 10
    fx = int(fx)
    return(Fraction(fx,denom))

print(posostoseklasma(15))
print(posostoseklasma(7))
print(posostoseklasma(48))
print(posostoseklasma(50))
\end{lstlisting}
Και το αποτέλεσμα είναι:
\begin{lstlisting}
3/20
7/100
12/25
1/2
\end{lstlisting}

\begin{exercise}
\sel[4]{81}
Υπολόγισε:  (α) το  10\% των 3000€,  (β) το  45\% της 1   ώρας,   (γ) το  20\% του λίτρου,
(δ) το  50\% των 500 γραμμαρίων, (ε) το  25\% του 1   κιλού.
\end{exercise}
\begin{lstlisting}
>>> 10/100*3000
300.0
\end{lstlisting}
Άρα 300€.

\begin{lstlisting}
>>> 45/100*60
27.0
\end{lstlisting}
Άρα 27 λεπτά.

\begin{lstlisting}
>>> 20/100*1000
200.0
\end{lstlisting}
Άρα 200ml

\begin{lstlisting}
>>> 50/100*500
250.0
\end{lstlisting}
Άρα 250g.

\begin{lstlisting}
>>> 25/100*1000
250.0
\end{lstlisting}
Άρα 250 γραμμάρια.

\begin{exercise}
\sel[5]{81}
Βρες     τι  ποσοστό     είναι:  (α)     τα  50€    για     τα  1.000€,  (β)     οι  30  ημέρες  για  το   1   έτος,
(γ) τα  50  στρέμματα   για τα  2.500   στρέμματα,  (δ) οι  3   παλάμες για τα  10  μέτρα.
\end{exercise}
\begin{lstlisting}
>>> 50/1000 *  100
5
\end{lstlisting}
Άρα 5\%.

\begin{lstlisting}
>>> 30/360 * 100
8.333333333333332
\end{lstlisting}
Άρα 8.33\%.

\begin{lstlisting}
>>> 50/2500 * 100
2.0
\end{lstlisting}
Άρα 2\%.

Παλάμη λέμε το δεκατόμετρο dm οπότε οι 3 παλάμες είναι 3 dm δηλαδή 30cm.
Οπότε:
\begin{lstlisting}
>>> 30 / (10*100) * 100
3
\end{lstlisting}
Άρα 3\%.

\begin{exercise}
\sel[6]{81}
Ένα  μπουκάλι    με  οινόπνευμα  παρέμεινε   ανοικτό και εξατμίστηκε το  22\% του όγκου   
του.    Το  μπουκάλι    περιείχε    αρχικά  0,610   lt. Πόσα    lt  οινοπνεύματος   εξατμίστηκαν;
\end{exercise}

\begin{lstlisting}
>>> 22*0.610 / 100
0.13419999999999999
\end{lstlisting}
Οπότε η Python δίνει μια προσέγγιση της σωστής απάντησης που είναι:
0.1342.

\begin{exercise}
Σε  ένα σημείο  της γήινης  σφαίρας,    ο   φλοιός  έχει    πάχος   50  Km, ο   
μανδύας 2.900   Km  και ο   πυρήνας 3.450   Km. (α) Να  βρεις   το  μήκος   
της ακτίνας της Γης σε  Km. (β) Να  βρεις   ποιο    ποσοστό της ακτίνας 
της Γης κατέχει ο   φλοιός, ο   μανδύας και ο   πυρήνας αντίστοιχα.
\end{exercise}
\begin{lstlisting}
x = [50,2900,3450]
print(sum(x))
for i in x:
    print(100*i/sum(x))
\end{lstlisting}
Το αποτέλεσμα του προγράμματος είναι:
\begin{lstlisting}
6400
0.78125
45.3125
53.90625
\end{lstlisting}
Οπότε το μήκος της ακτίνας της γης είναι 6400Km.
Ο φλοιός είναι το 0,78125\%, o μανδύας το 45,3125\% και ο πυρήνας το 53,90625\%.


\begin{exercise}
\sel{82}
Ένας ηλεκτρολόγος είχε έσοδα 2.856€ το δεύτερο τρίμηνο του έτους. Πόσα χρήματα
πρέπει να αποδώσει στο κράτος, αν ο Φ.Π.Α. που παρακρατά από τους πελάτες του
είναι 19\%.
\end{exercise}
Η σωστή απάντηση είναι:
\begin{lstlisting}
>>> 2856*19/119
456.0
\end{lstlisting}

\begin{exercise}
Στην περίοδο των εκπτώσεων, ένα κατάστημα έκανε έκπτωση 35\% στα είδη ρουχισμού
και 15\% στα παπούτσια. Πόσο θα πληρώσουμε για ένα πουκάμισο και ένα ζευγάρι
παπούτσια που κόστιζαν 58€ και 170€, αντίστοιχα, πριν τις εκπτώσεις.
\end{exercise}

\begin{lstlisting}
>>> 170 * 15/100
25.5
>>> 170 - 25.5
144.5
>>> 58*35/100
20.3
>>> 58-20.3
37.7
>>> 37.7 + 144.5
182.2
\end{lstlisting}
Μπορείς να κάνεις τις πράξεις αυτές σε μία συνάρτηση neatimi:
\begin{lstlisting}
def neatimi(timi,ekpt):
    return(timi-timi*ekpt/100)

neatimi(170,15)
neatimi(58,35)
\end{lstlisting}
Που δίνει σαν αποτέλεσμα:
\begin{lstlisting}
144.5
37.7
\end{lstlisting}
\begin{exercise}
\sel{82}
Ποσό 1.000€ κατατέθηκε σε λογαριασμό ταμιευτηρίου, με επιτόκιο 5\%. Πόσος είναι
ο τόκος που θα αποδώσει το κεφάλαιο αυτό, μετά από 18 μήνες, αν οι τόκοι
προστίθενται στο κεφάλαιο κάθε χρόνο;
\end{exercise}
Στον ένα χρόνο:
\begin{lstlisting}
>>> 1000*5/100
50.0
\end{lstlisting}
Για τους υπόλοιπους έξι μήνες θα είναι τα μισά οπότε:
\begin{lstlisting}
>>> 50.0/2
25.0
\end{lstlisting}
Συνολικά είναι:
\begin{lstlisting}
>>> 50.0 + 25.0
75.0
\end{lstlisting}
Σαν συνάρτηση γίνεται:
\begin{lstlisting}
def tokos(kefalaio,epitokio,mines):
    return(kefalaio*epitokio/ 100*mines/12)

print(tokos(1000,5,18))
\end{lstlisting}
Που δίνει το ίδιο αποτέλεσμα:
\begin{lstlisting}
75
\end{lstlisting}
\begin{exercise}
\sel[1]{82}
Επιχειρηματίας αγόρασε μετοχές μιας εταιρείας, προς 50€ την κάθε μετοχή. Σε ένα
μήνα η μετοχή έπεσε κατά 8\% και το επόμενο δίμηνο ανέβηκε κατά 5\% το μήνα.
(α) Ποια ήταν η τιμή της μετοχής στο τέλος του τρίτου μήνα; (β) Η επένδυση του
επιχειρηματία ήταν κερδοφόρα ή όχι; (γ) Ποιο είναι το ποσοστό κέρδους ή ζημίας του,
επί του αρχικού κεφαλαίου;
\end{exercise}
\begin{lstlisting}
>>> 50 - 8/100*50
46.0
>>> 46+5/100*46
48.3
>>> 48.3+5/100*48.3
50.714999999999996
\end{lstlisting}

Η τιμή της μετοχής είναι 50,715.

Η επένδυση ήταν κερδοφόρα.

Το ποσοστό κέρδους είναι:
\begin{lstlisting}
>>> (50.715 - 50)/50*100
1.4300000000000068
\end{lstlisting}

Άρα το αποτέλεσμα είναι 1,43\%.

\begin{exercise}
\sel[2]{82}
Κεφάλαιο 80.000€ κατατέθηκε, σε λογαριασμό ταμιευτηρίου, με επιτόκιο 4,5\% το χρόνο.
(α) Ποιος θα είναι ο τόκος στο τέλος του πρώτου έτους; (β) Ποιος θα είναι ο τόκος
στο τέλος του δεύτερου έτους, αν ο τόκος του πρώτου έτους κεφαλοποιηθεί;
\end{exercise}

\begin{lstlisting}
>>> 80000*4.5/100
3600
>>> 80000+3600
83600.0
>>> 83600*4.5/100
3762.0
\end{lstlisting}

\begin{exercise}
\sel[3]{82}
Ένα καινούριο αυτοκίνητο κόστιζε 20.000€. Το αγόρασε κάποιος και μετά από 1
χρόνο ήθελε να το πουλήσει, κατά 30\% λιγότερο, από όσο το αγόρασε. Ο υποψήφιος
αγοραστής έμαθε, ότι το ίδιο ακριβώς μοντέλο, καινούριο, κόστιζε 25.000€. (α) Σε
ποια τιμή θα αγόραζε το μεταχειρισμένο αυτοκίνητο; (β) Τι ποσοστό της τιμής του
καινούριου αυτοκινήτου είναι η τιμή του μεταχειρισμένου; (γ) Αν ένα μαγαζί που πουλάει
μεταχειρισμένα αυτοκίνητα δίνει το ίδιο μοντέλο σε τιμή 40\% φτηνότερα από την
τρέχουσα τιμή του καινούριου, από ποιον συμφέρει να αγοράσει το μεταχειρισμένο
αυτοκίνητο ο υποψήφιος αγοραστής;
\end{exercise}

\begin{lstlisting}
>>> 20000-20000*30/100
14000
\end{lstlisting}
Το αυτοκίνητο το πουλάει 14.000€.

\begin{lstlisting}
>>> pososto(14000/25000)
56.00%
\end{lstlisting}

Είναι το 56\%.

\begin{lstlisting}
>>> 25000-25000*40/100
15000
\end{lstlisting}

Άρα το μεταχειρισμένο είνα φτηνότερο.

\begin{exercise}
\sel[4]{82}
Σε ένα προϊόν, έγινε η προσφορά που φαίνεται στην πινακίδα. Στη
συσκευασία του προϊόντος υπήρχε σημειωμένη η συγκεκριμένη, για το
είδος προσφορά, δηλαδή για κάθε 300 κ.εκ., πρόσθεσαν άλλα 100 κ.εκ.
(α) Σύμφωνα, με όσα διαβάζεις, θεωρείς ότι αληθεύουν όσα γράφονται
στην προσφορά; (β) Σε ποια περίπτωση η εταιρεία θα πρόσφερε,
πράγματι, το 50\% του προϊόντος ΔΩΡΕΑΝ;
\end{exercise}

\begin{lstlisting}
>>> pososto(100/300)
33.33%
\end{lstlisting}
Άρα δεν ισχύει. Το 50\% του 300 είναι:
\begin{lstlisting}
>>> 50/100*300
150
\end{lstlisting}
150κ.εκ.
\begin{exercise}
\sel[5]{82}
Τι κεφάλαιο πρέπει να καταθέσουμε στην τράπεζα, για να πάρουμε στο
τέλος ενός έτους 1.000€, αν το επιτόκιο είναι 2\%;
\end{exercise}
$$ x + x*2\% = 1000$$
\begin{lstlisting}
>>> solve(x+x*2/100 - 1000)
[50000/51]
>>> 50000/51
980.3921568627451
\end{lstlisting}
Δηλαδή αν βάλει $980,39$€ θα έχει:
\begin{lstlisting}
>>> 980.39+980.39*2/100
999.9978
\end{lstlisting}

%\chapter{Ανάλογα ποσά - Αντιστρόφως ανάλογα ποσά}
\begin{exercise}
Να σχεδιάσεις ένα ορθοκανονικό σύστημα ημιαξόνων, με μονάδα το 1 cm και να
τοποθετήσεις τα σημεία Α(2,3), Β(3,2), Γ(4,5), Δ(5,5), Ε(1,4), Z(7,3), Η(7,2), Θ(6,2),
Ι(6,0), Κ(0,5). Τι παρατηρείς για τα σημεία Ι και Κ; Πού βρίσκονται αυτά; Μπορείς να
γενικεύσεις τις παρατηρήσεις σου για τα σημεία που έχουν τετμημένη ή τεταγμένη το
μηδέν;
\end{exercise}
\begin{lstlisting}
import matplotlib.pyplot as plt

plt.clf()
points = [(2,3), (3,2), (4,5), (5,5), (1,4), (7,3), (7,2), (6,2), (6,0), (0,5)]
pointName = ['Α','Β','Γ','Δ','Δ','Ε','Ζ','Η','Θ','Ι','Κ']
x = [p[0] for p in points]
y = [p[1] for p in points]
color=['m','g','r','b']
plt.grid()
plt.scatter(x,y, s=100 ,marker='o', c=color)
for (i,p) in enumerate(points):
    plt.annotate(pointName[i],(p[0],p[1]))

plt.show()
\end{lstlisting}
\begin{figure}
\includegraphics{graph1.png}
\end{figure}
\begin{exercise}
\sel[2]{89}
Σε ορθοκανονικό σύστημα ημιαξόνων να τοποθετήσεις τα σημεία Α(2,1), Β(1,2), Γ(2,3)
και Δ(3,2). Τι σχήμα είναι το ΑΒΓΔ; Αν τα ευθύγραμμα τμήματα ΑΓ και ΒΔ τέμνονται
στο σημείο Κ, ποιες είναι οι συντεταγμένες του Κ;
\end{exercise}
\begin{lstlisting}
import matplotlib.pyplot as plt

plt.clf()
points = [(2,1), (1,2), (2,3), (3,2)]
pointName = ['Α','Β','Γ','Δ']
x = [p[0] for p in points]
y = [p[1] for p in points]
color=['m','g','r','b']
plt.grid()
plt.scatter(x,y, s=100 ,marker='o', c=color)
for (i,p) in enumerate(points):
    plt.annotate(pointName[i],(p[0],p[1]))

x = [points[0][0],points[2][0]]
y = [points[0][1],points[2][1]]
plt.plot(x,y)
x = [points[1][0],points[3][0]]
y = [points[3][1],points[3][1]]
plt.plot(x,y)

plt.show()
\end{lstlisting}
\begin{figure}
\includegraphics{graph2.png}
\end{figure}

\begin{exercise}
\sel[3]{89}
Γράψε πέντε διατεταγμένα ζεύγη σημείων, των οποίων η τετμημένη τους είναι ίση με
την τεταγμένη τους. Μπορείς να τα
τοποθετήσεις, σε ένα ορθοκανονικό
σύστημα ημιαξόνων; Τι παρατηρείς;
\end{exercise}
\begin{lstlisting}
import matplotlib.pyplot as plt

plt.clf()
points = [(1,1), (2,2), (5,5), (10,10), (15,15)]
pointName = ['Α','Β','Γ','Δ','Ε']
x = [p[0] for p in points]
y = [p[1] for p in points]
color=['m','g','r','b']
plt.grid()
plt.scatter(x,y, s=100 ,marker='o', c=color)
for (i,p) in enumerate(points):
    plt.annotate(pointName[i],(p[0],p[1]))

plt.show()
\end{lstlisting}
\begin{figure}
\includegraphics{graph3.png}
\end{figure}

\begin{exercise}
\sel{90}
Συμπλήρωσε τον παρακάτω πίνακα:
\begin{table}
\begin{tabular}{|l|c|c|c|}
Πλευρά τετραγώνου& 1,5 cm& 4 cm& 4,5 cm\\\hline
Περίμετρος τετραγώνου&&&\\\hline
\end{tabular}
\end{table}
\begin{itemize}
\item Εξήγησε πώς προκύπτουν οι αριθμοί της δεύτερης σειράς.
\item Βρες για κάθε τετράγωνο το κλάσμα πλευρά προς περίμετρο.
\item Ποιο είναι το συμπέρασμα που βγάζεις;
\end{itemize}
\end{exercise}
\begin{lstlisting}
>>> 4*1.5
6.0
>>> 4*4
16
>>> 4*4.5
18.0
\end{lstlisting}
\begin{tabular}{|l|c|c|c|}
Πλευρά τετραγώνου& 1,5 cm& 4 cm& 4,5 cm\\\hline
Περίμετρος τετραγώνου&6&16&18\\\hline
\end{tabular}
Θυμηθείτε το ποσοστό σε κλάσμα:
\begin{lstlisting}
def posostoseklasma(fx):
    fx = float(fx)
    denom = 100
    while int(fx) != fx:
         fx *= 10
         denom *= 10
    fx = int(fx)
    return(Fraction(fx,denom))
\end{lstlisting}
Το fx είναι είναι ο αριθμητής ενός κλάσματος με παρονομαστή 100. Εδώ δεν θα υπάρχει ο παρονομαστής 100 οπότε έχουμε denom = 1.
\begin{lstlisting}
def dekadikosseklasma(fx):
    fx = float(fx)
    denom = 1
    while int(fx) != fx:
         fx *= 10
         denom *= 10
    fx = int(fx)
    return(Fraction(fx,denom))

dekadikosseklasma(1.5/6)
dekadikosseklasma(4/16)
dekadikosseklasma(4.5/18)
\end{lstlisting}
και το αποτέλεσμα είναι:
\begin{lstlisting}
>>> dekadikosseklasma(1.5/6)
Fraction(1, 4)
>>> dekadikosseklasma(4/16)
Fraction(1, 4)
>>> dekadikosseklasma(4.5/18)
Fraction(1, 4)
\end{lstlisting}
Άρα παντού το κλάσμα είναι $\frac{1}{4}$.
\begin{exercise}
\sel{90}
Χρησιμοποιούμε τη φωτογραφική μηχανή
για να απεικονίσουμε εικόνες αντικειμένων. Οι εικόνες αυτές δείχνουν τα
πραγματικά αντικείμενα σε σμίκρυνση.
Στη φωτογραφία το ύψος ενός παιδιού
είναι 2 cm ενώ γνωρίζουμε ότι το πραγματικό του ύψος είναι 1,65 m = 165 cm. Πόση θα είναι τότε η σμίκρυνσή του
στη φωτογραφία;
\end{exercise}
\begin{lstlisting}
>>> 2/165
0.012121212121212121
\end{lstlisting}

\begin{exercise}
\sel{91}
Μετρούμε μια απόσταση, σε χάρτη, με κλίμακα 1:10.000.000 και τη βρίσκουμε
ίση με 2,4 cm. Ποια είναι η πραγματική απόσταση των δύο σημείων;
\end{exercise}
\begin{lstlisting}
>>> x = 2.4*10000000
>>> x
24000000
>>> x = x/100
>>> x
240000
>>> x = x/1000
>>> x
240
\end{lstlisting}
240Km
\begin{exercise}
\sel[3]{92}
Σε μια φωτογραφία το ύψος ενός ανθρώπου είναι 4 cm, ενώ το
πραγματικό το ύψος είναι 1,76 m. Πόσο έχει σμικρυνθεί η εικόνα του
ανθρώπου στη φωτογραφία;
\end{exercise}
\begin{lstlisting}
def pososto(x):
    print(str(round(x*100,2))+'%')
\end{lstlisting}
Αν θυμηθούμε τη συνάρτηση pososto τότε
\begin{lstlisting}
>>> pososto(4/176)
2.27%
\end{lstlisting}
\begin{exercise}
\sel[4]{92}Ένας προβολέας διαφανειών προβάλλει το κείμενο μιας διαφάνειας στον απέναντι
τοίχο. Αν ένα ``A'' έχει ύψος 7 mm στη διαφάνεια και 4,2 cm στον τοίχο, ποια είναι η
μεγέθυνση που δίνει ο προβολέας
\end{exercise}
\begin{lstlisting}
>>> pososto(4.2/0.7)
600%
\end{lstlisting}
\begin{exercise}
\sel[5]{92}
Η σύνθεση μιας μπλούζας είναι 80\% βαμβάκι και το υπόλοιπο πολυεστέρας. Aν η μπλούζα
ζυγίζει 820 gr, πόσα γραμμάρια ζυγίζουν τα νήματα του πολυεστέρα που περιέχει;
\end{exercise}
\begin{lstlisting}
>>> 820*20/100
164.0
\end{lstlisting}
\begin{exercise}
\sel[6]{92}
Να συμπληρωθεί ο πίνακας
\begin{table}
\begin{tabular}{|c|c|c|c|c|c|}
\hline
Κλίμακα&1:5&3:8&1:30&&1:100\\\hline
Μήκος σε σχέδιο&4cm &&12cm&2cm&3,5cm\\\hline
Πραγματικό ύψος&&24m&&10m&\\\hline
\end{tabular}
\end{table}
\end{exercise}
\begin{lstlisting}
>>> from fractions import Fraction
>>> 5*4
20
>>> 3/8*24
9.0
>>> 12*30
360
>>> Fraction(2,1000)
Fraction(1, 500)
>>> 3.5*100
350
\end{lstlisting}
Άρα ο πίνακας γίνεται:
\begin{table}
\begin{tabular}{|c|c|c|c|c|c|}
\hline
Κλίμακα&1:5&3:8&1:30&1:500&1:100\\\hline
Μήκος σε σχέδιο&4cm &9cm&12cm&2cm&3,5cm\\\hline
Πραγματικό ύψος&20cm&24m&360cm&10m&350cm\\\hline
\end{tabular}
\end{table}
\begin{exercise}
\sel[7]{92}
Οι διαστάσεις ενός ορθογωνίου παραλληλογράμμου είναι $x+2$ και $x$.

(α) Να γράψεις τη σχέση που συνδέει την περίμετρο Π του ορθογωνίου με το x.

(β) Να συμπληρώσεις τον πίνακα:
\begin{table}
\begin{tabular}{|c|c|c|c|c|}
x&&2&&4\\\hline
Π&8&&16&\\\hline
\end{tabular}
\end{table}
\end{exercise}
α)
\begin{lstlisting}
>>> from sympy import *
>>> x = symbols('x')
>>> p = x+x+2+x+x+2
>>> p
4*x + 4
\end{lstlisting}
β)
\begin{lstlisting}
>>> solve(p-8)
[1]
>>> p.subs(x,2)
12
>>> solve(p-16)
[3]
>>> p.subs(x,4)
20
\end{lstlisting}
και ο πίνακας γίνεται:
\begin{table}
\begin{tabular}{|c|c|c|c|c|}
\hline
x&1 &2  &3  &4\\\hline
Π&8&12&16&20\\\hline
\end{tabular}
\end{table}
\begin{exercise}
\sel[8]{92}
Aν οι διαστάσεις ενός δωματίου, σε ένα σχέδιο με κλίμακα 1:250, είναι 3x5, οι
πραγματικές διαστάσεις του δωματίου θα είναι .....x..... .
\end{exercise}
\begin{lstlisting}
>>> 3*250
750
>>> 5*250
1250
\end{lstlisting}
Οπότε το δωμάτιο είναι 7,5m x 12,5m αν οι διαστάσεις ήταν σε cm.
\begin{exercise}
\sel[9]{92}
Αν ανακατέψουμε 2 κιλά κόκκινο χρώμα και 3 κιλά κίτρινο χρώμα,
φτιάχνουμε μια συγκεκριμένη απόχρωση του πορτοκαλί. Αν
ανακατέψεις 5 κιλά κόκκινο χρώμα και 6 κιλά κίτρινο, θα πάρεις
την ίδια απόχρωση; Δικαιολόγησε την απάντησή σου.
\end{exercise}
\begin{lstlisting}
>>> 3/2 == 6/5
False
\end{lstlisting}
Όχι δεν είναι η ίδια απόχρωση.
\begin{exercise}
\sel[2]{93} Όταν ο Κώστας έκλεισε τα δώδεκα χρόνια είχε το ένα τρίτο της
ηλικίας της μητέρας του. Όταν θα γίνει είκοσι χρόνων, ο λόγος των
δύο ηλικιών τους θα παραμείνει ο ίδιος;
\end{exercise}
\begin{lstlisting}
>>> ilikiaMiteras = 3*12
>>> ilikiaMiteras
36
>>> xronia = 20-12
>>> xronia
8
>>> neailikiaMiteras = ilikiaMiteras + xronia
>>> neailikiaMiteras = 44
>>> 44/20 == 36/12
False
\end{lstlisting}
Άρα όχι.
\begin{exercise}
Να συμπληρωθεί ο πίνακας, αν γνωρίζουμε ότι τα ποσά x και 􀁜 είναι ανάλογα, με
συντελεστή αναλογίας $\alpha = \frac{2}{3}$.
\begin{table}
\begin{tabular}{|c|c|c|c|c|c|c}
\hline
x &0 &1 &0,3& &\\\hline
y &    &  &       & $\frac{5}{3}$ & 3\\\hline
\end{tabular}
\end{table}
\end{exercise}
$$
y = \frac{2}{3}x
$$
\begin{lstlisting}
>>> from sympy import *
>>> (x,y) = symbols('x y')
>>> e = 2/3*x
>>> e.subs(x,0)
0
>>> e.subs(x,1)
0.666666666666667
>>> e.subs(x,0.3)
0.200000000000000
>>> solve(e-5/3)
[2.50000000000000]
>>> solve(e-3)
[4.50000000000000]
\end{lstlisting}
Και ο πίνακας γίνεται:
\begin{table}
\begin{tabular}{|c|c|c|c|c|c|c}
\hline
x &0 &1 &0,3& 2,5&4,5\\\hline
y & 0   & 0,6666 & 0,2      & $\frac{5}{3}$ & 3\\\hline
\end{tabular}
\end{table}
\begin{exercise}
\sel[2]{92}
Σε ένα διάλυμα ζάχαρης η περιεκτικότητα σε ζάχαρη είναι 23\%. Πόσα γραμμάρια
ζάχαρης υπάρχουν σε 300 gr διαλύματος;
\end{exercise}
\begin{lstlisting}
>>> 300*23/100
69.0
\end{lstlisting}
\begin{exercise}
\sel[3]{97}
Ένα πλοίο έχει σταθερή ταχύτητα και καλύπτει απόσταση 80 Km σε 2 ώρες. Σε πόσο
χρόνο θα καλύψει απόσταση 2.000 Km;
\end{exercise}
$$\frac{2}{80}=\frac{x}{2000}$$
\begin{lstlisting}
>>> from sympy import *
>>> x = symbols('x')
>>> solve(2/80-x/2000)
[50.0000000000000]
\end{lstlisting}
Η απάντηση είναι 50 ώρες.
\begin{exercise}
Εξέτασε αν τα ποσά που δίνονται στους παρακάτω πίνακες είναι ανάλογα:
(α) 
\begin{table}
\begin{tabular}{|c|c|c|c|}
\hline
x&3&5 &7\\\hline
y&8&10&12\\\hline
\end{tabular}
\end{table}
(β)
\begin{table}
\begin{tabular}{|c|c|c|c|c|}
\hline
x&3&4 &6&11\\\hline
y&0,9&1,2&1,8&3,3\\\hline
\end{tabular}
\end{table}
\end{exercise}
\begin{lstlisting}
>>> 8/3==10/5==12/7
False
>>> 0.9/3==1.2/4==1.8/6==3.3/11
True
\end{lstlisting}
\begin{exercise}
\sel[4]{98}
Στον πίνακα που ακολουθεί, τα ποσά x και y είναι ανάλογα. Υπολόγισε τον συντελεστή
αναλογίας τους και συμπλήρωσε τον πίνακα.
\begin{table}
\begin{tabular}{|c|c|c|c|c|c|c|c|c|}
x& 5& 0& 1& & & 3,7& 0,61&\\\hline
y&10,05& & &2 &0,125&&& 0,55\\hline
\end{tabular}
\end{table}
\end{exercise}
Η αναλογία είναι 
\begin{lstlisting}
>>> 10.05/5
2.0100000000000002
\end{lstlisting}
Όμως αυτό είναι 2,01
Οπότε:
\begin{lstlisting}
>>> 0*2.01
0.0
>>> 1*2.01
2.01
>>> 2/2.01
0.9950248756218907
>>> 0.125/2.01
0.06218905472636817
>>> 3.7*2.01
7.436999999999999
>>> 0.61*2.01
1.2260999999999997
>>> 0.55/2.01
0.27363184079601993
\end{lstlisting}
και προσεγγιστικά ο πίνακας γίνεται:
\begin{table}
\begin{tabular}{|c|c|c|c|c|c|c|c|c|}
x& 5       & 0& 1      &0,995 & 0,0622 & 3,7   & 0,61&  0,273632\\\hline
y&10,05& 0 & 2.01&2         &0,125      &7,437& 1.226&0,55\\hline
\end{tabular}
\end{table}

\section{Ανάλογα ποσά}
\begin{exercise}
Σε	μια	παρέα	κάποιος	υποστήριζε	ότι	το	βάρος	του	ανθρώπου	είναι	ανάλογο	του	ύψους	του. Μετρήθηκαν,	λοιπόν,	όλοι	και	έβαλαν	στον	παρακάτω	πίνακα	τα	αποτελέσματα σε Κ.
\begin{table}
\begin{tabular}{|c|c|c|c|c|}
\hline
Βάρος& 58& 71& 56& 68\\\hline
Ύψος & 1,60&1,65&1,62&1,72\\\hline
\end{tabular}
\end{table}
\begin{itemize}
\item Μπορείς	να	επιβεβαιώσεις	ή	να	απορρίψεις	τον		ισχυρισμό	αυτό;		
\item Πώς	δικαιολογείς	το	συμπέρασμά	σου;
\end{itemize}
\end{exercise}
\begin{lstlisting}
>>> 58/1.60 == 71/1.65
False
\end{lstlisting}
Οπότε ο ισχυρισμός απορρίπτεται.
\begin{exercise}
O	μανάβης	πουλάει	τα	καρπούζια	προς	0,4	Q
	το	κιλό.	Μέσα	σε	μια	ημέρα	πούλησε	11	καρπούζια	που	ζύγιζαν	100	κιλά	συνολικά.	Ο	μανάβης	έγραφε,	σ’	ένα	χαρτί,	τα	λεφτά	που	 εισέπραττε	κάθε	φορά.	Ξέχασε,	όμως,	μία	φορά	να	το	σημειώσει.

→		Μπορείς	να	τον	βοηθήσεις	συμπληρώνοντας 	τα	κενά	του	παρακάτω	πίνακα: 
\begin{table}
\begin{tabular}{|c|c|c|c|c|c|c|c|c|c|c|c|}
\hline
Tιμή& 6€ & 2,8€ & 5,2€ & 3,2€ & & 3,6€ & 4,8€ & 2,4€ & 1,6€ & 4,4€ & 2€\\\hline
Κιλά&       &         &          &         &&            &          &          &        &            &     \\\hline
\end{tabular}
\end{table}    
\begin{itemize}
 \item Δικαιολόγησε τα	αποτελέσματα	των	πράξεων	που 	έκανες	και	προσπάθησε	να		διατυπώσεις	έναν	γενικό	κανόνα. 
\end{itemize}

 \end{exercise}
 Τα χρήματα που πήρε συνολικά θα είναι $0,4*100=40$€. Οπότε μπορούμε να αθροίσουμε τα χρήματα και να βρούμε τα κιλά που πωλήθηκαν από τα χρήματα.
 \begin{lstlisting}
 >>> 6+2.8+5.2+3.2+3.6+4.8+2.4+1.6+4.4+2
36.0
>>> 36/0.4
90.0
>>> 100-90
10
>>> 10*0.4
4.0
\end{lstlisting}
Για να συμπληρώσουμε ολόκληρο τον πίνακα μπορούμε να βρούμε τα κιλά από τα χρήματα διαιρώντας με το 0,4.
\begin{lstlisting}
>>> 6/0.4
15.0
>>> 2.8/0.4
6.999999999999999
>>> 5.2/0.4
13.0
>>> 3.6/0.4
9.0
>>> 4.8/0.4
11.999999999999998
>>> 2.4/0.4
5.999999999999999
>>> 1.6/0.4
4.0
>>> 4.4/0.4
11.0
>>> 2/0.4
5.0
>>>
\end{lstlisting}
Αν λάβουμε υπόψη τις στρογγυλοποιήσεις ο πίνακας γίνεται:
\begin{table}
\begin{tabular}{|c|c|c|c|c|c|c|c|c|c|c|c|}
\hline
Tιμή& 6€ & 2,8€ & 5,2€ & 3,2€ & 4€ & 3,6€ & 4,8€ & 2,4€ & 1,6€ & 4,4€ & 2€\\\hline
Κιλά&  15 &  7    &   13    &   8      & 10& 9      &  12    &   6     &  4    &   1       &  5   \\\hline
\end{tabular}
\end{table}    
\begin{exercise}
\sel{99}
Η	σχέση,	μεταξύ	δύο	ανάλογων	ποσών	x	και		με	συντελεστή	αναλογίας	α	=	3,	δίνεται	από	τον	τύπο:		
$$ y =	3 \cdot x$$.
\begin{itemize}
\item Συμπλήρωσε	τα	κενά	του	πίνακα	και	με	άλλες	τιμές	των	αναλόγων	ποσών	x	και	.
\item Βρες	τα	σημεία	του	επιπέδου	που		αναπαριστούν	τα	παραπάνω		ζεύγη	τιμών.
\item Προσπάθησε	να	διαπιστώσεις,	εάν		τα	σημεία	ανήκουν	σε	μία	ημιευθεία		ή	όχι.	
\item Η	ημιευθεία	αυτή	περνάει	από		το	σημείο	Ο(0,0)	δηλαδή	την	αρχή		των	ημιαξόνων;
\end{itemize}
\end{exercise}
\begin{lstlisting}
import matplotlib.pyplot as plt
from random import randint
from math import floor
plt.clf()
points = []
for i in range(10):
    x = 0+randint(0,10)*0.5
    y = 3*x
    points.append((x,y))

x = [p[0] for p in points]
y = [p[1] for p in points]
color=['m','g','r','b']
plt.grid()
plt.scatter(x,y, s=100 ,marker='o', c=color)

plt.show()
\end{lstlisting}
\begin{figure}
\includegraphics{3x.png}
\end{figure}
Οπότε τα σημεία ανήκουν σε ημιευθεία η οποία περνάει από την αρχή των αξόνων.

\begin{exercise}
\sel{100}
Δίνονται	οι	 πίνακες	Α,	Β,	 Γ	και	Δ.	

(α)	Να	γίνει	η	γραφική	απεικόνιση	των	ζευγών	(x,y)	των	πινάκων	στο επίπεδο	και	
(β)	να	διαπιστωθεί	σε	ποια	περίπτωση	αυτά	παριστάνουν	ποσά	ανάλογα.   

\begin{table}
\begin{tabular}{|c|c|c|c|c|}
x&0&1&2&3\\
y&0&2&1&1.5\\
\end{tabular}
\caption{Πίνακας Α}
\end{table}

\begin{table}
\begin{tabular}{|c|c|c|c|c|}
x&0&1&2&3\\
y&1&1.5&2&2.5\\
\end{tabular}
\caption{Πίνακας B}
\end{table}

\begin{table}
\begin{tabular}{|c|c|c|c|c|}
x&0&1&2&3\\
y&0&1&2&3\\
\end{tabular}
\caption{Πίνακας Γ}
\end{table}


\begin{table}
\begin{tabular}{|c|c|c|c|c|}
x&0&1&2&3\\
y&0&0.5&1&1.5\\
\end{tabular}
\caption{Πίνακας Δ}
\end{table}
\end{exercise}
\begin{lstlisting}
import matplotlib.pyplot as plt

plt.clf()
pointsA = [(0,0),(1,2),(2,1),(3,1.5)]
pointsB = [(0,1),(1,1.5),(2,2),(3,2.5)]
pointsC = [(0,0),(1,1),(2,2),(3,3)]
pointsD = [(0,0),(1,0.5),(2,1),(3,1.5)]

x = [p[0] for p in pointsA]
y = [p[1] for p in pointsA]
plt.grid()
plt.plot(x,y, marker='o', c='r')


x = [p[0] for p in pointsB]
y = [p[1] for p in pointsB]
plt.grid()
plt.plot(x,y, marker='o', c='g')


x = [p[0] for p in pointsC]
y = [p[1] for p in pointsC]
plt.grid()
plt.plot(x,y,marker='o', c='b')


x = [p[0] for p in pointsD]
y = [p[1] for p in pointsD]
plt.grid()
plt.plot(x,y, marker='o', c='m')
plt.show()
\end{lstlisting}

\begin{figure}
\includegraphics{4plots.png}
\end{figure}

Μια πιο σύντομη έκδοση του προγράμματος που δίνει το ίδιο αποτέλεσμα είναι:
\begin{lstlisting}
import matplotlib.pyplot as plt

plt.clf()
pointslist = [[(0,0),(1,2),(2,1),(3,1.5)],
          [(0,1),(1,1.5),(2,2),(3,2.5)],
          [(0,0),(1,1),(2,2),(3,3)],
          [(0,0),(1,0.5),(2,1),(3,1.5)]]
colors = ['r','g','b','m']
for (i,points) in enumerate(pointslist):
    x = [p[0] for p in points]
    y = [p[1] for p in points]
    plt.grid()
    plt.plot(x,y, marker='o', c=colors[i])

plt.show()
\end{lstlisting}

\begin{exercise}
\sel[1]{101}
Δύο  ποσά  x και  είναι ανάλογα, με συντελεστή αναλογίας α = 1,5. 

(α)	Δημιούργησε έναν πίνακα τιμών των δύο ποσών, ο οποίος να περιέχει  τουλάχιστον δύο  ζεύγη τιμών. 

(β) Βρες τα σημεία που αναπαριστούν τα ζεύγη τιμών του πίνακά σου. 

(γ)  Σχεδίασε τη γραφική παράσταση της σχέσης αναλογίας των ποσών x και , σε  ένα ορθοκανονικό σύστημα ημιαξόνων.
\end{exercise}
\begin{table}
\begin{tabular}{|c|c|c|}
\hline
x&1&2\\\hline
y&1.5&3\\\hline
\end{tabular}
\end{table}
\begin{lstlisting}
import matplotlib.pyplot as plt

plt.clf()
points = [(1,1.5),(2,3)]

x = [p[0] for p in points]
y = [p[1] for p in points]

plt.grid()
plt.plot(x,y, marker='o', c='r')

plt.show()
\end{lstlisting}
\begin{figure}
\includegraphics{sel101_1.png}
\end{figure}
Καλύτερα όμως είναι να βάλουμε και το σημείο (0,0) ως εξής:
\begin{lstlisting}
import matplotlib.pyplot as plt

plt.clf()
points = [(0,0),(1,1.5),(2,3)]

x = [p[0] for p in points]
y = [p[1] for p in points]

plt.grid()
plt.plot(x,y, marker='o', c='r')

plt.show()
\end{lstlisting}
\begin{figure}
\includegraphics{sel101_1a.png}
\end{figure}

\begin{exercise}
\sel[2]{101}
Σε κατάλληλο ορθογώνιο σύστημα ημιαξόνων να σχεδιάσεις τις γραφικές παραστάσεις για κάθε μία από τις ακόλουθες σχέσεις αναλογίας: 

(α)	$y = \left(\frac{1}{2}\right)\cdot x$,	

(β)	$y = 3 \cdot x$,	

(γ)	$y =	5,5	\cdot x$

(δ)	$y =	10\cdot x$,	

(ε) $y  =	0,01 \cdot x$.
\end{exercise}

Μπορούμε να τις σχεδιάσουμε και όλες μαζί με διαφορετικά χρώματα. Έχουν ένα κοινό σημείο ενώ μπορούμε να υπολογίσουμε εύκολα ένα δεύτερο, π.χ. για $x=10$.
\begin{lstlisting}
import matplotlib.pyplot as plt
from sympy import *
x = symbols('x')

analogies = [1/2*x,3*x,5.5*x,10*x,0.01*x]
colors = ['r','g','b','c','m']
for (i,s) in enumerate(analogies):
    x = symbols('x')
    points = [(0,0),(10,s.subs(x,10))]
    x = [p[0] for p in points]
    y = [p[1] for p in points]
    plt.grid()
    plt.plot(x,y, marker='o', c=colors[i])

plt.show()
\end{lstlisting}
\begin{figure}
\includegraphics{sel101_2.png}
\end{figure}
Όμως τα συστήματα δεν είναι κατάλληλα για όλες τις γραφικές παραστάσεις ειδικά η τελευταία φαίνεται να είναι σταθερή στο 0. Αν τη σχεδιάσουμε μόνη της η matplotlib θα υπολογίσει ένα κατάλληλο σύστημα αξόνων, δες στον άξονα y.
\begin{lstlisting}
import matplotlib.pyplot as plt
from sympy import *

points = [(0,0),(10,0.01*10]
x = [p[0] for p in points]
y = [p[1] for p in points]
plt.grid()
plt.plot(x,y, marker='o', c='m')

plt.show()
\end{lstlisting}
\begin{figure}
\includegraphics{sel101_2a.png}
\end{figure}

\begin{exercise}
Αντιστοίχισε	κάθε	πίνακα	με	ένα	από	τους	προτεινόμενους	τύπους:
\begin{table}
\begin{tabular}{cccc}
(A) & 
\begin{tabular}{|c|c|c|c|}
\hline
x&4&7&12\\\hline
y&10&17,5&30\\\hline
\end{tabular}
&(1)&$y=2x+3$\\
(B) & 
\begin{tabular}{|c|c|c|c|}
\hline
x&5&7,5&9\\\hline
y&11&16&19\\\hline
\end{tabular}
&(2)&$y=3x$\\
(Γ) & 
\begin{tabular}{|c|c|c|c|}
\hline
x&2&3&10\\\hline
y&7&9&23\\\hline
\end{tabular}
&(3)&$y=12:x$\\
(Δ) & 
\begin{tabular}{|c|c|c|c|}
\hline
x&2&4&6\\\hline
y&6&3&2\\\hline
\end{tabular}
&(4)&$y=2,5x$\\
(E) & 
\begin{tabular}{|c|c|c|c|}
\hline
x&2&5&0,5\\\hline
y&1&2,5&0,25\\\hline
\end{tabular}
&(5)&$y=2x+2$\\
(Z) & 
\begin{tabular}{|c|c|c|c|}
\hline
x&0,2&6&10\\\hline
y&2,4&14&22\\\hline
\end{tabular}
&(6)&$y=2x+1$\\
(H) & 
\begin{tabular}{|c|c|c|c|}
\hline
x&1&1,2&2,5\\\hline
y&3&3,6&7,5\\\hline
\end{tabular}
&(7)&$y=4x-1$\\
(Θ) & 
\begin{tabular}{|c|c|c|c|}
\hline
x&0,8&1&1,5\\\hline
y&2,2&3&5\\\hline
\end{tabular}
&(8)&$y=0,5x$\\
\end{tabular}
\end{table}
\end{exercise}
\begin{lstlisting}
from sympy import *
x = symbols('x')
pointsList = [[(4,10),(7,17.5),(12,30)],
              [(5,11),(7.5,16),(9,19)],
              [(2,7),(3,9),(10,23)],
              [(2,6),(4,3),(6,2)],
              [(2,1),(5,2.5),(0.5,0.25)],
              [(0.2,2.4),(6,14),(10,22)],
              [(1,3),(1.2,3.6),(2.5,7.5)],
              [(0.8,2.2),(1,3),(1.5,5)]]

typoi = [2*x+3,3*x,12/x,2.5*x,2*x+2,2*x+1,4*x-1,0.5*x]
onomataPinaka = ['Α','Β','Γ','Δ','Ε','Ζ','Η','Θ']

for (arithmosPinaka,points) in enumerate(pointsList):
    for (arithmosTipou, t) in enumerate(typoi):
        for p in points:
            x = symbols('x')
            if abs((p[1] - t.subs(x,p[0]))) > 1e-8:
                break
        else:
            print(onomataPinaka[arithmosPinaka],'-',arithmosTipou+1)
\end{lstlisting}

Που δίνει το αποτέλεσμα:

\begin{lstlisting}
Α - 4
Β - 6
Γ - 1
Δ - 3
Ε - 8
Ζ - 5
Η - 2
Θ - 7
\end{lstlisting}

Χρησιμοποιούμε abs((p[1] - t.subs(x,p[0]))) > 1e-8 αντί για το πιο απλό p[1]!=t.subs(x,p[0]) γιατί στην δεύτερη περίπτωση θα έχουμε σφάλματα απο στρογγυλοποιήσεις.

\begin{exercise}
\sel[4]{101}
Ένας καταστηματάρχης αθλητικών ειδών διαθέτει 12.000€ για να αγοράσει φόρμες γυμναστικής, μαγιό και αθλητικά παπούτσια. Κάθε φόρμα κοστίζει 40€ , κάθε μαγιό 20€ και κάθε ζευγάρι παπούτσια 50€. 

(α)  Να βρεις τις σχέσεις αναλογίας “χρήματα-κομμάτια από κάθε είδος” και να τις  παραστήσεις γραφικά στο ίδιο σύστημα ορθογωνίων αξόνων. 

(β)  Ο καταστηματάρχης αποφάσισε να διαθέσει το ίδιο ποσό, για κάθε είδος. Βρες  πόσα κομμάτια από κάθε είδος θα αγοράσει με τα χρήματα που διαθέτει,   χρησιμοποιώντας μόνο τη γραφική παράσταση των σχέσεων που δημιούργησες  στο πρώτο ερώτημα της άσκησης.
\end{exercise}
$$y = x/40$$
$$y = x/20$$
$$y = x/50$$

Αν δώσει 0 ευρώ τότε το y είναι 0 σε όλες τις γραφικές παραστάσεις. Αν δώσει 200 ευρώ τότε θα πάρει 5 φόρμες ($200/40=5$), 10 μαγιό ($200/40=10$), και 4 ζευγάρια αθλητικά παπούτσια ($200/50=40$).
Οπότε μπορούμε να σχεδιάσουμε τις ευθείες με τα σημεία: (0,0) και (200,5), (200,10), (200,50) αντίστοιχα. 
\begin{lstlisting}
import matplotlib.pyplot as plt

plt.clf()
pointslist = [[(0,0),(200,5)],
                        [(0,0),(200,10)],
                        [(0,0),(200,50]]
colors = ['r','g','b']
for (i,points) in enumerate(pointslist):
    x = [p[0] for p in points]
    y = [p[1] for p in points]
    plt.grid([10,10])
    plt.plot(x,y, marker='o', c=colors[i])

plt.show()
\end{lstlisting}

\begin{figure}
\includegraphics{sel101_4a.png}
\end{figure}
\begin{exercise}
\sel{103}
Για	να	φτιάξουμε	γλυκό	βύσσινο	πρέπει	να	καθαρίσουμε	τα	βύσσινα	από	τα	κουκούτσια.	Αν	καθαρίσουμε	2,5	Kg	βύσσινο,	παίρνουμε	2	Kg	καθαρό	βύσσινο. Αν	καθαρίσουμε	5	Kg	βύσσινο,	τι	ποσότητα	καθαρού	βύσσινου	θα	πάρουμε; 
\end{exercise}
Για αυτή την άσκηση και όλες τις αντίστοιχες μπορούμε να προγραμματίσουμε μια συνάρτηση:
\begin{lstlisting}
def analoga(gnwsto1,gnwsto2,analogo1):
    #analogo2/analogo1  = gnwsto2/gnwsto1
    analogo2 = gnwsto2/gnwsto1*analogo1
    return(analogo2)

print(analoga(2,5,2,5))
\end{lstlisting}
Που δίνει το σωστό αποτέλεσμα
\begin{lstlisting}
4.0
\end{lstlisting}
\begin{exercise}
\sel{104}
Ένας	μεσίτης	αγοράζει	ένα	σπίτι	360.000	€	και	σκοπεύει	να	το	πουλήσει	με	κέρδος	28\%.	Σε	έναν	πελάτη	έκανε	έκπτωση	15\%,	επί	της	τιμής	πώλησης. 

(α)		Πόσο	πουλήθηκε	το	σπίτι	στον	πελάτη	αυτόν; 

(β)		Ποιο	είναι	το	ποσοστό	κέρδους	του	μεσίτη,	για	το		σπίτι	αυτό; 
\end{exercise}
\begin{lstlisting}
>>> arxikiTimi = analoga(100,128,360000)
>>> arxikiTimi
460800.0
>>> telikiTimi = analoga(100,85,460800)
>>> telikiTimi
391680.0
>>> kerdos = (telikiTimi-360000)/360000*100
>>> kerdos
8.799999999999999
>>> 
\end{lstlisting}
Άρα λόγω της έκπτωσης το τελικό κέρδος του μεσίτη ήταν 8.8\%.
\begin{exercise}
\sel[1]{105}
Ένας	πάσσαλος	ύψους	1,2	m	ρίχνει	σκιά	3	m.	Την	ίδια	στιγμή	ένα	δέντρο	ρίχνει	σκιά	14	m.	Αν	γνωρίζουμε	ότι	τα	ποσά	ύψος	-	σκιά	είναι	ανάλογα,	να	βρεθεί	το	ύψος	του	δέντρου.
\end{exercise}
\begin{lstlisting}
>>> analoga(3,1.2,14)
5.6
\end{lstlisting}
\begin{exercise}
\sel{2}[105]
Το	βάρος	στο	φεγγάρι	και	το	βάρος	στη	γη	είναι	ποσά	ανάλογα.	Ένας	αστροναύτης	ζύγιζει	στο	φεγγάρι	12,9	Kg	και	στη	γη	78	Kg.	Πόσο	θα	ζυγίζει	στο	φεγγάρι	ένα	παιδί,	που 	στη	γη	έχει	βάρος	52	Kg;
\end{exercise}
\begin{lstlisting}
>>> analoga(78,12.9,52)
8.6
\end{lstlisting}
\begin{exercise}
\sel[3]{105}
Aπό	100	Kg	σταφύλια	βγαίνουν	80	Kg	μούστος.	Ένας	αμπελουργός	θέλει	να	γεμίσει	με	μούστο	6	βαρέλια,	των	350	Kg	το	καθένα.	Πόσα	K	σταφύλια,	της	ίδιας	ποιότητας,	πρέπει	να	πατήσει;
\end{exercise}
\begin{lstlisting}
>>> analoga(80,100,6*350)
2625.0
\end{lstlisting}
\begin{exercise}
\sel[4]{105}
Δύο	εργάτες	δούλεψαν	σε	μια	οικοδομή	και	πήραν	μαζί	270€ .	O	πρώτος	δούλεψε	4	ημέρες	και	ο	δεύτερος	5	ημέρες.	Πόσα	χρήματα	αντιστοιχούν	στον	καθένα.
\end{exercise}
\begin{lstlisting}
>>> analoga(9,270,4)
120.0
>>> analoga(9,270,5)
150.0
\end{lstlisting}
Και όντως $120+150=270$.
\begin{exercise}
\sel[5]{105}
Το	θαλασσινό	νερό	περιέχει	αλάτι	σε	ποσοστό	3\%.	Πόσα	κιλά	θαλασσινό	νερό	πρέπει	να	εξατμιστούν	για	να	πάρουμε	60Kg	αλάτι;
\end{exercise}
\begin{lstlisting}
>>> analoga(3,100,60)
2000.0000000000002
\end{lstlisting}
Άρα 2 τόνοι θαλασσινό νερό.
\begin{exercise}
\sel[6]{105}
Ένας	γεωργός	είχε	ένα	χωράφι	7	στρέμματα	και	πήρε	και	το	γειτονικό	χωράφι	εμβαδού	8	στρεμμάτων,	για	να	φυτέψει	καλαμπόκι.	Η	συμφωνία	με	το	γείτονά	του	ήταν	να	του	δώσει	το	15	της	παραγωγής	του	χωραφιού	του.	Η	συνολική	παραγωγή	ήταν	14	τόνοι	καλαμπόκι.	Πόσους	τόνους	θα	πάρει	ο	γεωργός	και	πόσους	ο	γείτονάς	του;
\end{exercise}
\begin{lstlisting}
>>> xwrafi8 = analoga(7+8,14000,8)
>>> analoga(100,15,xwrafi8)
1120.0
\end{lstlisting}
\begin{exercise}
\sel[7]{105}
Αν	ψήσουμε	2,5	Κ	ωμό	κρέας	θα	μείνει	1,9	K	ψημένο	κρέας. (α)		Πόσο	είναι	το	ποσοστό	απώλειας	που	έχουμε; (β)		Πόσο	κρέας	πρέπει	να	ψήσουμε	για	να	έχουμε	2,3	K	ψημένο	κρέας;
\end{exercise}
\begin{lstlisting}
>>> analoga(2.5,2.5-1.9,100)
24.000000000000004
>>> analoga(1.9,2.5,2.3)
3.026315789473684
\end{lstlisting}
Άρα 3Kg ωμό κρέας.
\begin{exercise}
\sel[8]{105}Η	μηνιαία	κάρτα	απεριορίστων	διαδρομών	στοιχίζει	12	Q	και	η	τιμή	της	θα	αυξηθεί,	κατά	75.	Το	εισιτήριο	στο	αστικό	λεωφορείο	είναι	0,7	Q	και	θα	αυξηθεί,	κατά	50.	Ένας	εργαζόμενος	παίρνει	λεωφορείο,	για	να	πάει	και	να	γυρίσει	από	τη	δουλειά	του	κάθε	ημέρα,	για	είκοσι	φορές	το	μήνα.	Τον	συμφέρει	η	χρήση	της	κάρτας	ή	όχι;
\end{exercise}
Το κόστος της κάρτας θα είναι:
\begin{lstlisting}
>>> analoga(100,175,12)
21.0
\end{lstlisting}
Το κόστος του εισιτηρίου θα είναι:
\begin{lstlisting}
>>> analoga(100,150,0.7)
1.0499999999999998
\end{lstlisting}
1.05€
και το κόστος των 20 εισιτηρίων θα είναι:
\begin{lstlisting}
>>> 20*1.05
21.0
\end{lstlisting}
Άρα ίδιο κόστος με αυτό της κάρτας.

\begin{exercise}
\sel[9]{105}Ένα	κεφάλαιο	δίνει	τόκο	1.000	Q	το	χρόνο,	με	επιτόκιο	10.	Αν	το	επιτόκιο	μειωθεί	κατά	20,	πόσο	τοις	εκατό	πρέπει	ν’	αυξήσουμε	το	κεφάλαιό	μας	για	να	έχουμε	τον	ίδιο	τόκο,	παρά	τη	μείωση	του	επιτοκίου;
\end{exercise}
Το αρχικό κεφάλαιο είναι:
\begin{lstlisting}
>>> analoga(10,100,1000)
10000.0
\end{lstlisting}
Το νέο επιτόκιο είναι:
\begin{lstlisting}
>>> analoga(100,80,10)
8.0
\end{lstlisting}
Οπότε το νέο κεφάλαιο θα πρέπει να είναι:
\begin{lstlisting}
>>> analoga(8,100,1000)
12500.0
\end{lstlisting}
και το ποσοστό αύξησης του κεφαλαίο θα είναι:
\begin{lstlisting}
>>> analoga(10000,2500,100)
25.0
\end{lstlisting}
\begin{exercise}
\sel[10]{105} Συμπλήρωσε	τον	παρακάτω	πίνακα	και	σχεδίασε	διάγραμμα	που	αντιστοιχεί	στα	δεδομένα	του	προβλήματος.
\begin{table}
\begin{tabular}{|c|c|c|c|c|c|c|c|}\hline
 &ΣΥΝΟΛΟ & Με 0 παιδιά & Με 1 παιδί & Με 2 παιδιά & Με 3 παιδιά & Με 4 παιδιά & Πάνω από 5 παιδιά     \\\hline
Οικογένειες &200 & 10 & 40 & 80 & 50 & 15 & 5     \\\hline
Ποσοστά & 100\%& &  &  & &  &     \\\hline
\end{tabular}
\end{table}
\end{exercise}
\begin{lstlisting}
>>> analoga(200,10,100)
5.0
>>> analoga(200,40,100)
20.0
>>> analoga(200,80,100)
40.0
>>> analoga(200,50,100)
25.0
>>> analoga(200,15,100)
7.5
>>> analoga(200,5,100)
2.5
\end{lstlisting}
και ο πίνακας γίνεται:
\begin{table}
\begin{tabular}{|c|c|c|c|c|c|c|c|}\hline
 &ΣΥΝΟΛΟ & Με 0 παιδιά & Με 1 παιδί & Με 2 παιδιά & Με 3 παιδιά & Με 4 παιδιά & Πάνω από 5 παιδιά     \\\hline
Οικογένειες &200 & 10 & 40 & 80 & 50 & 15 & 5     \\\hline
Ποσοστά & 100\%&5 & 20 &40  &25 &7.5  & 2.5    \\\hline
\end{tabular}
\end{table}
Μια επαλήθευση δίνει:
\begin{lstlisting}
>>> 5 + 20 +40  +25 +7.5  + 2.5   
100.0
\end{lstlisting}
\section{Αντιστρόφως ανάλογα ποσά}
\begin{exercise}
\sel{106}
Ξεκινούν ταυτόχρονα από μια πόλη:  
(α)  ένα αυτοκίνητο που τρέχει με ταχύτητα 120 kmh  

(β)  ένα αεροπλάνο με 600 Kmh  

(γ)  μία μοτοσικλέτα με 75 Kmh  

(δ)  ένα λεωφορείο που τρέχει με 80 Kmh  

(ε)  ένα ελικόπτερο με 300 Kmh 

(στ)  ένα ταξί με 100 Kmh  

(ζ)  μία βέσπα με 60 Kmh και  

(η)  ένα πούλμαν με 90 Kmh

To τέλος της διαδρομής είναι μια άλλη πόλη, που απέχει 600 Km.
\begin{itemize}
 \item Βρες σε πόσες ώρες, θα φθάσει το καθένα στον προορισμό του και συμπλήρωσε  τον παρακάτω πίνακα:
 \begin{table}
 \begin{tabular}{|c|c|c|c|c|c|c|c|c|}
 Ταχύτητα σε Km/h&&&&&&&&\\\hline
 Χρόνος σε ώρες&&&&&&&&\\\hline
 \end{tabular}
 \end{table}
 \item Ποια σχέση συνδέει τα μεγέθη της ταχύτητας και του χρόνου;
 \item Toποθέτησε τα ζεύγη των τιμών που βρήκες, σε ένα σύστημα ημιαξόνων και  ένωσε τα σημεία, που ορίζουν τα ζεύγη αυτά, με μία γραμμή. Τι παρατηρείς;
\end{itemize}
\end{exercise}
\begin{lstlisting}
>>> 600/120
5.0
>>> 600/600
1.0
>>> 600/75
8.0
>>> 600/80
7.5
>>> 600/300
2.0
>>> 600/100
6.0
>>> 600/60
10.0
>>> 600/90
6.666666666666667
\end{lstlisting}
Οπότε ο πίνακας γίνεται:
\begin{table}
 \begin{tabular}{|c|c|c|c|c|c|c|c|c|}
 Ταχύτητα σε Km/h&120&600&75&80&300&100&60&90\\\hline
 Χρόνος σε ώρες&5&1&8&7.5&2&6&10&6,66\\\hline
 \end{tabular}
 \end{table}

Τα ποσά είναι αντιστρόφως ανάλογα.

\begin{lstlisting}
import matplotlib.pyplot as plt

plt.clf()
points = [(120,5), (600,1), (75,8), (80,7.5), (300,2), (100,6), (60,10), (90,6.66)]
x = [p[0] for p in points]
y = [p[1] for p in points]
color=['m','g','r','b','c']
plt.grid()
plt.scatter(x,y, s=100 ,marker='o', c=color)

plt.show()
\end{lstlisting}
\begin{figure}
\includegraphics{graph4.png}
\caption{Χρόνος ως προς την ταχύτητα}
\end{figure}
\begin{exercise}
\sel{106}
Ένα συνεργείο που αποτελείται από 8 εργάτες χρειάζεται 30 ημέρες για να ολοκληρώσει ένα οικοδομικό έργο. 
\begin{itemize}
	\item  Πόσες ημέρες θα χρειαστεί το συνεργείο,  που αποτελείται από 2, 4, 6, 10, 12, 24 ή 48  εργάτες για να τελειώσει το ίδιο έργο;
	\item Μπορείς να συμπληρώσεις τον παρακάτω  πίνακα;
	\begin{table}
	\begin{tabular}{|c|c|c|c|c|c|c|c|c|}\hline
	Εργάτες συνεργείου &2    & 4& 6& 8& 10& 12& 24& 48\\\hline
	Ημέρες εργασίας       & &    &   & 30 & & & & \\\hline
	\end{tabular}
	\end{table}
\item  Τι παρατηρείς για το γινόμενο ``εργάτες'' $\cdot$ ``ημέρες'';
\item  Τοποθέτησε τα ζεύγη των τιμών του πίνακα, σε ένα σύστημα ημιαξόνων και  ένωσε τα σημεία, που ορίζουν τα ζεύγη αυτά, με μία γραμμή. Τι παρατηρείς;
\end{itemize}
\end{exercise} 
\begin{lstlisting}
>>> 8/2*30
120.0
>>> 8/4*30
60.0
>>> 8/6*30
40.0
>>> 8/8*30
30.0
>>> 8/10*30
24.0
>>> 8/12*30
20.0
>>> 8/24*30
10.0
>>> 8/48*30
5.0
>>>
\end{lstlisting}
Ο πίνακας γίνεται:
	\begin{table}
	\begin{tabular}{|c|c|c|c|c|c|c|c|c|}\hline
	Εργάτες συνεργείου &2    & 4 & 6& 8& 10& 12& 24& 48\\\hline
	Ημέρες εργασίας       &120 &60& 40   & 30 &24 & 20&10 &5 \\\hline
	\end{tabular}
	\end{table}

\begin{lstlisting}
>>> erg = [2,4,6,8,10,12,24,48]
>>> mer = [120,60,40,30,24,20,10,5]
>>> for i in range(8):
...     print(erg[i]*mer[i])
...
240
240
240
240
240
240
240
240
\end{lstlisting}
Το γινόμενο είναι πάντα 240.
\begin{lstlisting}
import matplotlib.pyplot as plt

plt.clf()
points = [(1,120),(4,60),(6,40),(8,30),(10,24),(12,20),(24,10),(48,5)]
x = [p[0] for p in points]
y = [p[1] for p in points]
color=['m','g','r','b','c']
plt.grid()
plt.scatter(x,y, s=100 ,marker='o', c=color)

plt.show()
\end{lstlisting}
\begin{figure}
\includegraphics{graph5.png}
\end{figure}
\begin{exercise}
\sel{107}

Ένα ορθογώνιο παραλληλόγραμμο έχει διαστάσεις x και y. Aν γνωρίζεις ότι το εμβαδόν του ορθογωνίου είναι 144 m2, μπορείς να βρεις δεκατέσσερις ακέραιες τιμές των διαστάσεών του και να συμπληρώσεις τον παρακάτω πίνακα;
\begin{table}
\begin{tabular}{|c|c|c|c|c|c|c|c|c|c|c|c|c|c|c|}\hline
x & 1 & 2 & 3 &  4 & 6 & 12 & 18 & 20 & 22 & 24 & 30 &  32 & 34 & 36 \\\hline
y & & & & & & & & & & & & & &  \\\hline
\end{tabular}
\end{table}
\begin{itemize}
\item  Ποια σχέση συνδέει τις διαστάσεις του ορθογωνίου με το εμβαδόν του;
\item Τοποθέτησε τα ζεύγη των τιμών του πίνακα, σε ένα σύστημα ημιαξόνων και  ένωσε τα σημεία, που ορίζουν τα ζεύγη αυτά, με μία γραμμή. Τι παρατηρείς;
\item Ποιο ορθογώνιο, απ’ αυτά που βρήκες, έχει τη μικρότερη περίμετρο;
\end{itemize}
\end{exercise}
\begin{lstlisting}
>>> 144/1
144.0
>>> 144/2
72.0
>>> 144/3
48.0
>>> 144/4
36.0
>>> 144/6
24.0
>>> 144/12
12.0
>>> 144/18
8.0
>>> 144/20
7.2
>>> 144/22
6.545454545454546
>>> 144/24
6.0
>>> 144/30
4.8
>>> 144/32
4.5
>>> 144/34
4.235294117647059
>>> 144/36
4.0
\end{lstlisting}
\begin{table}
\begin{tabular}{|c|c|c|c|c|c|c|c|c|c|c|c|c|c|c|}\hline
x & 1 & 2 & 3 &  4 & 6 & 12 & 18 & 20 & 22 & 24 & 30 &  32 & 34 & 36 \\\hline
y &144&72&48&36&24&12&8&7.2&6.55&6&4.8&4.5&4,235&4 \\\hline
\end{tabular}
\end{table}

\begin{lstlisting}
import matplotlib.pyplot as plt

plt.clf()
x = [1,2,3,4,6,12,18,20,22,24,30,32,34,36]
y = [144/p for p in x]
plt.grid()
plt.scatter(x,y, s=100 ,marker='.', c='m')

plt.show()
\end{lstlisting}

\begin{figure}
\includegraphics{graph6.png}
\end{figure}
Για την περίμετρο μπορούμε να κάνουμε μια γραφική παράσταση για τα σημεία που έχουμε:
\begin{lstlisting}
import matplotlib.pyplot as plt

plt.clf()
x = [1,2,3,4,6,12,18,20,22,24,30,32,34,36]
y = [144/p for p in x]
perimetros = [2*p+2*144/p for p in x]
plt.grid()
plt.scatter(x,y, s=100 ,marker='.', c='m')
plt.scatter(x,perimetros,s=100,marker='.',c='r')
plt.show()
\end{lstlisting}
\begin{figure}
\includegraphics{graph7.png}
\end{figure}

Βλέπουμε ότι η μικρότερη περίμετρος προκύπτει όταν το x είναι 12.


Για την ακρίβεια μπορούμε να δούμε τη γραφική παράσταση για πολλά σημεία ως εξής:
\begin{lstlisting}
import matplotlib.pyplot as plt
from numpy import arange

plt.clf()
x = arange(1,20,0.5)
y = [144/p for p in x]
perimetros = [2*p+2*144/p for p in x]
plt.grid()
plt.scatter(x,y, s=100 ,marker='.', c='m')
plt.scatter(x,perimetros,s=100,marker='.',c='r')
plt.show()
\end{lstlisting}
\begin{figure}
\includegraphics{graph8.png}
\end{figure}
\begin{exercise}
\sel{108}
Ένας ελαιοπαραγωγός χρησιμοποιεί δοχεία των 20 lt, 15 lt, 10 lt και 5 lt, για να συσκευάσει το λάδι που παράγει. Η παραγωγή του είναι 3.600 lt. Θέλει να συσκευάσει την ίδια ποσότητα λαδιού σε κάθε μία από τις τέσσερις διαφορετικές συσκευασίες. (α)  Πόσα δοχεία χρειάζεται από κάθε είδος; (β)  Πόσο θα κοστίσει η συσκευασία της παραγωγής του αν στοιχίζει 0,4€ το δοχείο  των 20 lt, 0,3€ το δοχείο των 15 lt,  0,2€ το δοχείο των 10 lt και 0,1€ το δοχείο  των 5 lt; 
\end{exercise}
\begin{lstlisting}
posotita = 3600/4
x = [20,15,10,5]
y = []
for doxeio in x:
    plithos = posotita / doxeio
    y.append(plithos)

kostos = [0.4,0.3,0.2,0.1]
synolikoKostos = 0
for (i,p) in enumerate(y):
    synolikoKostos += kostos[i]*p

print(plithos)
print(kostos)
\end{lstlisting}
που δίνει αποτέλεσμα
\begin{lstlisting}
[45.0, 60.0, 90.0, 180.0]
72.0
\end{lstlisting}
Δηλαδή 45 δοχεία των 20lt, 60 δοχεία των 15lt, 90 δοχεία των 10lt και 180 δοχεία των 5lt. Το συνολικό κόστος για τα δοχεία έιναι 72€.
\begin{exercise}
\sel{109}
Εξέτασε τους παρακάτω πίνακες:

α)

\begin{table}
\begin{tabular}{|c|c|c|c|c|}
x&1&2&3&4\\\hline
y&2&1&$\frac{2}{3}$&$\frac{1}{2}$\\\hline
\end{tabular}
\end{table}

β)

\begin{table}
\begin{tabular}{|c|c|c|c|}
x&0,25&0,4&0,5\\\hline
y&10&6,25&5\\\hline
\end{tabular}
\end{table}

γ)

\begin{table}
\begin{tabular}{|c|c|c|c|c|}
x&$\frac{1}{100}$&$\frac{2}{58}$&$\frac{7}{10}$&4\\\hline
y&100&29&$\frac{10}{7}$&$\frac{1}{4}$\\\hline
\end{tabular}
\end{table}

δ)

\begin{table}
\begin{tabular}{|c|c|c|c|}
x&3&6&9\\\hline
y&9&5&3\\\hline
\end{tabular}
\end{table}

\end{exercise}
\sel[3]{109}
Θα φτιάξουμε μια συνάρτηση που να παίρνει σαν εισόδους δύο πίνακες με τιμές x και y και θα δίνει σαν αποτέλεσμα αν αυτές οι τιμές είναι αντιστρόφως ανάλογες ή όχι. Θα βασιστεί στο ότι αν τα ποσά είναι αντιστρόφως ανάλογα το γινόμενό τους είναι πάντα το ίδιο.
\begin{lstlisting}
def antistrofosanaloga(x,y):
    if len(x) != len(y):
        print('Τα δεδομένα δεν έχουν το ίδιο μέγεθος')
        return(None)
    ginomeno = x[0]*y[0]
    for i in range(len(x)):
        if x[i]*y[i]!=ginomeno:
            return(False)
    return(True)

print(antistrofosanaloga([1,2,3,4],[2,1,2/3,1/2]))
print(antistrofosanaloga([0.25,0.4,0.5],[10,6.25,5]))
print(antistrofosanaloga([1/100,2/58,7/10,4],[100,29,10/7,1/4]))
print(antistrofosanaloga([3,6,9],[9,5,3]))
\end{lstlisting}
που δίνει αποτέλεσμα
\begin{lstlisting}
True
True
True
False
\end{lstlisting}
Άρα οι πίνακες α,β,γ έχουν ποσά αντιστρόφως ανάλογα ενώ ο πίνακας δ όχι.
\begin{exercise}
\sel[4]{109}
Τα ποσά x και  είναι αντιστρόφως ανάλογα. 
(α) Συμπλήρωσε τον πίνακα: 

\begin{table}
\begin{tabular}{|c|c|c|c|c|c|c|c|c|c|c|c|}\hline
x&0,2&0,5&0,7&1      &      &      & 2,3&3&            &10&12\\\hline
y&       &     &      & 3,5 &2,5&1,75&      &   &0,875&     &\\\hline
\end{tabular}
\end{table}

(β)  Βρες τα σημεία που παριστάνουν κάθε ζευγάρι τιμών (x, ), σε κατάλληλο  σύστημα ορθογωνίων ημιαξόνων και σχεδίασε την 
υπερβολή. 
\end{exercise}
\begin{lstlisting}
>>> 3.5/0.2
17.5
>>> 3.5/0.5
7.0
>>> 3.5/0.7
5.0
>>> 3.5/2.5
1.4
>>> 3.5/1.75
2.0
>>> 3.5/2.3
1.5217391304347827
>>> 3.5/3
1.1666666666666667
>>> 3.5/0.875
4.0
>>> 3.5/10
0.35
>>> 3.5/12
0.2916666666666667
\end{lstlisting}

\begin{table}
\begin{tabular}{|c|c|c|c|c|c|c|c|c|c|c|c|}\hline
x&0,2 &0,5&0,7&1      &1,4& 2   & 2,3  &3        &4        &10   &12\\\hline
y&17,5&7    & 5  & 3,5 &2,5&1,75&1,52& 1,167&0,875&0,35 &0,29\\\hline
\end{tabular}
\end{table}

Για να σχεδιάσουμε φτιάχνουμε το πρόγραμμα:
\begin{lstlisting}
import matplotlib.pyplot as plt

plt.clf()
x=[0.2,0.5,0.7,1,1.4,2,2.3,3,4,10,12]
y=[17.5,7,5,3.5,2.5,1.75,1.52,1.167,0.875,0.35,0.29]
plt.grid()
plt.scatter(x,y, s=100 ,marker='.', c='m')
plt.show()
\end{lstlisting}
που δίνει το αποτέλεσμα
\begin{figure}
\includegraphics{graph9.png}
\end{figure}
\begin{exercise}
\sel[5]{109}
Για την αναδάσωση μιας πλαγιάς, εργάστηκαν 20 εργάτες για 10 ημέρες. Πόσοι εργάτες, ίδιας απόδοσης, χρειάζονται για να αναδασώσουν την έκταση αυτή, σε 8 ημέρες; 
\end{exercise}
Το γινόμενο εργάτες $\cdot$ ημέρες θα είναι σταθερό οπότε 
$$x\cdot 8 = 20\cdot 10$$
$$x = \frac{20\cdot 10}{8}$$
$$x =  25$$

\begin{lstlisting}
>>> 20*10/8
25.0
\end{lstlisting}
\begin{exercise}
\sel[6]{109}
Σε ένα αγρόκτημα, τοποθέτησαν ντομάτες σε 50 καφάσια, των 12 Kg το καθένα. Πόσα καφάσια των 20 kg θα χρειαζόντουσαν για να τοποθετήσουν τις ντομάτες. Αν κάθε καφάσι των 12 kg στοιχίζει 0,28€ και κάθε καφάσι των 20Kg 0,46€, ποια συσκευασία τους συμφέρει, ώστε να ελαχιστοποιηθεί το κόστος συσκευασίας του προϊόντος τους; 
\end{exercise}
\begin{lstlisting}
>>> ntomates = 50*12
>>> kafasia20 = ntomates/20
>>> kafasia20
30.0
>>> kostos12 = 50*0.28
>>> kostos12
14.000000000000002
>>> kostos20 = kafasia20*0.46
>>> kostos20
13.8
\end{lstlisting}
\begin{exercise}
\sel[8]{109}
Το πετρέλαιο που υπάρχει στη δεξαμενή μιας πολυκατοικίας, επαρκεί για 30 ημέρες, όταν καταναλώνονται 80 lt την ημέρα. Όταν το κρύο δυναμώνει, η ημερήσια κατανάλωση αυξάνεται, κατά 20\%. Για πόσες ημέρες θα φτάσει το πετρέλαιο;
\end{exercise}
\begin{lstlisting}
>>> neakatan = 80+80*20/100
>>> neakatan
96.0
>>> meres = 80*30/neakatan
>>> meres
25.0
\end{lstlisting}
\begin{exercise}
\sel[6]{112}
Συμπλήρωσε	τον	διπλανό	πίνακα	ανάλογων ποσών
\begin{table}
\begin{tabular}{|c|c|c|c|c|c|}
x&2&4 &  &12&16\\\hline
y&  &15&30& &\\hline
\end{tabular}
\end{table}
\end{exercise}
\begin{lstlisting}
>>> l = 15/4
>>> 2*l
7.5
>>> 30/l
8.0
>>> 12*l
45.0
>>> 16*l
60.0
\end{lstlisting}
\begin{table}
\begin{tabular}{|c|c|c|c|c|c|}
x&2   &4  &8 &12&16\\\hline
y&7.5&15&30&45 &60\\hline
\end{tabular}
\end{table}
\begin{exercise}
\sel[8]{112}
Συμπλήρωσε τον πίνακα των αντιστρόφως ανάλογων ποσών.
\begin{table}
\begin{tabular}{|c|c|c|c|c|c|}
x& 2&&&4&8\\\hline
y& 8&16&32&&\\\hline
\end{tabular}
\end{table}
\end{exercise}
\begin{lstlisting}
>>> gin = 2*8
>>> gin/16
1.0
>>> gin/32
0.5
>>> gin/4
4.0
>>> gin/8
2.0
\end{lstlisting}
\begin{table}
\begin{tabular}{|c|c|c|c|c|c|}
x& 2&  1&0.5&4&8\\\hline
y& 8&16&32&4&2\\\hline
\end{tabular}
\end{table}

%\chapter{Θετικοί και αρνητικοί αριθμοί}
\end{document}